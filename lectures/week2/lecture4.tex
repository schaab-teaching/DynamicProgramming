\documentclass[10pt]{beamer}
\usepackage{amsmath,amsfonts,microtype,nicefrac,amssymb, amsthm,centernot}
\usefonttheme{professionalfonts}
\usefonttheme{serif}

\usepackage{palatino}
\usepackage{mathpazo} % or: {newpxtext,newpxmath}
\renewcommand\familydefault{\rmdefault}

\usepackage[utf8]{inputenc}
\usepackage[T1]{fontenc}
\usepackage{textcomp}
\usepackage{bm}

\usepackage{tikz}
\usetikzlibrary{arrows.meta}
\usetikzlibrary{positioning}
\usetikzlibrary{arrows,shapes}
\usepackage{caption, subcaption}

%% BEAMER BUTTON
%\setbeamertemplate{button}{\tikz
%\node[
%	inner xsep = 2pt, 
%	draw = structure!0, 
%	fill = myblue, 
%	rounded corners = 4pt]{\color{white} \tiny\insertbuttontext};
%}


%% ALGORITHM 
\usepackage{algorithm}
\usepackage[noend]{algpseudocode}
\usepackage{multimedia}

%% THEME
\usetheme[frametitleformat=regular,titleformat=regular]{Madrid}
\setbeamerfont{frametitle}{shape=\normalfont}
\setbeamerfont{title}{shape=\normalfont}

%% PATHS
\graphicspath{{./results/}}
\makeatletter
\def\input@path{{../draft/tables/latexData/}}
\makeatother

%% FIGURE ENVIRONMENT 
\usepackage{booktabs,siunitx}

\usepackage{pgfplots} 
%\usepackage[outdir=./figures]{epstopdf}
\usepackage{epstopdf}
\usepackage{float}
\usepackage{graphicx}

\usepackage[absolute,overlay]{textpos}

%% COLORS
%% LINKS
\definecolor{myred}{RGB}{163,32,45}
\definecolor{navyblue}{rgb}{0.05,0.2,0.70}
\definecolor{myblue}{RGB}{0,51,150}
\definecolor{myorange}{RGB}{255,140,0}
\definecolor{myref}{RGB}{160,160,160}
\definecolor{shock}{RGB}{0, 125, 34}%{50, 168, 82}

%% TRANSPARENCY

\usepackage{transparent}

%% BEAMER TEMPLATE
\usetheme{Boadilla}

\makeatother
\setbeamertemplate{itemize items}{\large\raisebox{-0.25ex}{\textbullet}}
\setbeamertemplate{itemize subitem}{\footnotesize\raisebox{0.15ex}{--}}
\setbeamertemplate{itemize subsubitem}{\Tiny\raisebox{0.7ex}{$\blacktriangleright$}}
%\setbeamertemplate{itemize subsubitem}{\color{yellow}$\blacksquare$}

\setbeamertemplate{enumerate item}[default]
\setbeamertemplate{enumerate subitem}{\textbullet}
\setbeamertemplate{footline}{}
\makeatletter
\setbeamertemplate{navigation symbols}{}


% TODO:
% Introduce page number but no other footer
% Define appropriate itemize and enumerate environments so I never have to touch these
% What about an automatic subitemize environment? 


%% FORMATTING AUTHORS
%\usepackage{authblk}
\usepackage{url}
\usepackage{multirow}
\usepackage{array}

%% FRAME ITEMIZE SPACING
\usepackage{xpatch}

\xpatchcmd{\itemize}
{\def\makelabel}
{\setlength{\itemsep}{2.5ex}\def\makelabel}
{}
{}

\xpatchcmd{\enumerate}
{\def\makelabel}
{\setlength{\itemsep}{10ex}\def\makelabel}
{}
{}

%% APPENDIX 
\usepackage{appendixnumberbeamer}


%% TITLE AND OPENING
\title{\large M2: Lecture 4}
\subtitle{Dynamic Programming in Discrete Time (II)}
\author{Andreas Schaab}
\date{}


\begin{document}
\tikzstyle{every picture}+=[remember picture]
%\everymath{\displaystyle}
\thispagestyle{empty}
\maketitle 
\newpage

\addtocounter{framenumber}{-1}




%%%%%%%%%%%%%%%%%%%%%%%%%%  SLIDE   %%%%%%%%%%%%%%%%%%%%%%%%%%%%%%%%
\begin{frame}{Outline of today's lecture}
\begin{enumerate}
\item Example: growth model with labor supply
\item Stochastic dynamic programming
\item History notation
\item Stochastic growth model
\end{enumerate}
\end{frame}



%%%%%%%%%%%%%%%%%%%%%%%%%%  SLIDE   %%%%%%%%%%%%%%%%%%%%%%%%%%%%%%%%
\begin{frame}{1. Example: growth model with labor supply}
\begin{itemize}
\item Time is discrete with $t = 0, 1, \ldots$ and there is no uncertainty

\item Representative household consumes ($c_t$), saves and works ($l_t$)

\item Preferences over consumption and work given by
\begin{equation*}
	V(k_0) = \max_{ \{ c_t, l_t \}_{t=0}^\infty } \sum_{t = 0}^\infty \beta^t u(c_t, l_t)
\end{equation*}

\item We will use constant relative risk aversion (CRRA) preferences:
\begin{equation*}
	u(c, l) = \frac{1}{1-\gamma} c^{1-\gamma} - \frac{1}{1+\phi}l^{1+\phi}
\end{equation*}

\item Budget constraint: household owns capital and rents it at rate $r_t$
\begin{equation*}
	k_{t+1} = (1 + r_t - \delta) k_t + w_t l_t - c_t
\end{equation*}
\end{itemize}
\end{frame}


%%%%%%%%%%%%%%%%%%%%%%%%%%  SLIDE   %%%%%%%%%%%%%%%%%%%%%%%%%%%%%%%%
\begin{frame}{}
\textbf{Definition} (household problem): The representative household chooses sequences $\{c_t, l_t\}_{t=0}^\infty$ to maximize lifetime value $V(k_0)$ subject to the budget constraint, given initial capital $k_0$ and taking as given paths of prices $\{r_t, w_t\}_{t=0}^\infty$

\vspace{6mm}
\begin{itemize}
\item Household problem in sequence form associated with Lagrangian:
\begin{equation*}
	L(k_0) = \sum_{t = 0}^\infty \beta^t \bigg[ u(c_t, l_t) + \lambda_t \bigg( (1 + r_t - \delta) k_t + w_t l_t - c_t - k_{t+1} \bigg) \bigg]
\end{equation*}

\item FOCs for $c_t$, $l_t$, and $k_{t+1}$:
\begin{align*}
	0 &= u_c(c_t, l_t) - \lambda_t \\
	0 &= u_l(c_t, l_t) + w_t \lambda_t \\
	0 &= - \lambda_t + \beta \lambda_{t+1} (1 + r_t - \delta)
\end{align*}

\item Euler equation for consumption: $u_c(c_t, l_t) = \beta (1 + r_t - \delta) u_c(c_{t+1}, l_{t+1})$

\item Optimal labor supply: $MRS = - \frac{u_l(c_t, l_t)}{u_c(c_t, l_t)} = w_t$
\end{itemize}
\end{frame}



%%%%%%%%%%%%%%%%%%%%%%%%%%  SLIDE   %%%%%%%%%%%%%%%%%%%%%%%%%%%%%%%%
\begin{frame}{}
\begin{itemize}
\item Recursive representation (show this):
\begin{equation*}
	V(k) = \max_{c, l} \bigg\{ u(c, l) + \beta V((1 + r - \delta) k + w l - c) \bigg\}
\end{equation*}

\item FOCs:
\begin{align*}
	u_c &= \beta V_k(k') \\
	u_l &= - \beta w V_k(k')
\end{align*}
which again implies $- u_l / u_c = w$

\item Use the envelope condition to derive the consumption Euler equation

\item Plug in for CRRA utility and interpret Euler equation and optimal labor supply condition
\end{itemize}
\end{frame}



%%%%%%%%%%%%%%%%%%%%%%%%%%  SLIDE   %%%%%%%%%%%%%%%%%%%%%%%%%%%%%%%%
\begin{frame}{2. Stochastic dynamic programming}
\begin{itemize}
\item Follow Ljungqvist-Sargent notation, Chapter 3.2 

\item Under uncertainty, household problem takes the form 
\begin{equation*}
	\max_{ \{c_t\} } \mathbb E_0 \sum_{t=0}^\infty \beta^t u(c_t)
\end{equation*}
subject to $k_{t+1} = g(k_t, c_t, \epsilon_{t+1})$ \hspace{1mm} (\textit{first-order stochastic difference equation})

\item $\{\epsilon_t\}_{t=0}^\infty$ is sequence of iid random variables (\textit{stochastic process})

\item Initial condition $x_0$ given
\end{itemize}
\end{frame}


%%%%%%%%%%%%%%%%%%%%%%%%%%  SLIDE   %%%%%%%%%%%%%%%%%%%%%%%%%%%%%%%%
\begin{frame}{}
\begin{itemize}
\item Dynamic programming approach: we again look for recursive representation on state space $k_t \in \mathcal X$

\item The problem is to look for a \textit{policy function} $c(k)$ that solves 
\begin{equation*}
	V(k) = \max_{c} \bigg\{ u(c) + \beta \mathbb E \bigg[ V\Big( g(k, c, \epsilon) \Big) \, \Big| \, k \bigg] \bigg\}
\end{equation*}
where $\mathbb E[V(\cdot) \mid k] = \int V(\cdot) dF(\epsilon)$

\item $V(k)$ is the (lifetime) value that an agent obtains from solving this problem starting from $k$

\item FOC that characterizes the consumption policy function $c(k)$ is 
\begin{equation*}
	0 = u'(c(k)) + \beta \mathbb E \Big\{ \partial_k V \Big( g(k, c(k), \epsilon) \Big) \cdot \partial_c g(k, c(k), \epsilon) \, \Big| \, k \Big\} = 0
\end{equation*}
\end{itemize}
\end{frame}



%%%%%%%%%%%%%%%%%%%%%%%%%%  SLIDE   %%%%%%%%%%%%%%%%%%%%%%%%%%%%%%%%
\begin{frame}{3. History notation}
\begin{itemize}
\item A very popular approach to deal with uncertainty in macro is to use history notation (Ljungqvist-Sarget, e.g., chapters 8, 12)

\item Economy populated by individuals, indexed by $i \in I$

\item Time is discrete and indexed by $t = 0, 1, \ldots$

\item At every $t$, there is a realization of a stochastic event $s_t \in \mathcal S$

\item We denote the \textbf{history} of such events up to $t$ by $s^t = \{s_0, s_1, \ldots, s_t\}$

\item The unconditional probability of history $s^t$ is given by $\pi_t(s^t \mid s_0)$

\item If Markov, $\pi_t \Big( s^t \, | \, s_0 \Big) = \pi(s_t \mid s_{t-1}) \pi(s_{t-1} \mid s_{t-2}) \ldots \pi(s_0)$

\item Single consumption good (dollars) as numeraire
\end{itemize}
\end{frame}



%%%%%%%%%%%%%%%%%%%%%%%%%%  SLIDE   %%%%%%%%%%%%%%%%%%%%%%%%%%%%%%%%
\begin{frame}{}
\begin{itemize}
\item The \textbf{lifetime value} of individual $i$ is then defined as 
\begin{align*}
	V_i(s_0) = \sum_{t=0}^T \Big( \beta_i \Big)^t \sum_{s^t} \pi_t \Big( s^t \mid s_0 \Big) u_i \Big( c_t^i \Big( s^t \Big), n_t^i \Big(s^t \Big) \Big)
\end{align*}

\item \textit{Generalizations}: heterogeneous beliefs, general preferences (Epstein-Zin), recursive formulation, multiple commodities, intergenerational considerations

\item Suppose $c_t^i(\cdot)$ and $l_t^i(\cdot)$ are functions of some primitive (policy) $\theta$

\end{itemize}
\end{frame}


%%%%%%%%%%%%%%%%%%%%%%%%%%  SLIDE   %%%%%%%%%%%%%%%%%%%%%%%%%%%%%%%%
\begin{frame}{}
\begin{itemize}
\item Consider policy experiment $d \theta$, then $i$'s \textbf{private welfare assessment} is
\begin{align*}
	\frac{ d V_i(s_0) }{d \theta} = \sum_{t=0}^T \Big( \beta_i \Big)^t \sum_{s^t} \pi_t \Big( s^t \mid s_0 \Big)  \frac{\partial u_i (s^t)}{\partial c_t^i}  {\color{myorange} \frac{ d u_{i \mid c} (s^t)}{d \theta}}
\end{align*}

\vspace{4mm}
{\color{myorange} Instantaneous consumption-equivalent effect} of policy $d \theta$ at date ${\color{myorange} t}$, history ${\color{myorange} s^t}$ on individual ${\color{myorange} i}$:
\vspace*{0mm}
\begin{align*}
	{\color{myorange} \frac{ d u_{i \mid c} (s^t)}{d \theta}} \equiv \frac{ \frac{d u_i (c_t^i(s^t), n_t^i(s^t))}{d \theta} }{ \frac{\partial u_i (s^t)}{\partial c_t^i}}  = \underbrace{ \frac{d c_t^i (s^t)}{d \theta} + \frac{ \frac{\partial u_i (s^t)}{\partial n_t^i} }{ \frac{\partial u_i (s^t)}{\partial c_t^i}}  \frac{d n_t^i (s^t)}{d \theta} }_\text{in {\color{myorange} \textit{consumption} units}}
\end{align*}

\item Example: $d \theta$ gives $i$ marginal dollar at $s^t$, then
\begin{equation*}
	\frac{dV_i(s_0)}{d \theta} = (\beta_i)^t \pi_t(s^t \mid s_0) \frac{\partial u_i(s^t)}{\partial c_t^i}
\end{equation*}
\end{itemize}
\end{frame}

%%%%%%%%%%%%%%%%%%%%%%%%%%  SLIDE   %%%%%%%%%%%%%%%%%%%%%%%%%%%%%%%%
\begin{frame}{}
\vspace{4mm}
{\color{myblue} \textbf{Conventional benchmarks}}:
\begin{itemize}
	{\small
		
		\vspace{2mm}
		\item Policy $d \theta$ is {\color{myorange} Pareto improving} if $\frac{ d V_i(s_0) }{d \theta} \geq 0$ for all $i$, strictly for some $i$
		
		\vspace{0mm}
		\item {\color{myorange}Welfarist planners}: \hspace{0.5mm} SWF given by \hspace{0.2mm} $\mathcal{W}\left(\left\{ V_{i}\left(s_{0}\right)\right\} _{i\in I}\right)$
		
		\vspace{0.5mm}
		{\footnotesize \textit{Utilitarian, Isoelastic, Rawlsian, Nash, Dictator, ...}}
		
		\vspace{0mm}
		\item Policy $d \theta$ is desirable for a {\color{myorange} welfarist planner} if
		\vspace*{-1mm}
		\begin{equation*}
			\int \lambda_{i}\left(s_{0}\right) \frac{ d V_i(s_0) }{d \theta} d i > 0 \, , \hspace{10mm}  \text{where} \hspace{2mm} \lambda_{i}\left(s_{0}\right)=\frac{\partial\mathcal{W}\left(\left\{ V_{i}\left(s_{0}\right)\right\} _{i\in I}\right)}{\partial V_{i}}
		\end{equation*}
		}
\end{itemize}

\vspace{6mm}
{\color{navyblue} \textbf{Remarks}}:
{\small
\begin{itemize}
\item All welfarist planners agree when individuals ex-ante ($s_0$) homogeneous

\vspace{-1mm}
\item How to make welfare assessments with \textit{heterogeneous} individuals?

\vspace{-1mm}
\item If you're interested, see D\'avila-Schaab (2022)
\end{itemize}
}
\end{frame}



%%%%%%%%%%%%%%%%%%%%%%%%%%  SLIDE   %%%%%%%%%%%%%%%%%%%%%%%%%%%%%%%%
\begin{frame}{4. Stochastic Growth Model}
\begin{itemize}
\item Discrete time: $t \in \{0, 1, \ldots, T\}$, where $T \leq \infty$

\item At $t$, event $s_t \in \mathcal{S}$ is realized; history $s^t = (s_0, \ldots, s_t)$ has probability $\pi_t(s^t)$

\item Representative household has preferences of paths of consumption $c_t(s^t)$ and labor $l_t(s^t)$ 
\begin{equation*}
	\sum_{t=0}^\infty \beta^t \sum_{s^t} \pi_t\Big(s^t \Big) u \Big( c_t(s^t), l_t(s^t) \Big)
\end{equation*}

\item Inada conditions $\lim_{c\to 0} u_c(c, l) = \lim_{l\to0} u_l (c, l) = \infty$

\item At $t=0$, household endowed with $k_0$
\end{itemize}
\end{frame}


%%%%%%%%%%%%%%%%%%%%%%%%%%  SLIDE   %%%%%%%%%%%%%%%%%%%%%%%%%%%%%%%%
\begin{frame}{}
\begin{itemize}
\item Technology, capital accumulation, and budget / resource constraint:
\begin{align*}
	c_t(s^t) + \iota_t(s^t) &\leq A_t(s^t) F(k_t(s^{t-1}), l_t(s^t)) \\
	k_{t+1}(s^t) &= (1-\delta) k_t(s^{t-1}) + \iota_t(s^t)
\end{align*}

\item $F(\cdot)$ is twice continuously differentiable and constant returns to scale

\item Source of uncertainty is stochastic process for TFP $A_t(s^t)$

\item Standard regularity conditions on $F(\cdot)$ (see LS)
\end{itemize}
\end{frame}




%%%%%%%%%%%%%%%%%%%%%%%%%%  SLIDE   %%%%%%%%%%%%%%%%%%%%%%%%%%%%%%%%
\begin{frame}{Lagrangian approach to sequence problem}
\begin{itemize}
\item Form Lagrangian 
\begin{align*}
	L = &\sum_{t=0}^\infty \sum_{s^t} \beta^t \pi_t(s^t) \bigg\{ u (c_t(s^t), l_t(s^t)) \\
	&+ \lambda_t(s^t) \bigg[ A_t(s^t) F(k_t(s^{t-1}), l_t(s^t)) - c_t(s^t) + (1-\delta) k_t(s^{t-1})  - k_{t+1}(s^t) \bigg] \bigg\}
\end{align*}

\item FOCs for $c_t(s^t)$, $l_t(s^t)$ and $k_{t+1}(s^t)$ are given by
\begin{align*}
	u_c(s^t) &= \lambda_t(s^t) \\
	u_l(s^t) &= u_c(s^t) A_t(s^t) F_n(s^t) \\
	u_c(s^t) \pi_t(s^t) &= \beta \sum_{s^{t+1} \mid s^t} u_c(s^{t+1}) \pi_{t+1}(s^{t+1}) \bigg[ A_{t+1} (s^{t+1}) F_k (s^{t+1}) + (1-\delta) \bigg]
\end{align*}

\item Summation over $(s^{t+1} \mid s^t)$ is like conditional expectation (summing over histories that branch out from $s^t$)
\end{itemize}
\end{frame}



%%%%%%%%%%%%%%%%%%%%%%%%%%  SLIDE   %%%%%%%%%%%%%%%%%%%%%%%%%%%%%%%%
\begin{frame}{Recursive representation: dynamic programming}
\begin{itemize}
\item Assume time-homogeneous Markov process:
\begin{equation*}
	\mathbb E_t (A_{t+1}) = \mathbb E \Big[ A(s^{t+1}) \mid s^t \Big] = \mathbb E \Big[ A(s_{t+1}) \mid s_t \Big] = \sum_{s'} \pi(s' \mid s_t) A(s')
\end{equation*}

\item Drop $t$ subscripts: $s$ is current state, $s'$ denotes next period's draw

\item Denote by $X_t$ the \textit{endogenous state} of the problem: for now, assume there is such a representation (Fabrice's part will be all about this)

\item Intuitively: $s$ is the exogenous state and $X$ is the endogenous state 

\item Bellman equation can be written: 
\begin{equation*}
	V(X, s) = \max_{c, l} \bigg\{ u(c, l) + \beta \sum_{s'} \pi(s' \mid s) V(X', s') \bigg\}
\end{equation*}
subject to $X' = g(X, c, l, s, s')$
\end{itemize}
\end{frame}






%%%%%%%%%%%%%%%%%%%%%%%%%%%%%%%%%%%%%%%%%%%%%%%%%%%%%%%%%%%%%%
%%%%%%%%%%%%%%%%%%%%%%%%%%%%%%%%%%%%%%%%%%%%%%%%%%%%%%%%%%%%%%
%%%%%%%%%%%%%%%%%%%%%%%%%%%%%%%%%%%%%%%%%%%%%%%%%%%%%%%%%%%%%%

\appendix


\end{document}









