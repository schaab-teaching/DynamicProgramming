\documentclass[10pt]{beamer}
\usepackage{amsmath,amsfonts,microtype,nicefrac,amssymb, amsthm,centernot}
\usefonttheme{professionalfonts}
\usefonttheme{serif}

\usepackage{palatino}
\usepackage{mathpazo} % or: {newpxtext,newpxmath}
\renewcommand\familydefault{\rmdefault}

\usepackage[utf8]{inputenc}
\usepackage[T1]{fontenc}
\usepackage{textcomp}
\usepackage{bm}

\usepackage{tikz}
\usetikzlibrary{arrows.meta}
\usetikzlibrary{positioning}
\usetikzlibrary{arrows,shapes}
\usepackage{caption, subcaption}

%% BEAMER BUTTON
%\setbeamertemplate{button}{\tikz
%\node[
%	inner xsep = 2pt, 
%	draw = structure!0, 
%	fill = myblue, 
%	rounded corners = 4pt]{\color{white} \tiny\insertbuttontext};
%}


%% ALGORITHM 
\usepackage{algorithm}
\usepackage[noend]{algpseudocode}
\usepackage{multimedia}

%% THEME
\usetheme[frametitleformat=regular,titleformat=regular]{Madrid}
\setbeamerfont{frametitle}{shape=\normalfont}
\setbeamerfont{title}{shape=\normalfont}

%% PATHS
\graphicspath{{./results/}}
\makeatletter
\def\input@path{{../draft/tables/latexData/}}
\makeatother

%% FIGURE ENVIRONMENT 
\usepackage{booktabs,siunitx}

\usepackage{pgfplots} 
%\usepackage[outdir=./figures]{epstopdf}
\usepackage{epstopdf}
\usepackage{float}
\usepackage{graphicx}

\usepackage[absolute,overlay]{textpos}

%% COLORS
%% LINKS
\definecolor{myred}{RGB}{163,32,45}
\definecolor{navyblue}{rgb}{0.05,0.2,0.70}
\definecolor{myblue}{RGB}{0,51,150}
\definecolor{myorange}{RGB}{255,140,0}
\definecolor{myref}{RGB}{160,160,160}
\definecolor{shock}{RGB}{0, 125, 34}%{50, 168, 82}

%% TRANSPARENCY

\usepackage{transparent}

%% BEAMER TEMPLATE
\usetheme{Boadilla}

\makeatother
\setbeamertemplate{itemize items}{\large\raisebox{-0.25ex}{\textbullet}}
\setbeamertemplate{itemize subitem}{\footnotesize\raisebox{0.15ex}{--}}
\setbeamertemplate{itemize subsubitem}{\Tiny\raisebox{0.7ex}{$\blacktriangleright$}}
%\setbeamertemplate{itemize subsubitem}{\color{yellow}$\blacksquare$}

\setbeamertemplate{enumerate item}[default]
\setbeamertemplate{enumerate subitem}{\textbullet}
\setbeamertemplate{footline}{}
\makeatletter
\setbeamertemplate{navigation symbols}{}


%% FORMATTING AUTHORS
%\usepackage{authblk}
\usepackage{url}
\usepackage{multirow}
\usepackage{array}

%% FRAME ITEMIZE SPACING
\usepackage{xpatch}

\xpatchcmd{\itemize}
{\def\makelabel}
{\setlength{\itemsep}{2.5ex}\def\makelabel}
{}
{}

\xpatchcmd{\enumerate}
{\def\makelabel}
{\setlength{\itemsep}{10ex}\def\makelabel}
{}
{}

%% APPENDIX 
\usepackage{appendixnumberbeamer}


%% TITLE AND OPENING
\title{\large M2: Lecture 12}
\subtitle{Portfolio Choice and Asset Pricing}
\author{Andreas Schaab}
\date{}


\begin{document}
\tikzstyle{every picture}+=[remember picture]
%\everymath{\displaystyle}
\thispagestyle{empty}
\maketitle 
\newpage

\addtocounter{framenumber}{-1}




%%%%%%%%%%%%%%%%%%%%%%%%%%  SLIDE   %%%%%%%%%%%%%%%%%%%%%%%%%%%%%%%%
\begin{frame}{Outline of today's lecture}
\addtocounter{framenumber}{-1}

\begin{enumerate}
\item Equity premium

\item Merton model: portfolio choice in continuous time
\end{enumerate}
\end{frame}



%%%%%%%%%%%%%%%%%%%%%%%%%%  SLIDE   %%%%%%%%%%%%%%%%%%%%%%%%%%%%%%%%
\begin{frame}{1. Equity premium}
\begin{itemize}
\item Consider $i \in \mathcal I$ assets with stochastic return processes $\{ R_{i, t+1} \}$, with 
\begin{equation*}
	R_{i, t+1} = e^{ r_{i, t+1} + \sigma_i \epsilon_{i, t+1} - \frac{1}{2} \sigma_i^2}
\end{equation*}
where $\epsilon_{i, t} \sim \mathcal N(0, 1)$

\item For any asset $i$ the household can freely trade, an Euler equation must hold: 
\begin{equation*}
	u'(c_t) = \beta \mathbb{E}_t \Big( R_{i, t+1} u'(c_{t+1}) \Big)
\end{equation*}

\item Using CRRA with $u(c) = \frac{1}{1-\gamma} c^{1-\gamma}$, we get
\begin{equation*}
	1 = \mathbb E_t \bigg[ e^{- \rho + r_{i, t+1} + \sigma_i \epsilon_{i, t+1} - \frac{1}{2} \sigma_i^2} \bigg( \frac{ c_{t+1}}{c_t} \bigg) ^{-\gamma} \bigg]
\end{equation*}
\end{itemize}
\end{frame}


%%%%%%%%%%%%%%%%%%%%%%%%%%  SLIDE   %%%%%%%%%%%%%%%%%%%%%%%%%%%%%%%%
\begin{frame}{}
\begin{itemize}
\item Note that
\begin{equation*}
	\bigg( \frac{ c_{t+1}}{c_t} \bigg) ^{-\gamma} = \exp \bigg\{ - \gamma \log \frac{c_{t+1}}{c_t} \bigg\} = e^{- \gamma \Delta \log(c_{t+1}) }
\end{equation*}

\item Also, for $X \sim \mathcal N(\mu, \sigma)$, we have $\mathbb{E} [ e^{b x} ] = e^{ b \mu + \frac{1}{2} b^2 \sigma^2}$

\item Assuming log-normal consumption growth, we have
\begin{equation*}
	\mathbb{E}_t \bigg[ e^{ \sigma_i \epsilon_{i, t+1} - \gamma \Delta \ln(c_{t+1}) } \bigg] = e^{- \gamma \mathbb{E}_t \Delta \ln(c_{t+1}) + \frac{1}{2} \mathbb Var ( \sigma_i \epsilon_{i, t+1} - \gamma \Delta \ln(c_{t+1}) )}
\end{equation*}

\item Putting it all together:
\begin{align*}
	1 &= e^{- \rho + r_{i, t+1} - \frac{1}{2} \sigma_i^2} \mathbb E_t \bigg[ e^{\sigma_i \epsilon_{i, t+1} - \gamma \Delta \log(c_{t+1}) } \bigg] \\
	&= e^{- \rho + r_{i, t+1} - \frac{1}{2} \sigma_i^2 - \gamma \mathbb{E}_t \Delta \ln(c_{t+1}) + \frac{1}{2} \mathbb Var ( \sigma_i \epsilon_{i, t+1} - \gamma \Delta \ln(c_{t+1}) )}
\end{align*}
\end{itemize}
\end{frame}



%%%%%%%%%%%%%%%%%%%%%%%%%%  SLIDE   %%%%%%%%%%%%%%%%%%%%%%%%%%%%%%%%
\begin{frame}{}
\begin{itemize}
\item Read about variance and covariance

\item We have 
\begin{equation*}
	\mathbb Var (A + B) = \mathbb Var (A) +  \mathbb Var (B) + 2 \mathbb Cov (A, B)
\end{equation*}

\item So
\begin{align*}
	1 &= e^{- \rho + r_{i, t+1} - \frac{1}{2} \sigma_i^2 - \gamma \mathbb{E}_t \Delta \ln(c_{t+1}) + \frac{1}{2} \sigma_i^2 + \frac{\gamma^2}{2} \mathbb Var(\Delta \ln(c_{t+1}) ) - \gamma \mathbb Cov(\sigma_i \epsilon_{i, t+1}, \, \Delta \ln(c_{t+1})) }
\end{align*}
and finally:
\begin{align*}
	r_{i, t+1} &= \rho + \gamma \mathbb{E}_t \Delta \ln(c_{t+1}) - \frac{\gamma^2}{2} \mathbb Var(\Delta \ln(c_{t+1}) ) + \gamma \mathbb Cov(\sigma_i \epsilon_{i, t+1}, \, \Delta \ln(c_{t+1}))
\end{align*}

\item Riskfree bond is therefore priced according to:
\begin{equation*}
	r_{t+1}^f = \rho + \gamma \mathbb{E}_t \Delta \ln(c_{t+1}) - \frac{\gamma^2}{2} \mathbb Var(\Delta \ln(c_{t+1}) )
\end{equation*}
\end{itemize}
\end{frame}


%%%%%%%%%%%%%%%%%%%%%%%%%%  SLIDE   %%%%%%%%%%%%%%%%%%%%%%%%%%%%%%%%
\begin{frame}{}
\begin{itemize}
\item Let asset $i$ denote the stock market

\item Equity premium is then given by
\begin{equation*}
	r_{i, t+1} - r_{t+1}^f = \gamma \mathbb Cov(\sigma_i \epsilon_{i, t+1}, \, \Delta \ln(c_{t+1}))
\end{equation*}

\item The equity premium is equal to the coefficient of relative risk aversion ($\gamma$) times the covariance of stock returns with consumption growth. 

\item Key insight: risk is about \textit{covariance}, and in particular covariance with consumption growth demands compensation. ($\gamma$ is like the price of risk and the covariance is like the quantity of risk.)

\item Mehra-Prescott (1986): Rearrange and use post-war data 
\begin{equation*}
	\gamma = \frac{r_{equity} - r^f}{\sigma_{equity, \Delta c}} \approx \frac{0.06}{0.0003} = 200.
\end{equation*}

\end{itemize}
\end{frame}


%%%%%%%%%%%%%%%%%%%%%%%%%%  SLIDE   %%%%%%%%%%%%%%%%%%%%%%%%%%%%%%%%
\begin{frame}{}
\begin{itemize}
\item Mankiw-Zeldes (1991): What value of $X$ makes you indifferent between the following two gambles: 
\begin{equation*}
	\text{gamble 1} = \begin{cases}
		\$ 50,000 & \text{ with probability } 0.5 \\
		\$ 100,000 & \text{ with probability } 0.5
	\end{cases}
\end{equation*}
\begin{equation*}
	\text{gamble 2} = \$ X \quad \text{with probability} 1
\end{equation*}

\item Implied risk aversion coefficient: 
\begin{align*}
	\gamma &= 1 \quad \implies \quad X = \$70,711 \\ 
	\gamma &= 3 \quad \implies \quad X = \$63,246 \\ 
	\gamma &= 20 \quad \implies \quad X = \$51,858 \\ 
	\gamma &= 200 \quad \implies \quad X = \$50,174 \\ 
\end{align*}

\end{itemize}
\end{frame}



%%%%%%%%%%%%%%%%%%%%%%%%%%  SLIDE   %%%%%%%%%%%%%%%%%%%%%%%%%%%%%%%%
\begin{frame}{2. Portfolio choice in continuous time}
\begin{itemize}
\item Economy has two assets, populated by a representative household

\item \textbf{Lucas tree}:
\begin{itemize}
\vspace{1mm}
\item Tree is in fixed unit supply

\vspace{-2mm}
\item Pays dividend $D_t$ that evolves as $\frac{dD}{D} = \mu_D dt + \sigma_D dB$

\vspace{-2mm}
\item Trades at price $Q_t$, which follows $\frac{dQ}{Q} = \mu_Q dt + \sigma dB$ (conjecture)
\end{itemize}

\item We guess and verify that holding return evolves according to 
\begin{equation*}
	dR = \frac{D dt + dQ}{Q} = \frac{D}{Q} dt + \mu_Q dt + \sigma dB,
\end{equation*}
where $\rho_D = D/Q$ is the \textit{constant} dividend-price ratio and $\mu = D/Q + \mu_Q$

\item \textbf{Riskfree bond}: pays no dividend but price evolves as $\frac{dP}{P} = r dt$
\end{itemize}
\end{frame}



%%%%%%%%%%%%%%%%%%%%%%%%%%  SLIDE   %%%%%%%%%%%%%%%%%%%%%%%%%%%%%%%%
\begin{frame}{}
\begin{itemize}
\item Household's lifetime value given by
\begin{equation*}
	\max_{ \{c_t\}_{t \geq 0} } \mathbb E_0 \int_0^\infty e^{- \rho t} u(c_t) dt
\end{equation*}

\item Denote by $b_t$ and $k_t$ household's bond and stock holdings

\item Budget constraint:
\begin{equation*}
	Q dk + P db + c dt = D k dt
\end{equation*}

\item Define \textit{liquid net worth} $n$ and the \textit{risky portfolio share} $\theta$ implicity via 
\begin{align*}
	\theta n &= Q k \\
	(1-\theta) n &= Pb
\end{align*}

\item Can rearrange budget constraint now to obtain 
\begin{equation*}
	dn = \bigg[ r n + \theta n \Big( \mu - r \Big) - c \bigg] dt + \theta n \sigma dB
\end{equation*}

\end{itemize}
\end{frame}


%%%%%%%%%%%%%%%%%%%%%%%%%%  SLIDE   %%%%%%%%%%%%%%%%%%%%%%%%%%%%%%%%
\begin{frame}{}
Recursive representation yields HJB equation:
\begin{equation*}
	\rho V(n) = \max_{c, \theta} \Big\{ u(c) + \Big[ r n + \theta n \Big( \mu - r \Big) - c \Big] V'(n) + \frac{1}{2} (\sigma \theta n)^2 V''(n) \Big\}
\end{equation*}
with first-order conditions
\begin{align*}
	u'(c) &= V'(n) \\
	\theta &= - \frac{V'(n)}{n V''(n)} \cdot \frac{\mu - r}{\sigma^2}
\end{align*}

\vspace{5mm}
\textbf{Lemma.} The Euler equation for marginal utility is given by
\begin{equation*}
	\frac{du_c}{u_c} = (\rho - r) dt - \frac{\mu - r}{\sigma} dB.
\end{equation*}


\vspace{5mm}
\textbf{Lemma.} With CRRA utility, the Euler equation for consumption is
\begin{equation*}
	\frac{dc}{c} = \frac{r - \rho}{\gamma} dt + \frac{1+\gamma}{2} \bigg( \frac{\mu - r}{\gamma \sigma} \bigg)^2 dt + \frac{\mu - r}{\gamma \sigma} dB.
\end{equation*}
\end{frame}


%%%%%%%%%%%%%%%%%%%%%%%%%%  SLIDE   %%%%%%%%%%%%%%%%%%%%%%%%%%%%%%%%
\begin{frame}{}
\begin{itemize}
\item The representative household must consume the dividend (fruit) of the tree, so 
\begin{equation*}
	c_t = D_t
\end{equation*}

\item Recall:
\begin{align*}
	\frac{dD}{D} &= \mu_D dt + \sigma_D dB \\
	\frac{dc}{c} &= \frac{r - \rho}{\gamma} dt + \frac{1+\gamma}{2} \bigg( \frac{\mu - r}{\gamma \sigma} \bigg)^2 dt + \frac{\mu - r}{\gamma \sigma} dB
\end{align*}

\item So matching coefficients: 
\begin{align*}
	\mu_D &= \frac{r - \rho}{\gamma} + \frac{1+\gamma}{2} \bigg( \frac{\mu - r}{\gamma \sigma} \bigg)^2 \\
	\sigma_D &= \frac{\mu - r}{\gamma \sigma}
\end{align*}

\item Riskfree rate: $r = \rho + \gamma \mu_D - \frac{\gamma(1+\gamma)}{2} \sigma_D^2$.
\end{itemize}
\end{frame}


%%%%%%%%%%%%%%%%%%%%%%%%%%  SLIDE   %%%%%%%%%%%%%%%%%%%%%%%%%%%%%%%%
\begin{frame}{}
\begin{itemize}
\item We can do even better!

\item Guess and verify that 
\begin{equation*}
	V(n) = \frac{1}{1-\gamma} \kappa^{-\gamma} n^{1-\gamma}
\end{equation*}
where $\kappa = \frac{1}{\gamma} \Big[ \rho - (1-\gamma)r - \frac{1-\gamma}{2\gamma} \Big( \frac{\mu-r}{\sigma} \Big)^2 \Big]$

\item Solution of the model is then given by:
\begin{align*}
	c &= \kappa n \\
	\theta &= \frac{\mu - r}{\gamma \sigma^2} \\
	r &= \rho + \gamma \mu_D - \frac{\gamma(1+\gamma)}{2} \sigma_D^2 \\
	\mu - r &= \gamma \sigma \sigma_D = \gamma \mathbb Cov \Big( \frac{dc}{c} , dR \Big)
\end{align*}


\end{itemize}
\end{frame}


%%%%%%%%%%%%%%%%%%%%%%%%%%  SLIDE   %%%%%%%%%%%%%%%%%%%%%%%%%%%%%%%%
\begin{frame}{}
\begin{itemize}
\item What does this model tell us? Interesting special case: $\gamma = 1$

\item Consumption collapses to $c = \rho n$ $\implies$ consume constant fraction $\rho$ of lifetime net worth

\item How much should we invest in the stock market? 
\begin{equation*}
	\theta = \frac{\mu - r}{\gamma \sigma^2} \approx \frac{0.06}{1 \cdot (0.16)^2} = 2.34
\end{equation*}

\item In the absence of frictions and with $\gamma = 1$, this model tells us to invest $2.34$ times our total net worth (including human capital) in the stock market 

\end{itemize}
\end{frame}











%%%%%%%%%%%%%%%%%%%%%%%%%%%%%%%%%%%%%%%%%%%%%%%%%%%%%%%%%%%%%%
%%%%%%%%%%%%%%%%%%%%%%%%%%%%%%%%%%%%%%%%%%%%%%%%%%%%%%%%%%%%%%
%%%%%%%%%%%%%%%%%%%%%%%%%%%%%%%%%%%%%%%%%%%%%%%%%%%%%%%%%%%%%%

\appendix


\end{document}









