\documentclass[11pt]{extarticle}
\usepackage{fullpage,amsmath,amsfonts,microtype,nicefrac,amssymb, amsthm}
\usepackage[left=1in, bottom=1in, top=1in, right = 1in]{geometry}
\usepackage{textcomp}
\usepackage{mathpazo}
\usepackage{mathrsfs}
\usepackage[T1]{fontenc}
\usepackage[utf8]{inputenc}
\usepackage[english]{babel}
\usepackage{graphicx}

\usepackage{microtype}

\usepackage{bm}
\usepackage{dsfont}
\usepackage{enumerate}
\usepackage{ragged2e}

\setlength{\parindent}{24pt}
\setlength{\jot}{8pt}


\usepackage[shortlabels]{enumitem}


%% FOOTNOTES
\usepackage[bottom]{footmisc}
\usepackage{footnotebackref}


%% FIGURE ENVIRONMENT
%\graphicspath{{}}
\usepackage[margin=15pt, font=small, labelfont={bf}, labelsep=period]{caption}
\usepackage{subcaption}
\captionsetup[figure]{name={Figure}, position=above}
\usepackage{float}
\usepackage{epstopdf}


%% NEW COMMANDS
\renewcommand{\baselinestretch}{1.25} 
\renewcommand{\qedsymbol}{$\blacksquare$}
\newcommand{\R}{\mathbb{R}}
\newcommand{\indep}{\mathrel{\text{\scalebox{1.07}{$\perp\mkern-10mu\perp$}}}}
\renewcommand{\b}{\begin}
\newcommand{\e}{\end}

%% NEWTHEOREM
\theoremstyle{plain}
\newtheorem{thm}{Theorem}
\newtheorem{lem}[thm]{Lemma}
\newtheorem{prop}[thm]{Proposition}

\theoremstyle{definition}
\newtheorem{defn}[thm]{Definition}
\newtheorem{ex}[thm]{Example}
\newtheorem{remark}[thm]{Remark}
\newtheorem{cor}[thm]{Corollary}

%% LINKS and COLORS
\usepackage[dvipsnames]{xcolor}
\usepackage{hyperref}
\definecolor{myred}{RGB}{163, 32, 45}
\hypersetup{
	%backref=true,
	%pagebackref=true,
	colorlinks=true,
	urlcolor=myred,
	citecolor=myred, 
	linktoc=all,     
	linkcolor=myred,
}

%% TABLE OF CONTENTS
\addto\captionsenglish{
	\renewcommand{\contentsname}
	{}% This removes the heading over the table of contents.
}



%%%%%%%%%%%%%%%%%%%%%%%%%%%%%%%%%%%%%%%%%%%%%%%%%%%%%%%%%%%%
%%%%%%%%%%%%%%%%%%%%%%%%%%%%%%%%%%%%%%%%%%%%%%%%%%%%%%%%%%%%
%%%%%%%%%%%%%%%%%%%%%%            END PREAMBLE           %%%%%%%%%%%%%%%%%%%%%%%%
%%%%%%%%%%%%%%%%%%%%%%%%%%%%%%%%%%%%%%%%%%%%%%%%%%%%%%%%%%%%
%%%%%%%%%%%%%%%%%%%%%%%%%%%%%%%%%%%%%%%%%%%%%%%%%%%%%%%%%%%%

\title{Dynamic Optimization: Problem Set \#4: Solutions}

\author{Andreas Schaab}

\date{Fall, 2022}



\begin{document}

\maketitle
\thispagestyle{empty}
\setcounter{page}{0}


%%%%%%%%%%%%%%%%%%%%%%%%%%%%%%%%%%%%%%%%%%%%%%%%%%%%%%%%%%%%
%%%%%%%%%%%%%%%%%%%%%%%%%%%%%%%%%%%%%%%%%%%%%%%%%%%%%%%%%%%%
\vspace{10mm}
\section*{Problem 1 (hard) ***}

In this problem, we work through a simple variant of the seminal Brunnermeier and Sannikov (2014, AER) paper.

We study the consumption, savings, and investment problem of an agent. We denote by $\{ D_t \}_{t \geq 0}$ the \textit{dividend stream} and by $\{ Q_t \}_{t \geq 0}$ the asset price. Assume that the asset price evolves according to 
\begin{equation*}
	\frac{ dQ }{Q} = \mu_Q dt + \sigma_Q dB,
\end{equation*}
where you can interpret $\mu_Q$ and $\sigma_Q$ as simple constants (alternatively, think of them as more complicated objects that would be determined in general equilibrium, which we abstract from here).


This is a model of two assets, capital and bonds. Bonds pay the riskfree rate of return $r_t$. Capital is accumulated and owned by the agent. Capital is traded at price $Q_t$ and yields dividends at rate $D_t$.


The key interesting feature of this problem is that the agent faces both (idiosyncratic) earnings risk and (aggregate) asset price risk.


\vspace{10mm}
%\textbf{Gerard: clean up the following}
Households take as given all aggregate prices and behave according to preferences given by
\begin{equation*}
	\mathbb{E}_0 \int_0^\infty e^{- \rho t} u(c_t) dt.
\end{equation*}

Households consume and save, investing their wealth into bonds and capital. Letting $k$ denote a household's units of capital owned and $b$ units of bonds, the budget constraint is characterized by
\begin{align*}
	dk_t &= \Phi(\iota_t) k_t - \delta k_t \\
	db_t &= r_t b_t + D_t k_t + w_t z_t - c_t -  \iota_t k_t.
\end{align*}
The rate of investment is given by $\iota_t$. Investment adjustment costs are captured by the concave technology $\Phi$. Dividends are paid to households in units of the numeraire, thus entering the equation for $db$. The law of motion for household earnings are given by
\begin{equation*}
	dz = \mu_z dt + \sigma_z dW. 
\end{equation*}

\begin{enumerate}[(a)]
 \item Interpret all terms of the two budget constraints


 It is convenient to rewrite the household problem in terms of liquid net worth, defined by the equations 
\begin{align*}
	\theta n &= Q k \\
	(1-\theta) n &= b,
\end{align*}
so that total liquid net worth is $n = Qk + b$.

\item How would you refer to $\theta$? Assume households' capital accumulation is non-stochastic. That is, there is no capital quality risk. Show that the liquid net worth evolves according to 

\begin{equation*}
	dn = rn +  \theta n \bigg[ \frac{D - \iota }{Q} +  \frac{dQ}{Q} + \Phi(\iota) - \delta - r \bigg]  + w z - c .
\end{equation*}
 
\textbf{Solution:} 

 First derivating, $dn = k dQ + Q dk + (dk)(dQ) + db$. Because there is no risk therefore $(dk)(dQ) = 0$. Plugging in and simplifying, 
\begin{equation*}
	dn = rn +  \theta n \bigg[ \frac{D - \iota }{Q} +  \frac{dQ}{Q} + \Phi(\iota) - \delta - r \bigg]  + w z - c .
\end{equation*}


\item Argue why the choice of $\iota$ is entirely static in this setting and show it is only a function of capital $\iota = \iota(Q)$ 

\textbf{Solution:}
\begin{equation*}
	\Phi'(\iota) = \frac{1}{Q},
\end{equation*}


  Recall also that households take as given aggregate ``prices'' $(r, w, D, Q)$.  This will allow us to work with a simplified representation. Define 
\begin{equation*}
	dR = \underbrace{\frac{D - \iota (Q) }{Q}}_\text{Dividend yield} dt + \underbrace{\Big[\Phi(\iota(Q)) - \delta \Big] dt +  \frac{dQ}{Q}}_\text{Capital gains} \equiv \mu_R dt + \sigma_R dB
\end{equation*}
to be the effective rate of return on households' capital investments. And where
\begin{align*}
	\mu_R &= \frac{D - \iota (Q) }{Q}  + \Phi(\iota(Q)) - \delta + \mu_Q \\
	\sigma_R &= \sigma_Q. 
\end{align*}
After solving for $\iota = \iota(Q)$, this return is exogenous from the perspective of the household: it depends on macro conditions and prices, but not on the particular portfolio composition of the household.

\item Show that the law of motion of the household's liquid net worth satisfies the following equation. Why do you think using liquid net worth is useful? And why do we want this law of motion?

\begin{equation*}
	dn = rn + \theta n (\mu_R - r) + wz - c + \theta n \sigma_R dB. 
\end{equation*}


\textbf{Solution:}
 $dn = rn + \theta n (dR - r) + wz - c$ and then substitute using the definitons of $\mu_R$ and $\sigma_R$

\vspace{5mm}
\noindent
\textbf{State space.} We denote the agent's individual states by $(n, z)$. We assume there is a variable $\Gamma_t$ that evolves according to the diffusion process 
\begin{equation*}
	d \Gamma = \mu_\Gamma dt + \sigma_\Gamma dB.
\end{equation*}
We call $\Gamma$ the \textit{aggregate state of the economy}. We can write all capital prices as functions of this aggregate state, that is, 
\begin{equation*}
	r_t = r(\Gamma_t), \hspace{5mm}	D_t = D(\Gamma_t), \hspace{5mm}	Q_t = Q(\Gamma_t).
\end{equation*}
This now allows us to write the household problem recursively with $\Gamma$ as an extra state variable. Note that otherwise, we would need to keep track of three different aggregate state variables. 


\vspace{5mm}
\noindent
\textbf{Recursive representation.} 
The household problem can therefore be written in terms of the household state variables $(n,z)$ as well as the aggregate state space $\Gamma$. 
\item Show that the household problem satisfies the following HJB
\begin{align*}
	\rho V(n,z,\Gamma) = \max_{c,\theta} \bigg\{  & u(c) + V_n \Big[ rn + \theta n (\mu_R - r) + wz - c \Big] + \frac{1}{2} V_{nn} (\theta n \sigma_R)^2 +  V_z \mu_z + \frac{1}{2} V_{zz} \sigma_z^2  \\
	& + V_{n \Gamma} \theta n \sigma_R \sigma_\Gamma + V_\Gamma \mu_\Gamma + \frac{1}{2} \sigma_\Gamma^T V_{\Gamma \Gamma} \sigma_\Gamma \bigg\},
\end{align*}
where you can assume that $\mathbb{E}(dW dB) = 0$. This means that households' earnings risk is uncorrelated with the aggregate state $\Gamma$.

\textbf{Solution:} We have 3 state variables $(n,z,\Gamma)$ and 2 controls $(c,\theta)$. The HJB with the generator is $$\rho V(n,z,\Gamma) = \max_{c,\theta} \bigg\{u(c)+\mathcal A V \bigg\}$$
Rember also that we get the generator by taking expectations after applying Ito's Lemma. We do the two steps, apply Ito's lemma to $dV$ (ignoring the dt):

\begin{align*}
	& dV = V_n  \Big[ rn + \theta n (\mu_R - r) + wz - c \Big] + \frac{1}{2} V_{nn} (\theta n \sigma_R)^2  + V_n \theta n \sigma_R dB \\
	& + V_z \mu_z + \frac{1}{2} V_{zz} \sigma_z^2 + V_{zz} \sigma_z dW +  V_\Gamma \mu_\Gamma + \frac{1}{2} \sigma_\Gamma^2 V_{\Gamma \Gamma}  +V_{\Gamma \Gamma} \sigma_\Gamma dB + V_{n\Gamma}  (\theta n \sigma_R) \sigma_\Gamma + V_{nz}  (\theta n \sigma_R) \sigma_z dWdB \\
	& + V_{z\Gamma}  \sigma_\Gamma \sigma_z dWdB
\end{align*}
Because there are 3 state variables, we also have to take into account the cross-partial derivatives (and correlations). For the term  $V_{n\Gamma}$ because $n$ and $\Gamma$ depend on the same stochastic process we have $dB dB = dt$. Taking expectations the terms with $dB$, $dB$ and $dBdW$ disappear. Substituting back into the HJB we get the result.

\item Derive the first-order conditions for consumption and portfolio choice


\textbf{Solution}
\begin{align*}
	u_c &= V_n \\
	\theta  &= - \bigg( \frac{V_n}{n V_{nn} } \frac{\mu_R - r}{\sigma_R^2 } + \frac{V_{n \Gamma} }{n V_{nn} } \frac{ \bm \sigma_\Gamma }{\sigma_R} \bigg).
\end{align*}


\item (hard) Prove the following lemma (use envelope condition and apply Ito's lemma to $V_n(h,y,\Gamma)$)

\begin{lem}\label{lem:MUEuler}
	
	The household Euler equation for marginal utility is given by
	\begin{equation*}
		\frac{d u_c}{u_c} = (\rho - r) dt -  \frac{\mu_R - r}{\sigma_R} dB  - \gamma \frac{c_z}{c} \sigma_z dW .
	\end{equation*}
	
\end{lem}

\textbf{Solution:}

\vspace{2mm}
\noindent
\begin{proof}
	
	The HJB envelope condition is given by
	\begin{align*}
		(\rho - r) V_n = & V_{nn} s + V_n  \theta (\mu_R - r) + \frac{1}{2} V_{nnn} (\theta n \sigma_R )^2 +  V_{nn} n (\theta \sigma_R )^2  + V_{ny} \mu_y + \frac{\sigma_y^2}{2} V_{yny} \\
		& + V_{nn \Gamma} \theta n \sigma_R  \bm \sigma_\Gamma + V_{n \Gamma} \theta \sigma_R  \bm \sigma_\Gamma  + V_{n\Gamma} \bm \mu_\Gamma  + \frac{1}{2} \bm \sigma_\Gamma^T V_{ n \Gamma \Gamma} \bm \sigma_\Gamma.
	\end{align*}
	Applying Ito's lemma to $V_n(n, y, \Gamma)$, we have 
	\begin{align*}
		dV_n &= V_{nn} dn + \frac{1}{2} V_{nnn} (dn)^2 + V_{ny} dy + \frac{1}{2} V_{nyy} (dy)^2 + V_{n \Gamma} d\Gamma + \frac{1}{2} V_{n\Gamma \Gamma} (d\Gamma)^2  + V_{nn \Gamma} (dn) ( d \Gamma) \\
		&= V_{nn} (s + \theta n \sigma_R dB) + \frac{1}{2} V_{nnn} (\theta n \sigma_R)^2 + V_{ny} (\mu_y + \sigma_y dW) + \frac{1}{2} V_{nyy} \sigma_y^2 \\
		& \;\;\;\;\; + V_{n \Gamma} (\bm \mu_\Gamma + \bm \sigma_\Gamma dB ) + \frac{1}{2} \bm{\sigma}_\Gamma^T V_{n\Gamma \Gamma} \bm \sigma_\Gamma  + V_{nn \Gamma} (\theta n \sigma_R) ( \bm{\sigma}_\Gamma).
	\end{align*}
	Putting this together with 
	\begin{align*}
		(\rho - r) V_n - V_n  \theta (\mu_R - r) -  V_{nn} n (\theta \sigma_R )^2 - V_{n \Gamma} \theta \sigma_R  \bm \sigma_\Gamma  = & V_{nn} s + \frac{1}{2} V_{nnn} (\theta n \sigma_R )^2  + V_{ny} \mu_y + \frac{\sigma_y^2}{2} V_{yny} \\
		& + V_{nn \Gamma} \theta n \sigma_R  \bm \sigma_\Gamma + V_{n\Gamma} \bm \mu_\Gamma  + \frac{1}{2} \bm \sigma_\Gamma^T V_{ n \Gamma \Gamma} \bm \sigma_\Gamma,
	\end{align*}
	we have 
	\begin{align*}
		dV_n = &V_{nn} \theta n \sigma_R dB + V_{ny} \sigma_y dW + V_{n \Gamma}  \bm \sigma_\Gamma dB  \\
		&  + (\rho - r) V_n - V_n  \theta (\mu_R - r) -  V_{nn} n (\theta \sigma_R )^2 - V_{n \Gamma} \theta \sigma_R  \bm \sigma_\Gamma. 
	\end{align*}
	Using 
	\begin{equation*}
		\theta n V_{nn} = - \bigg( V_n \frac{\mu_R - r}{\sigma_R^2 } + V_{n \Gamma} \frac{ \bm \sigma_\Gamma }{\sigma_R} \bigg),
	\end{equation*}
	we have 
	\begin{align*}
		dV_n = & - \bigg( V_n \frac{\mu_R - r}{\sigma_R^2 } + V_{n \Gamma} \frac{ \bm \sigma_\Gamma }{\sigma_R} \bigg) \sigma_R dB + V_{ny} \sigma_y dW + V_{n \Gamma}  \bm \sigma_\Gamma dB  \\
		&  + (\rho - r) V_n - V_n  \theta (\mu_R - r) + \bigg( V_n \frac{\mu_R - r}{\sigma_R^2 } + V_{n \Gamma} \frac{ \bm \sigma_\Gamma }{\sigma_R} \bigg) \theta  \sigma_R^2 - V_{n \Gamma} \theta \sigma_R  \bm \sigma_\Gamma. 
	\end{align*}
	Simplifying, 
	\begin{align*}
		dV_n = & -  V_n \frac{\mu_R - r}{\sigma_R} dB  - V_{n \Gamma}  \bm \sigma_\Gamma  dB + V_{ny} \sigma_y dW + V_{n \Gamma}  \bm \sigma_\Gamma dB  \\
		&  + (\rho - r) V_n - V_n  \theta (\mu_R - r) +  V_n (\mu_R - r)\theta + V_{n \Gamma}  \bm \sigma_\Gamma \theta  \sigma_R - V_{n \Gamma} \theta \sigma_R  \bm \sigma_\Gamma. 
	\end{align*}
	Combining terms yield the result. 
\end{proof}


\item  (hard)  Prove the following lemma 

\begin{lem}
	
	The household Euler equation for consumption is given by
	\begin{align*}
		\frac{dc}{c} = & \frac{r - \rho}{\gamma} dt + \frac{1}{2} (1+\gamma) \bigg[ \bigg( \frac{\mu_R - r}{\gamma \sigma_R} \bigg)^2 + \bigg( \frac{c_z}{c} \sigma_z \bigg)^2 \bigg] dt +  \frac{\mu_R - r}{\gamma \sigma_R} dB  + \frac{c_z}{c} \sigma_z dW .
	\end{align*}	
\end{lem}


\textbf{Solution:}

\vspace{2mm}
\noindent
\begin{proof}
	
	The consumption policy function is given by $c = c(n, y, \Gamma)$. Thus, 
	\begin{equation*}
		dc = c_n dn + c_y dy + c_\Gamma d \Gamma  + \frac{1}{2} c_{nn} (dn)^2 + \frac{1}{2} c_{yy} \sigma_y^2 + \frac{1}{2} \bm \sigma_\Gamma^T c_{\Gamma \Gamma} \bm \sigma_\Gamma + c_{n \Gamma} \theta n \sigma_R \bm \sigma_\Gamma,
	\end{equation*}
	and so 
	\begin{align*}
		(dc)^2 &= \bigg( c_n \theta n \sigma_R dB + c_y \sigma_y dW + c_\Gamma \bm \sigma_\Gamma dB \bigg)^2 \\
		&= \bigg( c_n \theta n \sigma_R dB + c_\Gamma \bm \sigma_\Gamma dB \bigg)^2 \bigg( c_y \sigma_y dW \bigg)^2 \\
		&= \bigg[ (c_n \theta n \sigma_R)^2 + 2 c_n c_\Gamma \theta n \sigma_R \bm \sigma_\Gamma + (c_\Gamma \bm \sigma_\Gamma)^2 + (c_y \sigma_y)^2 \bigg] dt.
	\end{align*}
	
	
	
	
	
	Let's simplify this expression a bit. Notice that we have
	\begin{align*}
		(c_n n \sigma_R)^2 \theta \theta + 2c_n c_\Gamma \theta n \sigma_R \bm \sigma_\Gamma &= - (c_n n \sigma_R)^2 \theta \bigg( \frac{u_c}{n u_{cc} c_n } \frac{\mu_R - r}{\sigma_R^2 } + \frac{ u_{cc} c_\Gamma }{n u_{cc} c_n } \frac{ \bm \sigma_\Gamma }{\sigma_R} \bigg) + 2 c_n c_\Gamma \theta n \sigma_R \bm \sigma_\Gamma \\
		&= - (c_n n \sigma_R)^2 \theta \frac{1}{n c_n } \bigg( - c \frac{\mu_R - r}{\gamma \sigma_R^2 } +   c_\Gamma \frac{ \bm \sigma_\Gamma }{\sigma_R} \bigg) + 2c_n c_\Gamma \theta n \sigma_R \bm \sigma_\Gamma \\
		&= (c_n n \sigma_R)^2 \theta \frac{1}{n c_n } c \frac{\mu_R - r}{\gamma \sigma_R^2 } - (c_n n \sigma_R) \theta  c_\Gamma \bm \sigma_\Gamma  + 2c_n c_\Gamma \theta n \sigma_R \bm \sigma_\Gamma \\
		&= \bigg( \frac{\mu_R - r}{\gamma \sigma_R^2 } - \frac{c_\Gamma}{c} \frac{ \bm \sigma_\Gamma }{\sigma_R} \bigg) c^2 \frac{\mu_R - r}{\gamma } + c_n c_\Gamma \theta n \sigma_R \bm \sigma_\Gamma \\
		&=  c^2 \bigg( \frac{\mu_R - r}{\gamma \sigma_R} \bigg)^2  - c_\Gamma  \bm \sigma_\Gamma  c \frac{\mu_R - r}{\gamma \sigma_R} + c_n c_\Gamma \theta n \sigma_R \bm \sigma_\Gamma.
	\end{align*}
	Therefore, we have
	\begin{equation*}
		(dc)^2 = \bigg[ c^2 \bigg( \frac{\mu_R - r}{\gamma \sigma_R} \bigg)^2  - c_\Gamma  \bm \sigma_\Gamma  c \frac{\mu_R - r}{\gamma \sigma_R} + c_n c_\Gamma \theta n \sigma_R \bm \sigma_\Gamma + (c_\Gamma \bm \sigma_\Gamma)^2 + (c_y \sigma_y)^2 \bigg] dt
	\end{equation*}
	We can take one last step here, noting that 
	\begin{align*}
		(dc)^2 &= \bigg[ c^2 \bigg( \frac{\mu_R - r}{\gamma \sigma_R} \bigg)^2  - c_\Gamma  \bm \sigma_\Gamma  c \frac{\mu_R - r}{\gamma \sigma_R} +  c_\Gamma  \bm \sigma_\Gamma c \frac{\mu_R - r}{\gamma \sigma_R } - (c_\Gamma  \bm \sigma_\Gamma)^2 + (c_\Gamma \bm \sigma_\Gamma)^2 + (c_y \sigma_y)^2 \bigg] dt \\
		&= \bigg[ c^2 \bigg( \frac{\mu_R - r}{\gamma \sigma_R} \bigg)^2  + (c_y \sigma_y)^2 \bigg] dt.
	\end{align*}
	I haven't used this yet in the rest of the proof, but it's the same thing... 
	
	
	Using Ito's lemma for $u_c(c)$, we have 
	\begin{equation*}
		du_c = u_{cc} dc + \frac{1}{2} u_{ccc} (dc)^2. 
	\end{equation*}
	Plugging in and using CRRA, we have 
	\begin{gather*}
		(\rho - r) dt -  \frac{\mu_R - r}{\sigma_R} dB  - \gamma \frac{c_y}{c} \sigma_y dW \\
		= \frac{u_{cc}}{u_c} dc + \frac{1}{2} \frac{u_{ccc}}{u_c} \bigg[ c^2 \bigg( \frac{\mu_R - r}{\gamma \sigma_R} \bigg)^2  - c_\Gamma  \bm \sigma_\Gamma  c \frac{\mu_R - r}{\gamma \sigma_R} + c_n c_\Gamma \theta n \sigma_R \bm \sigma_\Gamma + (c_\Gamma \bm \sigma_\Gamma)^2 + (c_y \sigma_y)^2 \bigg] dt.
	\end{gather*}
	Plugging in for CRRA, 
	\begin{gather*}
		\frac{r - \rho}{\gamma} dt +  \frac{\mu_R - r}{\gamma \sigma_R} dB  + \frac{c_y}{c} \sigma_y dW \\
		= \frac{dc}{c} - \frac{1}{2} \frac{1+\gamma}{c^2} \bigg[ c^2 \bigg( \frac{\mu_R - r}{\gamma \sigma_R} \bigg)^2  - c_\Gamma  \bm \sigma_\Gamma  c \frac{\mu_R - r}{\gamma \sigma_R} + c_n c_\Gamma \theta n \sigma_R \bm \sigma_\Gamma + (c_\Gamma \bm \sigma_\Gamma)^2 + (c_y \sigma_y)^2 \bigg] dt.
	\end{gather*}	
	Therefore, we have 
	\begin{align*}
		\frac{dc}{c} = & \frac{r - \rho}{\gamma} dt +  \frac{\mu_R - r}{\gamma \sigma_R} dB  + \frac{c_y}{c} \sigma_y dW \\
		&+ \frac{1}{2} (1+\gamma) \bigg[ \bigg( \frac{\mu_R - r}{\gamma \sigma_R} \bigg)^2  - \frac{c_\Gamma }{c} \bm \sigma_\Gamma \frac{\mu_R - r}{\gamma \sigma_R} + \frac{1}{c^2} c_n c_\Gamma \theta n \sigma_R \bm \sigma_\Gamma + \bigg( \frac{c_\Gamma}{c} \bm \sigma_\Gamma \bigg)^2 + \bigg( \frac{c_y}{c} \sigma_y \bigg)^2 \bigg] dt.
	\end{align*}
	Lastly, I can substitute in for $\theta$ again, so that this term becomes 
	\begin{align*}
		\frac{1}{c^2} c_n c_\Gamma \theta n \sigma_R \bm \sigma_\Gamma &= \frac{1}{c^2} c_n c_\Gamma n \sigma_R \bm \sigma_\Gamma \frac{c}{n c_n} \bigg( \frac{\mu_R - r}{\gamma \sigma_R^2 } - \frac{c_\Gamma}{c} \frac{ \bm \sigma_\Gamma }{\sigma_R} \bigg) \\
		&= \frac{1}{c} c_\Gamma \sigma_R \bm \sigma_\Gamma \bigg( \frac{\mu_R - r}{\gamma \sigma_R^2 } - \frac{c_\Gamma}{c} \frac{ \bm \sigma_\Gamma }{\sigma_R} \bigg) \\
		&= \frac{c_\Gamma }{c}  \bm \sigma_\Gamma \frac{\mu_R - r}{\gamma \sigma_R } -\bigg( \frac{c_\Gamma}{c}  \bm \sigma_\Gamma  \bigg)^2.
	\end{align*}
	Therefore, 
	\begin{align*}
		\frac{dc}{c} = & \frac{r - \rho}{\gamma} dt +  \frac{\mu_R - r}{\gamma \sigma_R} dB  + \frac{c_y}{c} \sigma_y dW \\
		&+ \frac{1}{2} (1+\gamma) \bigg[ \bigg( \frac{\mu_R - r}{\gamma \sigma_R} \bigg)^2  - \frac{c_\Gamma }{c} \bm \sigma_\Gamma \frac{\mu_R - r}{\gamma \sigma_R} + \frac{c_\Gamma }{c}  \bm \sigma_\Gamma \frac{\mu_R - r}{\gamma \sigma_R } -\bigg( \frac{c_\Gamma}{c}  \bm \sigma_\Gamma  \bigg)^2 + \bigg( \frac{c_\Gamma}{c} \bm \sigma_\Gamma \bigg)^2 + \bigg( \frac{c_y}{c} \sigma_y \bigg)^2 \bigg] dt.
	\end{align*}
	Simplifying yields the result. 
	
\end{proof}




%\vspace{5mm}
%\noindent
%\textbf{Comparison to Brunnermeier and Sannikov.} It turns out that I already derived the analog to Bru-San's Proposition II.2 a long time ago. I have 
%\begin{align*}
%	\frac{dV_n}{V_n} = & (\rho - r)  -   \frac{\mu_R - r}{\sigma_R} dB  + \frac{V_{nz}}{V_n} \sigma_z dW,
%\end{align*}
%where they call $V_n = \theta$. For the sake of comparison, let $V_n = \theta^\text{BS}$. Then, 
%\begin{equation*}
%	\frac{d \theta^\text{BS}}{\theta^\text{BS}} = \mu_\theta dt + \sigma_\theta^B dB + \sigma_\theta^W dW,
%\end{equation*}
%where 
%\begin{align*}
%	\mu_\theta  &= \rho - r \\
%	\underbrace{- \sigma_Q \sigma_\theta^B }_\text{Risk premium} &= \underbrace{ \frac{D - \iota (Q) }{Q}  + \Phi(\iota(Q)) - \delta + \mu_Q  - r }_\text{Expected excess return on capital} \\
%	\sigma_\theta^W  &= \frac{V_{nz}}{V_n} \sigma_z .
%\end{align*}
%Of course, they don't have earnings risk. 


\end{enumerate}


%%%%%%%%%%%%%%%%%%%%%%%%%%%%%%%%%%%%%%%%%%%%%%%%%%%%%%%%%%%%
%%%%%%%%%%%%%%%%%%%%%%%%%%%%%%%%%%%%%%%%%%%%%%%%%%%%%%%%%%%%
\vspace{10mm}
\section*{Problem 2 * (good for intuition)}

\textbf{Credit:} Gabriel Chodorow-Reich

\vspace{5mm}
\noindent
%\textbf{Gerard:} add ``Homework4'' here (whatever you find valuable)

\subsection*{ Problem 2.1}
 IES. Consider the isoelastic utility function:
$$
u(c)=\frac{c^{1-\frac{1}{\sigma}}-1}{1-\frac{1}{\sigma}},
$$
where $\sigma>0$.

\begin{enumerate}[(a)]
\item Prove that  $\lim _{\sigma \rightarrow 1} u(c)=\ln (c)$. (Hint: use l'hopital's rule) 


\textbf{Solution:} As $\sigma \rightarrow 1$, we can see that $\frac{c^{1-1 / \sigma}-1}{1-1 / \sigma} \rightarrow \frac{0}{0}$, which doesn't tell us much. To solve this, recall that L'Hopital's Rule says that $\lim \frac{f(x)}{g(x)}=\lim \frac{f^{\prime}(x)}{g^{\prime}(x)}$. Therefore, as $\sigma \rightarrow 1$, this utility function approaches $\frac{\ln (c) \cdot \frac{1}{\sigma^2}}{\frac{1}{\sigma^2}}=\ln (c)$.


\item The coefficient of relative prudence is $$-\frac{u'''(c)c}{u''(c)}$$ derive it. What is it related to?

\textbf{Solution:}

\end{enumerate}

\subsection*{Problem 2.2}

Euler equation with uncertainty. Consider an agent who lives for two periods, $t=0,1$. The agent can freely borrow or lend at interest rate $r$. The agent has period preferences given by $u\left(c_t\right)=\frac{c_t^{1-\frac{1}{\sigma}}}{1-\frac{1}{\sigma}}$. In period 0, the agent discounts the utility of future consumption at rate $\beta$. The agent receives income $y_0$ in period 0 , but has uncertain income in period 1. Therefore, the agent maximizes expected utility subject to certain income $y_0$ and expected income $E\left[y_1\right]$.

\begin{enumerate}[(a)]

\item Derive the Euler Equation



\textbf{Solution:} 

$$\beta(1+r) E\left[c_1^{-\frac{1}{\sigma}}\right]=c_0^{-\frac{1}{\sigma}}$$

\item Suppose $\sigma \rightarrow \infty$. What conditions are placed on $\beta(1+r)$ if the agent has positive consumption in period 0? Interpret your answer in light of question 1. Recall that we call $\sigma$ the intertemporal elasticity of substitution.

\textbf{Solution:} When $\sigma \rightarrow \infty$, the Euler Equation requires $\beta(1+r)=1$. In this situation, utility has no curvature and so the agent is perfectly willing to substitute consumption across time (though he still discounts). So, if we had $\beta(1+r)>1$, the agent would save all his income every period - a "corner solution," where consumption is not positive. If $\beta(1+r)<1$, he would consume his entire lifetime income in period 0 . In order for an interior solution to exist then, we must have $\beta(1+r)=1$ - so the market benefits from saving are exactly offset by the agent's impatience, and he is indifferent as to when he consumes.

\item Assume $y_1 \in\left\{y_L, y_H\right\}$, with $y_H>y_L$. Argue that this implies $c_1 \in\left\{c_L, c_H\right\}$, with $c_H>c_L$, for some unknown values $c_L$ and $c_H$. Let $b_0$ denote period 0 savings. Then $c_1=(1+r) b_0+y_1$

\textbf{Solution:}  Based on his decision in period 0 , the agent will bring some amount of savings into period $1,(1+r)\left(y_0-c_0\right)$. Since period 1 is the terminal period, he will consume whatever is left at the end of the period: $(1+r)\left(y_0-c_0\right)+y_H$ in the high state, and $(1+r)\left(y_0-c_0\right)+y_L$ in the low state. Those are the values of $c_L$ and $c_H$.

\end{enumerate}


\subsection*{Problem 2.3}
 Income and substitution effects. Recall the consumption function from the 2 period model derived in lecture, with for simplicity $y_1=0$ :
$$
c_0=\frac{1}{1+\beta^\sigma(1+r)^{\sigma-1}} y_0 .
$$

\begin{enumerate}[(a)]

\item Differentiate $c_0$ with respect to $1+r$.


\textbf{Solution:} Have:
$$
c_0=y_0\left[1+\beta^\sigma(1+r)^{\sigma-1}\right]^{-1} .
$$
Chain Rule:
$$
\begin{array}{r}
\frac{\partial c_0}{\partial(1+r)}=-(\sigma-1) \beta^\sigma(1+r)^{\sigma-2}\left[1+\beta^\sigma(1+r)^{\sigma-1}\right]^{-2} y_0 \\
=\frac{(1-\sigma) \beta^\sigma(1+r)^{\sigma-2} y_0}{\left(1+\beta^\sigma(1+r)^{\sigma-1}\right)^2}
\end{array}
$$


\item Explain why your answer to (a) shows that period 0 consumption responds positively to a decrease in the real interest rate if and only if $\sigma>1$. 

\textbf{Solution:} All the terms are positive with the possible exception of $(1-\sigma)$. If $\sigma>1$, we will have a negative derivative, meaning that a decrease in $(1+r)$ leads to an increase in $c_0$, as was asserted in class.


\item  Show that:
$$
c_1=\left[\frac{\beta^\sigma(1+r)^\sigma}{1+\beta^\sigma(1+r)^{\sigma-1}}\right] y_0 .
$$


\textbf{Solution:} Because $y_1=0$, we know $c_0+\frac{c_1}{1+r}=y_0$, meaning $c_1=(1+r)\left(y_0-c_0\right)$. So:
$$
\begin{aligned}
c_1 &=(1+r)\left[1-\frac{1}{1+\beta^\sigma(1+r)^{\sigma-1}}\right] y_0 \\
&=(1+r)\left[\frac{\beta^\sigma(1+r)^{\sigma-1}}{1+\beta^\sigma(1+r)^{\sigma-1}}\right] y_0 \\
&=\left[\frac{\beta^\sigma(1+r)^\sigma}{1+\beta^\sigma(1+r)^{\sigma-1}}\right] y_0
\end{aligned}
$$

\item Show that if $\sigma>0$, then $\frac{\partial c_1}{\partial(1+r)}>0$.

\textbf{Solution:} This is just quotient rule and Chain Rule:
$$\frac{\partial c_1}{\partial(1+r)}=y_0\left[\sigma \beta^\sigma(1+r)^{\sigma-1}\left(1+\beta^\sigma(1+r)^{\sigma-1}\right)-\beta^\sigma(1+r)^\sigma\left((\sigma-1) \beta^\sigma(1+r)^{\sigma-2}\right)\right]$$
$$=y_0 \beta^\sigma(1+r)^{\sigma-1}\left[\sigma\left(1+\beta^\sigma(1+r)^{\sigma-1}\right)-(\sigma-1) \beta^\sigma(1+r)^{\sigma-1}\right]$$
$$=y_0 \beta^\sigma(1+r)^{\sigma-1}\left[\sigma\left(1+\beta^\sigma(1+r)^{\sigma-1}-\beta^\sigma(1+r)^{\sigma-1}\right)+\beta^\sigma(1+r)^{\sigma-1}\right]$$
$$=y_0 \beta^\sigma(1+r)^{\sigma-1}\left[\sigma+\beta^\sigma(1+r)^{\sigma-1}\right]>0$$

\item  Why does the response of $c_0$ to $(1+r)$ depend on the value of $\sigma$, but the response of $c_1$ does not? (Hint: your answer should reference the direction of income and substitution effects for consumption in each period.)

\textbf{Solution:} Because this person earns no income in period 1, we know he is a net saver. Therefore, when $r$ increases, he becomes richer, so the income effect says he should consume more in both period 0 and period 1.
The substitution effect says he should consume more in period 1 and less in period 0 , because $r$ is like an increase in the relative price of consumption today.
So, for $c_0$, the income and substitution effects go in opposite directions, and when $\sigma>1$, the substitution effect dominates. However when we are thinking about $c_1$, the two effects go in the same direction (they both say consume more $c_1$ ), so we will consume more regardless of how strong the substitution effect (i.e. how large $\sigma$ ) is.

\end{enumerate}

%%%%%%%%%%%%%%%%%%%%%%%%%%%%%%%%%%%%%%%%%%%%%%%%%%%%%%%%%%%%
%%%%%%%%%%%%%%%%%%%%%%%%%%%%%%%%%%%%%%%%%%%%%%%%%%%%%%%%%%%%
\vspace{10mm}
\section*{Problem 3 ** (Canonical)}

We now solve a version of the income fluctuations problem in continuous time. In discrete time, the problem is as follows:


\paragraph{Discrete time.} The canonical buffer stock model in discrete-time is a variant of the life-cycle model of consumption featuring idiosyncratic income risk. A household's preferences are given by 
\begin{equation}
	\sum_{t=0}^\infty \beta^t u(c_t),
\end{equation}
and the critical assumption is that $\beta R = 1$, where $\beta$ is the household's discount parameter and $R$ is the gross interest rate. In a model without uncertainty, this assumption would imply a constant consumption profile over time.

The household's budget constraint is encoded in its evolution of wealth, given by 
\begin{equation}
	a_{t+1} = R(y_t + a_t - c_t),
\end{equation}
where $y_t \sim^{iid} F$ is an income shock that is independent and identically distributed over time. We may furthermore assume that agents face an exogenously determined, uniform borrowing constraint, $a_t \geq \underline{a}$, that is tighter than the natural borrowing constraint. 

The recursive problem of the household can then be written using the Bellman equation
\begin{equation}
	v(a_t) = \max_{c_t} u(c_t) + \beta E_t[v(a_{t+1})],
\end{equation}
subject to 
\begin{align*}
	a_{t+1} &= R(a_t + y_t - c_t) \\
	a_{t} \geq 0.
\end{align*}


\vspace{5mm}
\noindent
\textbf{Continuous time without borrowing constraint.} In continuous time, the evolution of wealth is given by 
\begin{equation}
	da_t = (ra_t - c_t) dt + \sigma dB_t,
\end{equation}
where the Brownian term is the appropriate analog to independent and identically distributed income shocks in discrete time. 

We now assume that households are not subject to a borrowing constraint.

\begin{enumerate}[(a)]

\item Write the generator for the stochastic process of wealth. Use it to derive the HJB:
\begin{equation}
	\rho v(a_t) = \max_{c_t} u(c_t) + v'(a_t) [ra_t - c_t] + \frac{1}{2} v''(a_t) \sigma^2.
\end{equation}

\textbf{Solution:} The generator is $$\mathcal A f = (ra_t-c_t)f'+\frac{1]}{2} f''\sigma^2$$. And the HJB is as usual
$$\rho v(a_t) = \max_{c_t} u(c_t) + \mathcal A v$$


The HJB in our simple setting is given by an ordinary instead of a partial differential equation because we only have one state variable. 
\item Show we have the following HJB envelop condition 
\begin{equation*}
	(\rho - r) v'(a_t) = v''(a_t) [ra_t - c(a_t)] + \frac{1}{2} v'''(a_t) \sigma^2.
\end{equation*}


\textbf{Solution:}
The implied first-order condition for consumption is given by
\begin{equation}
	u'(c_t) = v'(a_t),
\end{equation}
where $c_t$ now denotes the optimal consumption policy function. Differentiating the FOC again with respect to wealth, we have $u''(c_t) c'(a_t) = v''(a_t)$. Similarly, differentiating the HJB with respect to wealth yields 
\begin{equation*}
	\rho v'(a_t) = u'(c_t)c'(a_t) + v''(a_t) [ra_t - c(a_t)] + [r - c'(a_t)] v'(a_t) + \frac{1}{2} v'''(a_t) \sigma^2,
\end{equation*}
where we could drop the $\max$ operator since $c(a_t)$ denotes the policy function. Simplifying and using the first-order conditions, we have the following HJB envelope condition
\begin{equation*}
	(\rho - r) v'(a_t) = v''(a_t) [ra_t - c(a_t)] + \frac{1}{2} v'''(a_t) \sigma^2.
\end{equation*}


To obtain a closed-form solution to the buffer-stock model, let utility be log so that $u(c_t) = \ln(c_t)$. We will write the consumption policy function as $c(a_t)$. 
\item Show the HJB satisfies (take FOC, differentiate FOC wrt a and integrate FOC)
\begin{equation*}
	\rho \kappa + \frac{\rho}{c'(a_t)} \ln[c(a_t)] = \ln[c(a_t)] + \frac{r a_t}{c(a_t)} - 1 - \frac{\sigma^2}{2} \frac{c'(a_t)}{c(a_t)^2}.
\end{equation*}

\textbf{Solution:}
The first-order condition for consumption then implies that 
\begin{equation}
	v'(a_t) = \frac{1}{c(a_t)}.
\end{equation}
Integrating both sides with respect to $a_t$ yields
\begin{equation}\label{eq:vfansatz}
	v(a_t) = \frac{1}{c'(a_t)} \ln[c(a_t)] + \kappa.
\end{equation}
The HJB must therefore satisfy 
\begin{equation*}
	\rho \kappa + \frac{\rho}{c'(a_t)} \ln[c(a_t)] = \ln[c(a_t)] + \frac{r a_t}{c(a_t)} - 1 - \frac{\sigma^2}{2} \frac{c'(a_t)}{c(a_t)^2}.
\end{equation*}

 We can see immediately that the $c(a_t)^2$ term in the denominator on the RHS is going to make solving for a policy function $c$ very difficult. It also implies a simple fix, however: Consider an alternative wealth evolution equation given by $da_t = (r a_t - c_t)dt + \sigma a_t dB_t$. 
\item Show the HJB becomes $$\rho v(a_t) = u(c_t) + v'(a_t)[ra_t -c_t] + \frac{\sigma^2}{2} a_t^2 v''(a_t)$$ 

\textbf{Solution:} Apply same steps as in the first question, but in the second term of the generator have $(\sigma a)^2$.

\item Using the Ansatz that the policy function for consumption is linear in wealth, in particular $c(a_t) = \rho a_t$, show \begin{align}
	v(a_t) &= \frac{1}{\rho} \ln(\rho a_t) + \frac{r - \rho}{\rho^2} - \frac{\sigma^2}{2\rho^2}.
\end{align} 

\textbf{Solution:}
\begin{equation*}
	\rho \kappa + \ln(\rho a_t) = \ln(\rho a_t) + \frac{r-\rho}{\rho} - \frac{\sigma^2a_t^2}{2 \rho a_t^2}.
\end{equation*}
The postulated value function (\ref{eq:vfansatz}) therefore satisfies the HJB for $\kappa = \frac{r - \rho}{\rho^2} - \frac{\sigma^2}{2\rho^2}$. The closed-form solution of the buffer stock model with log utility is then given by 
\begin{align}
	c(a_t) &= \rho a_t \\
	v(a_t) &= \frac{1}{\rho} \ln(\rho a_t) + \frac{r - \rho}{\rho^2} - \frac{\sigma^2}{2\rho^2}.
\end{align} 

\end{enumerate}


%%%%%%%%%%%%%%%%%%%%%%%%%%%%%%%%%%%%%%%%%%%%%%%%%%%%%%%%%%%%
%%%%%%%%%%%%%%%%%%%%%%%%%%%%%%%%%%%%%%%%%%%%%%%%%%%%%%%%%%%%
\vspace{10mm}
\section*{Problem 4 ** (Canonical)}

\textbf{Credit:} David Laibson

\vspace{5mm}
\noindent
Solve the ``\textit{eat the pie problem}'' on problem set (PSET) \#2 here: \url{https://projects.iq.harvard.edu/econ2010c/problem-sets-david-laibson}.





%%%%%%%%%%%%%%%%%%%%%%%%%%%%%%%%%%%%%%%%%%%%%%%%%%%%%%%%%%%%
%%%%%%%%%%%%%%%%%%%%%%%%%%%%%%%%%%%%%%%%%%%%%%%%%%%%%%%%%%%%
\vspace{10mm}
\section*{Problem 5 **}


To illustrate this approach, I will consider a simple problem where household preferences are 
\begin{equation*}
	\max_{\{c_t\}} \int_0^\infty e^{- \rho t} u(c_t) dt. 
\end{equation*}
There is no uncertainty. The household budget constraint is given by
\begin{equation*}
	da_t = r_t a_t + w_t - c_t. 
\end{equation*}
Aggregate prices follow a deterministic process, $\bm{r} = \{r_t\}$ and $\bm{w} = \{w_t\}$. Finally, the household starts with an initial wealth position $a_0$, and wealth is the only state variable. 



\textcolor{red}{The next Lemma derives more formally the steps we have already been using in the lectures and psets}

\vspace{5mm}
\noindent
\textbf{Step \#1: Lifetime budget constraint}

\vspace{3mm}
\noindent
\begin{lem} (Lifetime Budget Constraint) For any linear ODE 
	\begin{equation*}
		\frac{dy}{dt} = r(t) y(t) + x(t)
	\end{equation*}
	we have the integration result 
	\begin{equation*}
		y(T)  = y(0) e^{\int_0^T r(s) ds} +  \int_0^T e^{\int_t^T r(s) ds }  x(t) dt.
	\end{equation*}
	
\end{lem}
\vspace{8mm}
\begin{proof}
	Consider any ODE
	\begin{equation*}
		\frac{dy}{dt} = r(t) y(t) + x(t).
	\end{equation*}
	Using an integrating factor approach, we have 
	\begin{equation*}
		e^{\int -r(s) ds} \frac{dy}{dt} - e^{\int -r(s) ds}  r(t) y(t) = e^{\int -r(s) ds} x(t).
	\end{equation*}
	The LHS can then be written as a product rule, so that 
	\begin{equation*}
		\frac{d}{dt} \bigg( y(t) e^{\int -r(s) ds} \bigg) =  \frac{dy}{dt} e^{\int -r(s) ds}  + y(t) e^{\int -r(s) ds} \frac{d}{dt} \bigg( \int -r(s) ds \bigg)  = e^{\int -r(s) ds} x(t).
	\end{equation*}
	The last derivative follows from the fundamental theorem of calculus for indefinite integrals. 
	
	Alternatively, since I know that I will work on the definite time horizon $t \in [0,T]$, I can choose a slightly different integrating factor: I can write $u(t) = e^{- \int_0^t r(s) ds}$, so 
	\begin{equation*}
		e^{\int_0^t -r(s) ds} \frac{dy}{dt} - e^{\int_0^t -r(s) ds}  r(t) y(t) = e^{\int_0^t -r(s) ds} x(t).
	\end{equation*}
	Using Leibniz rule, I have 
	\begin{equation*}
		\frac{d}{dt} \bigg( y(t) e^{\int_0^t -r(s) ds} \bigg) = \frac{dy}{dt}  e^{\int_0^t -r(s) ds} + y(t) e^{\int_0^t -r(s) ds} \frac{d}{dt} \bigg( \int_0^t -r(s) ds \bigg) =  \frac{dy}{dt}  e^{\int_0^t -r(s) ds} - y(t) e^{\int_0^t -r(s) ds} r(t).
	\end{equation*}
	Now, I have 
	\begin{equation*}
		\frac{d}{dt} \bigg( y(t) u(t) \bigg) = u(t) x(t).
	\end{equation*}
	Finally, this implies 
	\begin{equation*}
		y(T) u(T) - y(0) u(0) = \int_0^T u(t) x(t),
	\end{equation*}
	or, noting $u(0) = 1$, 
	\begin{equation*}
		y(T) e^{- \int_0^T r(s) ds} = y(0) +  \int_0^T e^{- \int_0^t r(s) ds} x(t) dt.
	\end{equation*}
	Rearranging, 
	\begin{equation*}
		y(T)  = y(0) e^{\int_0^T r(s) ds} +  \int_0^T e^{\int_t^T r(s) ds }  x(t) dt.
	\end{equation*}
\end{proof}
\begin{enumerate}[(a)]


\item Use this result to characterize the initial lifetime wealth $W$ of the household. We have $W = W(a_0, \bm{r}, \bm{w})$. 

\textbf{Solution:} Using this formula for the household budget constraint, we have
	\begin{equation*}
		\lim_{T \to \infty} a(T)  = a(0) \lim_{T \to \infty} e^{\int_0^T r(s) ds} -  \lim_{T \to \infty} \int_0^T e^{\int_t^T r(s) ds } \Big[ c(t) - w_t \Big] dt.
	\end{equation*}
	Under the no-Ponzi assumption of $\lim_{T \to \infty} a(T) = 0$, this further simplifies to 
	\begin{equation*}
		\int_0^\infty e^{- \int_0^t r(s) ds }   c(t) dt = a(0) + \int_0^\infty e^{- \int_0^t r(s) ds }   w(t) dt \equiv W.
	\end{equation*}

\vspace{5mm}
\noindent
It will be a useful exercise to characterize the response of initial lifetime wealth $dW$ to a general perturbation of this economy, $\{da_0, d \bm r, d \bm w\}$. We have:
\begin{equation*}
	dW = W_{a_0} da_0 + W_{\bm r} d \bm{r} + W_{ \bm w} d \bm w. 
\end{equation*}

\item  Work out each of these derivatives ($ W_{a_0}$, $W_{\bm r}$ and $W_{ \bm w}$) and interpret 



\textbf{Solution:}

Example: Let $R_{0,t} = e^{\int_0^t r_s ds }$ for ease of exposition. The only difficult term is $W_{\bm r}$. We have 
\begin{align*}
	W_{r_\tau} &= \frac{\partial }{\partial r_{\tau}} \bigg[ \int_0^\infty e^{- \int_0^t r_s ds }   w_t dt \bigg] \\
	&= \int_0^\infty \frac{\partial }{\partial r_{\tau}} \bigg[e^{- \int_0^t r_s ds } \bigg] w_t dt  \\
	&= \int_\tau^\infty \frac{\partial }{\partial r_{\tau}} \bigg[e^{- \int_0^t r_s ds } \bigg] w_t dt  \\
	&= - \int_\tau^\infty e^{- \int_0^t r_s ds } \bigg[ \frac{\partial }{\partial r_{\tau}} \int_0^t r_s ds \bigg] w_t dt  \\
	&= - \int_\tau^\infty e^{- \int_0^t r_s ds } w_t dt  \\
	&= - \int_\tau^\infty e^{- (\int_0^t r_s ds + \int_0^\tau r_s ds - \int_0^\tau r_s ds)} w_t dt  \\
	&= - e^{- \int_0^\tau r_s ds } \int_\tau^\infty e^{- \int_\tau^t r_s ds} w_t dt .
\end{align*}
Rewriting, this yields 
\begin{equation*}
	W_{r_\tau} = - \frac{1}{R_{0, \tau}} \int_\tau^\infty \frac{w_t}{R_{\tau, t}} dt.
\end{equation*}
Interpret: This is the effect on lifetime wealth of a change in interest rates at time $\tau$.


\vspace{5mm}
\noindent
\textbf{Step \#2: Euler equation }



\item Use the HJB to derive the euler equation
\begin{equation*}
	\frac{dV_a}{V_a} = \frac{du_c}{u_c} =  (\rho - r_t) dt. 
\end{equation*}

\textbf{Solution:} 
\vspace{3mm}
\noindent
To derive the continuous-time Euler equation for the household problem, I start by writing down the HJB. It is 
\begin{equation*}
	\rho V(t,a) = V_t + u(c) + V_a \Big[ r_t a + w_t - c \Big]. 
\end{equation*}
The envelope condition is 
\begin{equation*}
	(\rho - r_t) V_a = V_{at} +  V_{aa} \Big[ r_t a + w_t - c \Big] .
\end{equation*}
Next, note that since $c = c(t,a)$ we also have $u_c = u_c(t,a)$. Using Ito's lemma, 
\begin{equation*}
	dV_a = V_{at} dt + V_{aa} \Big[ r_t a + w_t - c \Big] dt. 
\end{equation*}
And so directly I obtain 
\begin{equation*}
	\frac{dV_a}{V_a} = \frac{du_c}{u_c} =  (\rho - r_t) dt. 
\end{equation*}


\item In this setting, where $dr_t$ and $dw_t$ are entirely deterministic, the Euler equation is of course also a deterministic equation. Prove the following Lemma (where $R_{s,t}=e^{-\int_s^t r_s ds}$)
\begin{lem} 
	
	The continuous-time Euler equation in this simple setting without uncertainty between two dates $t > s$ is given by
	\begin{equation*}
		u_c(c_s) = e^{- \rho (t-s)} R_{s,t} u_c(c_t). 
	\end{equation*}
\end{lem}

\textbf{Solution}
Using separation of variables, I can integrate and obtain 
\begin{align*}
	\int_0^t \frac{1}{u_c} du_c &=  \int_0^t (\rho - r_s) ds \\
	\ln(u_c(t)) - \ln(u_c(0)) &=  \int_0^t (\rho - r_s) ds,
\end{align*}
or simply 
\begin{equation*}
	u_c(t) = u_c(0) e^{\int_0^t (\rho - r_s) ds} = u_c(0) e^{\int_0^t \rho ds} e^{- \int_0^t r_s ds} = u_c(0) \frac{1 }{e^{-\rho t}R_{0,t}}.
\end{equation*}
Thus, we have integrated back up to the discrete-time Euler equation. 


\item Show that with CRRA utility we can write consumption as 
\begin{equation*}
	c_t = c_0 \bigg[ e^{- \rho t} R_{0,t} \bigg]^\frac{1}{\gamma} .
\end{equation*}



%\vspace{3mm}
%\noindent
%In particular, as long as households remain unconstrained I can express consumption at time $t$ in terms of the consumption choice at any other date. In particular, 
%\begin{equation*}
	%c_t = (u_c)^{-1} \bigg[ u_c(c_0) \frac{1 }{e^{-\rho t}R_{0,t}} \bigg].
%\end{equation*}
%Now the only thing I need to be really careful about is whether I want preference parameters (other than $\rho$) to be time-varying. That is, do I want $u_t = u(c_t)$, or $u_t = u(t, c_t)$? If I decide that preferences are only a function of consumption but otherwise constant, then so is $u_c$ and 
%\begin{equation*}
	%c_t = c_0 (u_c)^{-1} \bigg[ \frac{1 }{e^{-\rho t}R_{0,t}} \bigg].
%\end{equation*}
%In particular, under CRRA where $u_c = c^{-\gamma}$, I get 
%\begin{equation*}
	%c_t = c_0 \bigg[ e^{- \rho t} R_{0,t} \bigg]^\frac{1}{\gamma} .
%\end{equation*}



\vspace{5mm}
\noindent
\textbf{Step \#3: MPC}



\vspace{3mm}
\noindent
All that is left for me to do is put together the lifetime budget constraint with the Euler equation, and then take a derivative. To that end, define the household's MPC as 
\begin{equation*}
	\text{MPC}_{0,t} = \frac{\partial c_t}{\partial a_0} = \frac{\partial c_t}{\partial W}.
\end{equation*}
This definition of course captures the intuition that the household experiences a marginal change in assets (or wealth or unearned income) in period $0$, and then changes his path of consumption expenditures $\{ c_t\}$ accordingly. 

\item Using the lifetime budget and the consumption policy function show
\begin{align*}
	W &= c_0 \int_0^\infty e^{- \frac{\rho}{\gamma} t} R_{0,t}^\frac{1-\gamma}{\gamma} dt \\
\end{align*}


\textbf{Solution:}
I have
\begin{equation*}
	\int_0^\infty e^{- \int_0^t r(s) ds }   c(t) dt = \int_0^\infty \frac{c_t}{R_{0,t}} dt = W.
\end{equation*}
Substituting in for $c_t$, I obtain
\begin{align*}
	W &= \int_0^\infty \frac{c_0 \big[ e^{- \rho t} R_{0,t} \big]^\frac{1}{\gamma} }{R_{0,t}} dt \\
	&= c_0 \int_0^\infty e^{- \frac{\rho}{\gamma} t} R_{0,t}^\frac{1-\gamma}{\gamma} dt \\
\end{align*}


\item To get started with a simple case, assume that $r_t = r$ is constant. Then 
\begin{equation*}
	R_{0,t} = e^{\int_0^t r ds } = e^{ r t}.
\end{equation*}
Let $\kappa =- \frac{1}{\gamma} \big[ \rho - (1-\gamma) r \big] $, prove the following Lemma

\vspace{3mm}
\noindent
\begin{lem}
	The MPC in this setting is constant and given by
	\begin{equation*}
		\text{MPC} = \kappa
	\end{equation*}
	
\end{lem}


\textbf{Solution:}

Differentiating with respect to $W$, 
\begin{equation*}
	1 = \frac{\partial c_0}{\partial W} \int_0^\infty e^{- \frac{1}{\gamma} \big[ \rho - (1-\gamma) r \big] t }  dt.
\end{equation*}
Solving this out,\footnote{
	Mathematically, for any $c \neq 0$, we have 
	\begin{equation*}
		\int e^{c x} dx = \frac{1}{c} e^{cx} + C. 
	\end{equation*}
} assuming for now that $\kappa > 0$, 
\begin{equation*}
	\text{MPC} = \bigg[\int_0^\infty e^{- \kappa t }  dt\bigg]^{-1}  = \bigg[ - \frac{1}{\kappa} e^{- \kappa x}  \bigg|_0^\infty  \bigg]^{-1}  = - \kappa \bigg[e^{- \kappa \infty }  - e^{- \kappa 0}  \bigg]^{-1} = - \kappa \bigg[0 - 1  \bigg]^{-1} = \kappa. 
\end{equation*}


\end{enumerate}

%%%%%%%%%%%%%%%%%%%%%%%%%%%%%%%%%%%%%%%%%%%%%%%%%%%%%%%%%%%%
%%%%%%%%%%%%%%%%%%%%%%%%%%%%%%%%%%%%%%%%%%%%%%%%%%%%%%%%%%%%
\vspace{10mm}
\section*{Problem 6 **}



For the discrete-time, representative agent economy, the Euler equation is given by
\begin{equation*}
	U'(C_t) = \beta E[R_{t+1}^j U'(C_{t+1})]
\end{equation*}
for the potentially stochastic return $R_{t+1}^j$ of any asset $j$, and where $\beta = e^{-\rho}$. We can write 
\begin{equation*}
	1 = e^{-\rho} E\bigg[R_{t+1}^j \frac{U'(C_{t+1})}{U'(C_t)} \bigg].
\end{equation*}
\begin{enumerate}[(a)]
\item Starting with a CRRA utility function, and let $r_{t+1}^j = \ln R_{t+1}^j$ . Show 
\begin{equation*}
	1 = E\bigg[ e^{r_{t+1}^j - \rho -\gamma \Delta \ln  C_{t+1}} \bigg].
\end{equation*}

\textbf{Solution:}
\begin{equation*}
	1 = e^{-\rho} E\bigg[R_{t+1}^j \frac{C_{t+1}^{-\gamma}}{C_t^{-\gamma}} \bigg] = E\bigg[e^{-\rho}  R_{t+1}^j e^{-\gamma \ln(C_{t+1} / C_t)} \bigg] = E\bigg[ R_{t+1}^j e^{- \rho -\gamma  \Delta \ln  C_{t+1}} \bigg].
\end{equation*}
Finally, denote by $r_{t+1}^j = \ln R_{t+1}^j$ the log return of the asset, then we arrive at 
\begin{equation*}
	1 = E\bigg[ e^{r_{t+1}^j - \rho -\gamma \Delta \ln  C_{t+1}} \bigg].
\end{equation*}


\vspace{5mm}
\noindent
\textbf{Euler equation under log-normality.} A log-normal RV is characterized via the representation
\begin{equation*}
	X = e^{\mu + \sigma Z},
\end{equation*}
where $Z$ is a standard normal random variable, and $(\mu,\sigma)$ are the parameters of the log-normal. The mean of the log-normal is given by
\begin{equation*}
	E(X) = e^{\mu + \frac{1}{2} \sigma^2}
\end{equation*}
and its variance by
\begin{equation*}
	\text{Var}(X) = [e^{\sigma^2}-1] e^{2\mu + \sigma^2}.
\end{equation*}


The Euler equation can be further simplified when we assume
\begin{equation*}
	R_{t+1}^j = e^{r_{t+1}^j + \sigma^j \epsilon_{t+1}^j - \frac{1}{2} (\sigma^j)^2},
\end{equation*}
where $\epsilon_{t+1}^j \sim \mathcal{N}(0,1)$, so that 
\begin{equation*}
	R_{t+1}^j  \sim \log \mathcal{N} \bigg(r_{t+1}^j - \frac{1}{2} (\sigma^j)^2, \sigma^j\bigg).
\end{equation*}
Assume also that $\Delta \ln  C_{t+1}$ is conditionally normal, with mean $\mu_{C,t}$ and variance $\sigma_{C,t}^2$. Furthermore assume that the two normals are also jointly, conditionally normal.

\item Derive the asset pricing equation
\begin{equation*}
	1 = E_t [\exp(X_t)],
\end{equation*}
where 
\begin{equation*}
	X_t = -\rho + r_{t+1}^j + \sigma^j \epsilon_{t+1}^j - \frac{1}{2} (\sigma^j)^2 - \gamma \Delta \ln C_{t+1} 
\end{equation*}
so
\begin{equation*}
X_t \sim -\rho + \mathcal{N}\bigg(r_{t+1}^j - \frac{1}{2} (\sigma^j)^2 - \gamma \mu_{C,t}, (\sigma^j)^2 + \gamma^2 \sigma_{C,t}^2 - 2 \rho_{j,C} \gamma \sigma^j \sigma_{C,t} \bigg).
\end{equation*}

\textbf{Solution:}


\item Taking expecations and logs show:
\begin{equation}
	0 =  - \rho + r_{t+1}^j - \frac{1}{2} (\sigma^j)^2 - \gamma E_t(\Delta \ln C_{t+1}) + \frac{1}{2} \text{Var}_t( \sigma^j \epsilon_{t+1}^j - \gamma \Delta \ln C_{t+1})
\end{equation}


\textbf{Solution}
Since $\exp(X_t)$ is log-normal, we have 
\begin{equation*}
	1 = \exp \bigg\{ - \rho + r_{t+1}^j - \frac{1}{2} (\sigma^j)^2 - \gamma \mu_{C,t} + \frac{1}{2} (\sigma^j)^2 + \frac{1}{2} \gamma^2 \sigma_{C,t}^2 - \gamma \rho_{j,C} \sigma^j \sigma_{C,t} \bigg\}.
\end{equation*}
Taking the log from both sides, we finally arrive at
\begin{equation}
	0  = - \rho + r_{t+1}^j - \gamma \mu_{C,t} + \frac{\gamma^2}{2} \sigma_{C,t}^2 - \gamma \rho_{r,C} \sigma^j \sigma_{C,t}.
\end{equation}
More generall, we can leave this as 
\begin{equation}
	0 =  - \rho + r_{t+1}^j - \frac{1}{2} (\sigma^j)^2 - \gamma E_t(\Delta \ln C_{t+1}) + \frac{1}{2} \text{Var}_t( \sigma^j \epsilon_{t+1}^j - \gamma \Delta \ln C_{t+1})
\end{equation}


\item Use this last formula to derive the risk-free rate $r^f$ (Hint: for $j=f$ set $\sigma^f=0$)

\textbf{Solution:}
\vspace{5mm}
\noindent
\textbf{Risk-free rate.} For the risk-free rate, simply plug in $\sigma^f = 0$. Then we have 
\begin{equation}
	0 =  - \rho + r_{t+1}^f - \gamma E_t(\Delta \ln C_{t+1}) + \frac{\gamma^2}{2} \text{Var}_t( \Delta \ln C_{t+1}).
\end{equation}


\item Consider a class of equities with risk $\sigma^E$, we define the equity premium as $$\pi_{t+1}^E \equiv r_{t+1}^E - r_{t+1}^f $$
Show $$\pi_{t+1}^E = \gamma \sigma_{C,E}$$ where $\sigma_{C,E}$ is the covariance between equity returns and log consumption growth. 

\textbf{Solution:}
\vspace{5mm}
\noindent
\textbf{Equity premium.} For the asset class of equities, which we denote by the return $R_{t+1}^E$, we have 
\begin{equation*}
	\pi_{t+1}^E \equiv r_{t+1}^E - r_{t+1}^f = \frac{1}{2} (\sigma^E)^2 - \frac{1}{2} \text{Var}_t( \sigma^E \epsilon_{t+1} - \gamma \Delta \ln C_{t+1}) + \frac{\gamma^2}{2} \text{Var}_t( \Delta \ln C_{t+1}).
\end{equation*}
Now use the formula 
\begin{equation*}
	\text{Var}(aX-bY) = a^2 \text{Var}(X) + b^2 \text{Var}(Y) - 2ab \text{Cov}(X,Y).
\end{equation*}
Thus, 
\begin{equation*}
	\pi_{t+1}^E = \frac{1}{2} (\sigma^E)^2 - \frac{1}{2} \bigg(\text{Var}_t( \sigma^E \epsilon_{t+1}) + \text{Var}_t(\gamma \Delta \ln C_{t+1}) - 2 \sigma^E \gamma \text{Cov}_t(\epsilon_{t+1}, \Delta \ln C_{t+1}) \bigg) + \frac{\gamma^2}{2} \text{Var}_t( \Delta \ln C_{t+1}).
\end{equation*}
Rewriting, we arrive at 
\begin{equation}
	\pi_{t+1}^E = \gamma \sigma_{C,E}
\end{equation}
where $\sigma_{C,E}$ is the covariance between equity returns and log consumption growth. 


\end{enumerate}

\end{document}











