\documentclass[10pt]{beamer}
\usepackage{amsmath,amsfonts,microtype,nicefrac,amssymb, amsthm,centernot}
\usefonttheme{professionalfonts}
\usefonttheme{serif}

\usepackage{palatino}
\usepackage{mathpazo} % or: {newpxtext,newpxmath}
\renewcommand\familydefault{\rmdefault}

\usepackage[utf8]{inputenc}
\usepackage[T1]{fontenc}
\usepackage{textcomp}
\usepackage{bm}

\usepackage{tikz}
\usetikzlibrary{arrows.meta}
\usetikzlibrary{positioning}
\usetikzlibrary{arrows,shapes}
\usepackage{caption, subcaption}

%% BEAMER BUTTON
%\setbeamertemplate{button}{\tikz
%\node[
%	inner xsep = 2pt, 
%	draw = structure!0, 
%	fill = myblue, 
%	rounded corners = 4pt]{\color{white} \tiny\insertbuttontext};
%}


%% ALGORITHM 
\usepackage{algorithm}
\usepackage[noend]{algpseudocode}
\usepackage{multimedia}

%% THEME
\usetheme[frametitleformat=regular,titleformat=regular]{Madrid}
\setbeamerfont{frametitle}{shape=\normalfont}
\setbeamerfont{title}{shape=\normalfont}

%% PATHS
\graphicspath{{./results/}}
\makeatletter
\def\input@path{{../draft/tables/latexData/}}
\makeatother

%% FIGURE ENVIRONMENT 
\usepackage{booktabs,siunitx}

\usepackage{pgfplots} 
%\usepackage[outdir=./figures]{epstopdf}
\usepackage{epstopdf}
\usepackage{float}
\usepackage{graphicx}

\usepackage[absolute,overlay]{textpos}

%% COLORS
%% LINKS
\definecolor{myred}{RGB}{163,32,45}
\definecolor{navyblue}{rgb}{0.05,0.2,0.70}
\definecolor{myblue}{RGB}{0,51,150}
\definecolor{myorange}{RGB}{255,140,0}
\definecolor{myref}{RGB}{160,160,160}
\definecolor{shock}{RGB}{0, 125, 34}%{50, 168, 82}

%% TRANSPARENCY

\usepackage{transparent}

%% BEAMER TEMPLATE
\usetheme{Boadilla}

\makeatother
\setbeamertemplate{itemize items}{\large\raisebox{-0.25ex}{\textbullet}}
\setbeamertemplate{itemize subitem}{\footnotesize\raisebox{0.15ex}{--}}
\setbeamertemplate{itemize subsubitem}{\Tiny\raisebox{0.7ex}{$\blacktriangleright$}}
%\setbeamertemplate{itemize subsubitem}{\color{yellow}$\blacksquare$}

\setbeamertemplate{enumerate item}[default]
\setbeamertemplate{enumerate subitem}{\textbullet}
\setbeamertemplate{footline}{}
\makeatletter
\setbeamertemplate{navigation symbols}{}


%% FORMATTING AUTHORS
%\usepackage{authblk}
\usepackage{url}
\usepackage{multirow}
\usepackage{array}

%% FRAME ITEMIZE SPACING
\usepackage{xpatch}

\xpatchcmd{\itemize}
{\def\makelabel}
{\setlength{\itemsep}{2.5ex}\def\makelabel}
{}
{}

\xpatchcmd{\enumerate}
{\def\makelabel}
{\setlength{\itemsep}{10ex}\def\makelabel}
{}
{}

%% APPENDIX 
\usepackage{appendixnumberbeamer}


%% TITLE AND OPENING
\title{\large M2: Lecture 9}
\subtitle{Models of Consumption: Marginal Propensity to Consume}
\author{Andreas Schaab}
\date{}


\begin{document}
\tikzstyle{every picture}+=[remember picture]
%\everymath{\displaystyle}
\thispagestyle{empty}
\maketitle 
\newpage

\addtocounter{framenumber}{-1}




%%%%%%%%%%%%%%%%%%%%%%%%%%  SLIDE   %%%%%%%%%%%%%%%%%%%%%%%%%%%%%%%%
\begin{frame}{Outline of today's lecture}
\addtocounter{framenumber}{-1}

\begin{enumerate}
\item Closed-form examples continued
\item GHH preferences
\item Return risk
\item Income fluctuations and precautionary savings 
\item Linearization of consumption Euler equation: deterministic and stochastic 
\item Interest rate risk (Gabaix linearization)
\end{enumerate}
\end{frame}



%%%%%%%%%%%%%%%%%%%%%%%%%%  SLIDE   %%%%%%%%%%%%%%%%%%%%%%%%%%%%%%%%
\begin{frame}{1.1. Example: no risk, no income (eat-the-pie)}
\begin{itemize}
\item Time is continuous, $t \in [0, \infty)$

\item Infinitely-lived household's wealth evolves according to 
\begin{equation*}
	\frac{da}{dt} = ra - c
\end{equation*}
given initial wealth position $a_0$, constant interest rate $r$

\item The HJB equation is given by
\begin{equation*}
	\rho V(a) = \frac{1}{\gamma} c(a)^{1-\gamma} + V_a (r a - c(a))
\end{equation*}
where $c(a)$ solves first-order condition $c(a)^{-\gamma} = V_a$
\end{itemize}
\end{frame}


%%%%%%%%%%%%%%%%%%%%%%%%%%  SLIDE   %%%%%%%%%%%%%%%%%%%%%%%%%%%%%%%%
\begin{frame}{}
\begin{itemize}
\item Consider Ansatz $V = Aa^B$, so that $V_a = AB a^{B-1}$ and $c = [AB a^{B-1}]^{- \frac{1}{\gamma}}$

\item Then HJB becomes
\begin{align*}
	\rho A a^B &= \frac{1}{\gamma} \bigg( [AB a^{B-1}]^{- \frac{1}{\gamma}} \bigg)^{1-\gamma} + AB a^{B-1} \bigg( r a - [AB a^{B-1}]^{- \frac{1}{\gamma}} \bigg) \\
	\rho A a^B - r A B a^B &= \frac{\gamma}{1-\gamma} [A B a^{B-1}]^\frac{\gamma-1}{\gamma}
\end{align*}
Plug in $B = 1-\gamma$ and solve for (do this yourself!):
\begin{equation*}
	A = \frac{1}{1-\gamma} \bigg[ \frac{1}{\gamma} \rho - \frac{1-\gamma}{\gamma} r \bigg]^{-\gamma}
\end{equation*}

\item Solution of this model:
\begin{align*}
	V(a) &= \frac{1}{1-\gamma} \kappa^{-\gamma} a^{1-\gamma} \\
	V_a(a) &= \kappa^{-\gamma} a^{-\gamma} \\
	c(a) &= \kappa a,
\end{align*}
\end{itemize}
\end{frame}




%%%%%%%%%%%%%%%%%%%%%%%%%%  SLIDE   %%%%%%%%%%%%%%%%%%%%%%%%%%%%%%%%
\begin{frame}{}
\begin{itemize}
\item Marginal propensity to consume (MPC) is defined as
\begin{equation*}
	c'(a) = \kappa \equiv \frac{1}{\gamma} \rho - \frac{1-\gamma}{\gamma} r
\end{equation*}

\item Consider $u(c) = \log(c)$, with $\gamma \to 1$ (income and substitution effects offset each other), then 
\begin{equation*}
	c'(a) = \kappa = \rho \approx 0.5 \% \text{ annually }
\end{equation*}

\item The standard CRRA calibration with $\gamma = 2$ yields
\begin{equation*}
	c'(a) = \kappa \equiv \frac{1}{2} (\rho + r).
\end{equation*}
In models with uninsurable risk and incomplete financial markets, $r < \rho$

\end{itemize}
\end{frame}




%%%%%%%%%%%%%%%%%%%%%%%%%%  SLIDE   %%%%%%%%%%%%%%%%%%%%%%%%%%%%%%%%
\begin{frame}{1.2. Example: unearned labor income}
\begin{itemize}
\item Infinitely-lived household's wealth now evolves according to 
\begin{equation*}
	\frac{da}{dt} = ra + w - c
\end{equation*}

\item The HJB equation now given by
\begin{equation*}
	\rho V(a) = \frac{1}{\gamma} c(a)^{1-\gamma} + V_a (r a + w - c(a))
\end{equation*}

\item Solution of this model (work this out yourself):
\begin{align*}
	V(a) &= \frac{1}{1-\gamma} \kappa^{-\gamma} \Big( a + \frac{w}{r} \Big)^{1-\gamma} \\
	c(a) &= \kappa \Big(a + \frac{w}{r} \Big)
\end{align*}
with MPC given by: $c'(a) = \kappa$ (as before)

\item Intuition? Human capital affects lifetime wealth but not MPC
\end{itemize}
\end{frame}




%%%%%%%%%%%%%%%%%%%%%%%%%%  SLIDE   %%%%%%%%%%%%%%%%%%%%%%%%%%%%%%%%
\begin{frame}{2. Example: labor supply with GHH preferences}
\begin{itemize}
\item What happens when households can decide how much to work? 

\item Households solve: $\max \int_0^\infty e^{- \rho t} u(c_t, h_t) dt$.  
\begin{align*}
	u(c, h) &= \frac{1}{1-\gamma} \bigg( c - \frac{ h^{1+\eta}}{1+\eta} \bigg)^{1-\gamma} \\
	\rho V(a) &= u(c, h) + (r a + w h - c) \partial_a V(a)
\end{align*}
where FOCs are $u_c = V_a$ and $u_h = - w V_a$, so 
\begin{align*}
	\bigg( c - \frac{h^{1+\eta}}{1+\eta} \bigg)^{-\gamma} &= V_a \\
	\bigg( c - \frac{h^{1+\eta}}{1+\eta} \bigg)^{-\gamma} h^\eta &= w V_a
\end{align*}

\item Putting FOCs together, $c = V_a^{- \frac{1}{\gamma}} + \frac{1}{1 + \eta} w^\frac{1+\eta}{\eta}$ and $h = w^\frac{1}{\eta}$
\end{itemize}
\end{frame}


%%%%%%%%%%%%%%%%%%%%%%%%%%  SLIDE   %%%%%%%%%%%%%%%%%%%%%%%%%%%%%%%%
\begin{frame}{}
\begin{itemize}
\item HJB becomes
\begin{align*}
	\rho V &= \frac{1}{\gamma} V_a^\frac{\gamma-1}{\gamma} + V_a \Big( ra + w h - c \Big) \\
	&= \frac{1}{\gamma} V_a^\frac{\gamma-1}{\gamma} + V_a \bigg( ra + \frac{\eta}{1 + \eta} w^\frac{1+\eta}{\eta} - V_a^{- \frac{1}{\gamma}} \bigg)
\end{align*}

\item Solution of this model given by (work this out yourself):
\begin{align*}
	V(a) &= \frac{1}{1-\gamma} \kappa^{-\gamma} \bigg( a + \frac{\eta}{1+\eta} \frac{w^\frac{1+\eta}{\eta}}{r} \bigg)^{1-\gamma}
\end{align*}
implying
\begin{align*}
	V'(a) &= \kappa^{-\gamma} \bigg( a + \frac{\eta}{1+\eta} \frac{w^\frac{1+\eta}{\eta}}{r} \bigg)^{-\gamma} \\
	c(a) &= \kappa a + \bigg( \frac{\eta}{1+\eta} \frac{\tilde \kappa}{r} + \frac{1}{1+\eta} \bigg) w^\frac{1+\eta}{\eta}
\end{align*}
\end{itemize}
\end{frame}




%%%%%%%%%%%%%%%%%%%%%%%%%%  SLIDE   %%%%%%%%%%%%%%%%%%%%%%%%%%%%%%%%
\begin{frame}{}
\begin{itemize}
\item The household's MPC (out of wealth) is still constant and given by $c'(a) = \text{MPC} = \kappa$ (same $\kappa$ as before)

\item Intuition: we correct for an \textit{effective wage adjustment} in human capital / lifetime wealth but consumption is unaffected by labor supply under GHH

\item This is because GHH shuts down income effects on labor supply!

\end{itemize}
\end{frame}




%%%%%%%%%%%%%%%%%%%%%%%%%%  SLIDE   %%%%%%%%%%%%%%%%%%%%%%%%%%%%%%%%
\begin{frame}{3. Example: Return risk}
\begin{itemize}
\item In the data, we see (a) much higher MPCs and (b) MPCs are higher at low income / wealth

\item So far: we considered deterministic consumption-savings problems. They all yielded (roughly) $\text{MPC} \approx \rho \approx 5\%$ annually. 

\item We now start exploring theories of consumption that can break this and match the data much better

\item Let's start with a simple example of return risk: You can only trade stocks (no bonds) and stocks trade at a stochastic price
\begin{equation*}
	Q dk = Dk - c
\end{equation*}
where $D$ is the dividend
\end{itemize}
\end{frame}


%%%%%%%%%%%%%%%%%%%%%%%%%%  SLIDE   %%%%%%%%%%%%%%%%%%%%%%%%%%%%%%%%
\begin{frame}{}
\begin{itemize}
\item Define net worth as $a = Qk$, so $da = k dQ + Q dk$ by Ito's product rule, noting that $(dk)(dQ) = 0$, so 
\begin{equation*}
	da = \frac{D}{Q} - c + a \frac{dQ}{Q}
\end{equation*}

\item Assume stock prices follow a diffusion process (geometric Brownian): 
\begin{equation*}
	\frac{dQ}{Q} = \mu dt + \sigma dB
\end{equation*}

\item Rewrite wealth as: $da = (\mu_R - c) dt + a \sigma dB$, where $\mu_R = \mu_Q + \frac{D}{Q}$ (dividend + capital gains yield)

\item HJB: 
\begin{equation*}
	\rho V(a) = u(c(a) + V_a (\mu_R a - c) + \frac{1}{2} (a \sigma)^2 V_{aa}
\end{equation*}
\end{itemize}
\end{frame}



%%%%%%%%%%%%%%%%%%%%%%%%%%  SLIDE   %%%%%%%%%%%%%%%%%%%%%%%%%%%%%%%%
\begin{frame}{}
\begin{itemize}
\item Solution of this model is (work this out yourself): 
\begin{align*}
	V(a) &= \frac{1}{1-\gamma} \tilde \kappa^{-\gamma} a^{1-\gamma} \\
	c(a) &= \tilde \kappa a,
\end{align*}
where 
\begin{equation*}
	\kappa \equiv \frac{1}{\gamma} \bigg[ \rho - (1-\gamma) \mu_R + \underbrace{ \gamma(1-\gamma) \frac{\sigma^2}{2}}_\text{Precautionary savings} \bigg] 
\end{equation*}

\item Households' MPC now has a precautionary savings term $\frac{1}{2}(1-\gamma) \sigma^2$

\item The consumption Euler equation for this model (work this out yourself) is:
\begin{equation*}
	\mathbb{E} \bigg( \frac{dc}{c} \bigg) = \frac{r - \rho}{\gamma} dt - \frac{1-\gamma}{2} \sigma^2 dt
\end{equation*}

\item For $\gamma = 2$, precautionary term is negative, so households tilt consumption profile towards the future (hence, precautionary savings)
\end{itemize}
\end{frame}



%%%%%%%%%%%%%%%%%%%%%%%%%%  SLIDE   %%%%%%%%%%%%%%%%%%%%%%%%%%%%%%%%
\begin{frame}{4. Income fluctuations}
\begin{itemize}
\item We now work through arguably \textit{the} benchmark model of consumption in macro

\item Households face uninsurable income risk $dz_t$. They can trade a bond, $a_t$ allowing for partial self-insurance, but face borrowing constraint $a_t \geq 0$.

\item Sequence problem:
\begin{equation*}
	V_0 = \max_{ \{ c_t \}_{t \geq 0} } \mathbb{E}_0 \int_0^\infty e^{- \rho t} u(c_t) dt
\end{equation*}
subject to 
\begin{align*}
	da_t &= r a_t + e^{z_t} - c_t \\
	dz_t &= - \theta z_t dt + \sigma dB_t
\end{align*}

\item We will use isoelastic (CRRA) preferences with $u(c) = \frac{1}{1-\gamma} c^{1-\gamma}$
\end{itemize}
\end{frame}


%%%%%%%%%%%%%%%%%%%%%%%%%%  SLIDE   %%%%%%%%%%%%%%%%%%%%%%%%%%%%%%%%
\begin{frame}{}
\begin{itemize}
\item Household states are $(a, z)$, so recursive representation is 
\begin{equation*}
	\rho V = u(c) + (r a + e^z - c) V_a - \theta z V_z + \frac{\sigma^2}{2} V_{zz}
\end{equation*}
where $c(a, z) = V_a(a, z)^{- \frac{1}{\gamma}}$

\item First step in deriving Euler equation: envelope condition
\begin{equation*}
	(\rho - r) V_a = (ra + e^z - c) V_{aa} - \theta z V_{z a} + \frac{\sigma^2}{2} V_{zz a }
\end{equation*}

\item Using Ito's lemma, noting $V_a$ is function of $(a_t, z_t)$, 
\begin{align*}
	dV_a &= V_{aa} da + V_{az} dz + \frac{1}{2} V_{azz} (dz)^2 \\
	&= V_{aa} (r a + e^z - c) dt + V_{az} (- \theta z dt + \sigma dB) + \frac{\sigma^2}{2} V_{azz} dt \\
	&= (\rho - r)V_a dt +  \sigma V_{az} dB
\end{align*}
\end{itemize}
\end{frame}


%%%%%%%%%%%%%%%%%%%%%%%%%%  SLIDE   %%%%%%%%%%%%%%%%%%%%%%%%%%%%%%%%
\begin{frame}{}
\begin{itemize}
\item Using $u_c = V_a$ and $V_{az} = u_{cc} c_z$, Euler equation for marginal utility is
\begin{equation*}
	\frac{du_c}{u_c} = (\rho - r) dt + \sigma \frac{u_{cc} c_z}{u_c} dB
\end{equation*}

\item For isoelastic (CRRA), we have $\frac{u_{cc}}{u_c} = - \frac{\gamma}{c}$

\item Consumption Euler equation (derivation is on PSET):
\begin{equation*}
	\frac{dc}{c} = \frac{r - \rho}{\gamma} dt + \frac{1+\gamma}{2} \bigg( \frac{\sigma c_z}{c} \bigg)^2 dt + \frac{c_z}{c} \sigma dB,
\end{equation*}
where $c_z$ is $\approx$ the marginal propensity out of income shocks

\item The term $ \frac{1+\gamma}{2} \big( \frac{\sigma c_z}{c} \big)^2$ captures a precautionary savings motive due to uncertainty about future income fluctuations (that are not insurable)

\item As always: Euler equation doesn't hold at the borrowing constraint
\end{itemize}
\end{frame}









%%%%%%%%%%%%%%%%%%%%%%%%%%%%%%%%%%%%%%%%%%%%%%%%%%%%%%%%%%%%%%
%%%%%%%%%%%%%%%%%%%%%%%%%%%%%%%%%%%%%%%%%%%%%%%%%%%%%%%%%%%%%%
%%%%%%%%%%%%%%%%%%%%%%%%%%%%%%%%%%%%%%%%%%%%%%%%%%%%%%%%%%%%%%

\appendix


\end{document}









