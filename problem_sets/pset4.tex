\documentclass[11pt]{extarticle}
\usepackage{fullpage,amsmath,amsfonts,microtype,nicefrac,amssymb, amsthm}
\usepackage[left=1in, bottom=1in, top=1in, right = 1in]{geometry}
\usepackage{textcomp}
\usepackage{mathpazo}
\usepackage{mathrsfs}
\usepackage[T1]{fontenc}
\usepackage[utf8]{inputenc}
\usepackage[english]{babel}
\usepackage{graphicx}

\usepackage{microtype}

\usepackage{bm}
\usepackage{dsfont}
\usepackage{enumerate}
\usepackage{ragged2e}

\setlength{\parindent}{24pt}
\setlength{\jot}{8pt}


\usepackage[shortlabels]{enumitem}


%% FOOTNOTES
\usepackage[bottom]{footmisc}
\usepackage{footnotebackref}


%% FIGURE ENVIRONMENT
%\graphicspath{{}}
\usepackage[margin=15pt, font=small, labelfont={bf}, labelsep=period]{caption}
\usepackage{subcaption}
\captionsetup[figure]{name={Figure}, position=above}
\usepackage{float}
\usepackage{epstopdf}


%% NEW COMMANDS
\renewcommand{\baselinestretch}{1.25} 
\renewcommand{\qedsymbol}{$\blacksquare$}
\newcommand{\R}{\mathbb{R}}
\newcommand{\indep}{\mathrel{\text{\scalebox{1.07}{$\perp\mkern-10mu\perp$}}}}
\renewcommand{\b}{\begin}
\newcommand{\e}{\end}

%% NEWTHEOREM
\theoremstyle{plain}
\newtheorem{thm}{Theorem}
\newtheorem{lem}[thm]{Lemma}
\newtheorem{prop}[thm]{Proposition}

\theoremstyle{definition}
\newtheorem{defn}[thm]{Definition}
\newtheorem{ex}[thm]{Example}
\newtheorem{remark}[thm]{Remark}
\newtheorem{cor}[thm]{Corollary}

%% LINKS and COLORS
\usepackage[dvipsnames]{xcolor}
\usepackage{hyperref}
\definecolor{myred}{RGB}{163, 32, 45}
\hypersetup{
	%backref=true,
	%pagebackref=true,
	colorlinks=true,
	urlcolor=myred,
	citecolor=myred, 
	linktoc=all,     
	linkcolor=myred,
}

%% TABLE OF CONTENTS
\addto\captionsenglish{
	\renewcommand{\contentsname}
	{}% This removes the heading over the table of contents.
}



%%%%%%%%%%%%%%%%%%%%%%%%%%%%%%%%%%%%%%%%%%%%%%%%%%%%%%%%%%%%
%%%%%%%%%%%%%%%%%%%%%%%%%%%%%%%%%%%%%%%%%%%%%%%%%%%%%%%%%%%%
%%%%%%%%%%%%%%%%%%%%%%            END PREAMBLE           %%%%%%%%%%%%%%%%%%%%%%%%
%%%%%%%%%%%%%%%%%%%%%%%%%%%%%%%%%%%%%%%%%%%%%%%%%%%%%%%%%%%%
%%%%%%%%%%%%%%%%%%%%%%%%%%%%%%%%%%%%%%%%%%%%%%%%%%%%%%%%%%%%

\title{Dynamic Optimization: Problem Set \#4: Solutions}

\author{Andreas Schaab}

\date{Fall, 2022}



\begin{document}

\maketitle
\thispagestyle{empty}
\setcounter{page}{0}


%%%%%%%%%%%%%%%%%%%%%%%%%%%%%%%%%%%%%%%%%%%%%%%%%%%%%%%%%%%%
%%%%%%%%%%%%%%%%%%%%%%%%%%%%%%%%%%%%%%%%%%%%%%%%%%%%%%%%%%%%
\vspace{10mm}
\section*{Problem 1 (hard)}

In this problem, we work through a simple variant of the seminal Brunnermeier and Sannikov (2014, AER) paper.

We study the consumption, savings, and investment problem of an agent. We denote by $\{ D_t \}_{t \geq 0}$ the \textit{dividend stream} and by $\{ Q_t \}_{t \geq 0}$ the asset price. Assume that the asset price evolves according to 
\begin{equation*}
	\frac{ dQ }{Q} = \mu_Q dt + \sigma_Q dB,
\end{equation*}
where you can interpret $\mu_Q$ and $\sigma_Q$ as simple constants (alternatively, think of them as more complicated objects that would be determined in general equilibrium, which we abstract from here).


This is a model of two assets, capital and bonds. Bonds pay the riskfree rate of return $r_t$. Capital is accumulated and owned by the agent. Capital is traded at price $Q_t$ and yields dividends at rate $D_t$.


The key interesting feature of this problem is that the agent faces both (idiosyncratic) earnings risk and (aggregate) asset price risk.


\vspace{10mm}
%\textbf{Gerard: clean up the following}
Households take as given all aggregate prices and behave according to preferences given by
\begin{equation*}
	\mathbb{E}_0 \int_0^\infty e^{- \rho t} u(c_t) dt.
\end{equation*}

Households consume and save, investing their wealth into bonds and capital. Letting $k$ denote a household's units of capital owned and $b$ units of bonds, the budget constraint is characterized by
\begin{align*}
	dk_t &= \Phi(\iota_t) k_t - \delta k_t \\
	db_t &= r_t b_t + D_t k_t + w_t z_t - c_t -  \iota_t k_t.
\end{align*}
The rate of investment is given by $\iota_t$. Investment adjustment costs are captured by the concave technology $\Phi$. Dividends are paid to households in units of the numeraire, thus entering the equation for $db$. The law of motion for household earnings are given by
\begin{equation*}
	dz = \mu_z dt + \sigma_z dW. 
\end{equation*}

\begin{enumerate}[(a)]
 \item Interpret all terms of the two budget constraints


 It is convenient to rewrite the household problem in terms of liquid net worth, defined by the equations 
\begin{align*}
	\theta n &= Q k \\
	(1-\theta) n &= b,
\end{align*}
so that total liquid net worth is $n = Qk + b$.

\item How would you refer to $\theta$? Assume households' capital accumulation is non-stochastic. That is, there is no capital quality risk. Show that the liquid net worth evolves according to 

\begin{equation*}
	dn = rn +  \theta n \bigg[ \frac{D - \iota }{Q} +  \frac{dQ}{Q} + \Phi(\iota) - \delta - r \bigg]  + w z - c .
\end{equation*}
 


\item Argue why the choice of $\iota$ is entirely static in this setting and show it is only a function of capital $\iota = \iota(Q)$ 



  Recall also that households take as given aggregate ``prices'' $(r, w, D, Q)$.  This will allow us to work with a simplified representation. Define 
\begin{equation*}
	dR = \underbrace{\frac{D - \iota (Q) }{Q}}_\text{Dividend yield} dt + \underbrace{\Big[\Phi(\iota(Q)) - \delta \Big] dt +  \frac{dQ}{Q}}_\text{Capital gains} \equiv \mu_R dt + \sigma_R dB
\end{equation*}
to be the effective rate of return on households' capital investments. And where
\begin{align*}
	\mu_R &= \frac{D - \iota (Q) }{Q}  + \Phi(\iota(Q)) - \delta + \mu_Q \\
	\sigma_R &= \sigma_Q. 
\end{align*}
After solving for $\iota = \iota(Q)$, this return is exogenous from the perspective of the household: it depends on macro conditions and prices, but not on the particular portfolio composition of the household.

\item Show that the law of motion of the household's liquid net worth satisfies the following equation. Why do you think using liquid net worth is useful? And why do we want this law of motion?

\begin{equation*}
	dn = rn + \theta n (\mu_R - r) + wz - c + \theta n \sigma_R dB. 
\end{equation*}



\vspace{5mm}
\noindent
\textbf{State space.} We denote the agent's individual states by $(n, z)$. We assume there is a variable $\Gamma_t$ that evolves according to the diffusion process 
\begin{equation*}
	d \Gamma = \mu_\Gamma dt + \sigma_\Gamma dB.
\end{equation*}
We call $\Gamma$ the \textit{aggregate state of the economy}. We can write all capital prices as functions of this aggregate state, that is, 
\begin{equation*}
	r_t = r(\Gamma_t), \hspace{5mm}	D_t = D(\Gamma_t), \hspace{5mm}	Q_t = Q(\Gamma_t).
\end{equation*}
This now allows us to write the household problem recursively with $\Gamma$ as an extra state variable. Note that otherwise, we would need to keep track of three different aggregate state variables. 


\vspace{5mm}
\noindent
\textbf{Recursive representation.} 
The household problem can therefore be written in terms of the household state variables $(n,z)$ as well as the aggregate state space $\Gamma$. 
\item Show that the household problem satisfies the following HJB
\begin{align*}
	\rho V(n,z,\Gamma) = \max_{c,\theta} \bigg\{  & u(c) + V_n \Big[ rn + \theta n (\mu_R - r) + wz - c \Big] + \frac{1}{2} V_{nn} (\theta n \sigma_R)^2 +  V_z \mu_z + \frac{1}{2} V_{zz} \sigma_z^2  \\
	& + V_{n \Gamma} \theta n \sigma_R \sigma_\Gamma + V_\Gamma \mu_\Gamma + \frac{1}{2} \sigma_\Gamma^T V_{\Gamma \Gamma} \sigma_\Gamma \bigg\},
\end{align*}
where you can assume that $\mathbb{E}(dW dB) = 0$. This means that households' earnings risk is uncorrelated with the aggregate state $\Gamma$.



\item Derive the first-order conditions for consumption and portfolio choice


\item (hard) Prove the following lemma (use envelope condition and apply Ito's lemma to $V_n(h,y,\Gamma)$)

\begin{lem}\label{lem:MUEuler}
	
	The household Euler equation for marginal utility is given by
	\begin{equation*}
		\frac{d u_c}{u_c} = (\rho - r) dt -  \frac{\mu_R - r}{\sigma_R} dB  - \gamma \frac{c_z}{c} \sigma_z dW .
	\end{equation*}
	
\end{lem}


\item  (hard)  Proof the following lemma 

\begin{lem}
	
	The household Euler equation for consumption is given by
	\begin{align*}
		\frac{dc}{c} = & \frac{r - \rho}{\gamma} dt + \frac{1}{2} (1+\gamma) \bigg[ \bigg( \frac{\mu_R - r}{\gamma \sigma_R} \bigg)^2 + \bigg( \frac{c_z}{c} \sigma_z \bigg)^2 \bigg] dt +  \frac{\mu_R - r}{\gamma \sigma_R} dB  + \frac{c_z}{c} \sigma_z dW .
	\end{align*}	
\end{lem}





%\vspace{5mm}
%\noindent
%\textbf{Comparison to Brunnermeier and Sannikov.} It turns out that I already derived the analog to Bru-San's Proposition II.2 a long time ago. I have 
%\begin{align*}
%	\frac{dV_n}{V_n} = & (\rho - r)  -   \frac{\mu_R - r}{\sigma_R} dB  + \frac{V_{nz}}{V_n} \sigma_z dW,
%\end{align*}
%where they call $V_n = \theta$. For the sake of comparison, let $V_n = \theta^\text{BS}$. Then, 
%\begin{equation*}
%	\frac{d \theta^\text{BS}}{\theta^\text{BS}} = \mu_\theta dt + \sigma_\theta^B dB + \sigma_\theta^W dW,
%\end{equation*}
%where 
%\begin{align*}
%	\mu_\theta  &= \rho - r \\
%	\underbrace{- \sigma_Q \sigma_\theta^B }_\text{Risk premium} &= \underbrace{ \frac{D - \iota (Q) }{Q}  + \Phi(\iota(Q)) - \delta + \mu_Q  - r }_\text{Expected excess return on capital} \\
%	\sigma_\theta^W  &= \frac{V_{nz}}{V_n} \sigma_z .
%\end{align*}
%Of course, they don't have earnings risk. 


\end{enumerate}


%%%%%%%%%%%%%%%%%%%%%%%%%%%%%%%%%%%%%%%%%%%%%%%%%%%%%%%%%%%%
%%%%%%%%%%%%%%%%%%%%%%%%%%%%%%%%%%%%%%%%%%%%%%%%%%%%%%%%%%%%
\vspace{10mm}
\section*{Problem 2}

\textbf{Credit:} Gabriel Chodorow-Reich

\vspace{5mm}
\noindent
%\textbf{Gerard:} add ``Homework4'' here (whatever you find valuable)

\subsection*{ Problem 2.1}
 IES. Consider the isoelastic utility function:
$$
u(c)=\frac{c^{1-\frac{1}{\sigma}}-1}{1-\frac{1}{\sigma}},
$$
where $\sigma>0$.

\begin{enumerate}[(a)]
\item Prove that  $\lim _{\sigma \rightarrow 1} u(c)=\ln (c)$. (Hint: use l'hopital's rule) 


\item The coefficient of relative prudence is $$-\frac{u'''(c)c}{u''(c)}$$ derive it. What is it related to?


\end{enumerate}

\subsection*{Problem 2.2}

Euler equation with uncertainty. Consider an agent who lives for two periods, $t=0,1$. The agent can freely borrow or lend at interest rate $r$. The agent has period preferences given by $u\left(c_t\right)=\frac{c_t^{1-\frac{1}{\sigma}}}{1-\frac{1}{\sigma}}$. In period 0, the agent discounts the utility of future consumption at rate $\beta$. The agent receives income $y_0$ in period 0 , but has uncertain income in period 1. Therefore, the agent maximizes expected utility subject to certain income $y_0$ and expected income $E\left[y_1\right]$.

\begin{enumerate}[(a)]

\item Derive the Euler Equation



\item Suppose $\sigma \rightarrow \infty$. What conditions are placed on $\beta(1+r)$ if the agent has positive consumption in period 0? Interpret your answer in light of question 1. Recall that we call $\sigma$ the intertemporal elasticity of substitution.


\item Assume $y_1 \in\left\{y_L, y_H\right\}$, with $y_H>y_L$. Argue that this implies $c_1 \in\left\{c_L, c_H\right\}$, with $c_H>c_L$, for some unknown values $c_L$ and $c_H$. Let $b_0$ denote period 0 savings. Then $c_1=(1+r) b_0+y_1$



\end{enumerate}


\subsection*{Problem 2.3}
 Income and substitution effects. Recall the consumption function from the 2 period model derived in lecture, with for simplicity $y_1=0$ :
$$
c_0=\frac{1}{1+\beta^\sigma(1+r)^{\sigma-1}} y_0 .
$$

\begin{enumerate}[(a)]

\item Differentiate $c_0$ with respect to $1+r$.



\item Explain why your answer to (a) shows that period 0 consumption responds positively to a decrease in the real interest rate if and only if $\sigma>1$. 

\item  Show that:
$$
c_1=\left[\frac{\beta^\sigma(1+r)^\sigma}{1+\beta^\sigma(1+r)^{\sigma-1}}\right] y_0 .
$$



\item Show that if $\sigma>0$, then $\frac{\partial c_1}{\partial(1+r)}>0$.

\item  Why does the response of $c_0$ to $(1+r)$ depend on the value of $\sigma$, but the response of $c_1$ does not? (Hint: your answer should reference the direction of income and substitution effects for consumption in each period.)

\end{enumerate}

%%%%%%%%%%%%%%%%%%%%%%%%%%%%%%%%%%%%%%%%%%%%%%%%%%%%%%%%%%%%
%%%%%%%%%%%%%%%%%%%%%%%%%%%%%%%%%%%%%%%%%%%%%%%%%%%%%%%%%%%%
\vspace{10mm}
\section*{Problem 3}

We now solve a version of the income fluctuations problem in continuous time. In discrete time, the problem is as follows:


\paragraph{Discrete time.} The canonical buffer stock model in discrete-time is a variant of the life-cycle model of consumption featuring idiosyncratic income risk. A household's preferences are given by 
\begin{equation}
	\sum_{t=0}^\infty \beta^t u(c_t),
\end{equation}
and the critical assumption is that $\beta R = 1$, where $\beta$ is the household's discount parameter and $R$ is the gross interest rate. In a model without uncertainty, this assumption would imply a constant consumption profile over time.

The household's budget constraint is encoded in its evolution of wealth, given by 
\begin{equation}
	a_{t+1} = R(y_t + a_t - c_t),
\end{equation}
where $y_t \sim^{iid} F$ is an income shock that is independent and identically distributed over time. We may furthermore assume that agents face an exogenously determined, uniform borrowing constraint, $a_t \geq \underline{a}$, that is tighter than the natural borrowing constraint. 

The recursive problem of the household can then be written using the Bellman equation
\begin{equation}
	v(a_t) = \max_{c_t} u(c_t) + \beta E_t[v(a_{t+1})],
\end{equation}
subject to 
\begin{align*}
	a_{t+1} &= R(a_t + y_t - c_t) \\
	a_{t} \geq 0.
\end{align*}


\vspace{5mm}
\noindent
\textbf{Continuous time without borrowing constraint.} In continuous time, the evolution of wealth is given by 
\begin{equation}
	da_t = (ra_t - c_t) dt + \sigma dB_t,
\end{equation}
where the Brownian term is the appropriate analog to independent and identically distributed income shocks in discrete time. 

We now assume that households are not subject to a borrowing constraint.

\begin{enumerate}[(a)]

\item Write the generator for the stochastic process of wealth. Use it to derive the HJB:
\begin{equation}
	\rho v(a_t) = \max_{c_t} u(c_t) + v'(a_t) [ra_t - c_t] + \frac{1}{2} v''(a_t) \sigma^2.
\end{equation}


The HJB in our simple setting is given by an ordinary instead of a partial differential equation because we only have one state variable. 
\item Show we have the following HJB envelop condition (take FOC, differentiate FOC wrt a and use envelope condition)
\begin{equation*}
	(\rho - r) v'(a_t) = v''(a_t) [ra_t - c(a_t)] + \frac{1}{2} v'''(a_t) \sigma^2.
\end{equation*}

To obtain a closed-form solution to the buffer-stock model, let utility be log so that $u(c_t) = \ln(c_t)$. We will write the consumption policy function as $c(a_t)$. 
\item Show the HJB satisfies
\begin{equation*}
	\rho \kappa + \frac{\rho}{c'(a_t)} \ln[c(a_t)] = \ln[c(a_t)] + \frac{r a_t}{c(a_t)} - 1 - \frac{\sigma^2}{2} \frac{c'(a_t)}{c(a_t)^2}.
\end{equation*}

 We can see immediately that the $c(a_t)^2$ term in the denominator on the RHS is going to make solving for a policy function $c$ very difficult. It also implies a simple fix, however: Consider an alternative wealth evolution equation given by $da_t = (r a_t - c_t)dt + \sigma a_t dB_t$. 
\item Show the HJB becomes $$\rho v(a_t) = u(c_t) + v'(a_t)[ra_t -c_t] + \frac{\sigma^2}{2} a_t^2 v''(a_t)$$ 


\item Using the Ansatz that the policy function for consumption is linear in wealth, in particular $c(a_t) = \rho a_t$, show \begin{align}
	v(a_t) &= \frac{1}{\rho} \ln(\rho a_t) + \frac{r - \rho}{\rho^2} - \frac{\sigma^2}{2\rho^2}.
\end{align} 

\end{enumerate}


%%%%%%%%%%%%%%%%%%%%%%%%%%%%%%%%%%%%%%%%%%%%%%%%%%%%%%%%%%%%
%%%%%%%%%%%%%%%%%%%%%%%%%%%%%%%%%%%%%%%%%%%%%%%%%%%%%%%%%%%%
\vspace{10mm}
\section*{Problem 4}

\textbf{Credit:} David Laibson

\vspace{5mm}
\noindent
Solve the ``\textit{eat the pie problem}'' on problem set (PSET) \#2 here: \url{https://projects.iq.harvard.edu/econ2010c/problem-sets-david-laibson}.





%%%%%%%%%%%%%%%%%%%%%%%%%%%%%%%%%%%%%%%%%%%%%%%%%%%%%%%%%%%%
%%%%%%%%%%%%%%%%%%%%%%%%%%%%%%%%%%%%%%%%%%%%%%%%%%%%%%%%%%%%
\vspace{10mm}
\section*{Problem 5}


To illustrate this approach, I will consider a simple problem where household preferences are 
\begin{equation*}
	\max_{\{c_t\}} \int_0^\infty e^{- \rho t} u(c_t) dt. 
\end{equation*}
There is no uncertainty. The household budget constraint is given by
\begin{equation*}
	da_t = r_t a_t + w_t - c_t. 
\end{equation*}
Aggregate prices follow a deterministic process, $\bm{r} = \{r_t\}$ and $\bm{w} = \{w_t\}$. Finally, the household starts with an initial wealth position $a_0$, and wealth is the only state variable. 



\textcolor{red}{The next Lemma derives more formally the steps we have already been using in the lectures and psets}

\vspace{5mm}
\noindent
\textbf{Step \#1: Lifetime budget constraint}

\vspace{3mm}
\noindent
\begin{lem} (Lifetime Budget Constraint) For any linear ODE 
	\begin{equation*}
		\frac{dy}{dt} = r(t) y(t) + x(t)
	\end{equation*}
	we have the integration result 
	\begin{equation*}
		y(T)  = y(0) e^{\int_0^T r(s) ds} +  \int_0^T e^{\int_t^T r(s) ds }  x(t) dt.
	\end{equation*}
	
\end{lem}
\vspace{8mm}
\begin{proof}
	Consider any ODE
	\begin{equation*}
		\frac{dy}{dt} = r(t) y(t) + x(t).
	\end{equation*}
	Using an integrating factor approach, we have 
	\begin{equation*}
		e^{\int -r(s) ds} \frac{dy}{dt} - e^{\int -r(s) ds}  r(t) y(t) = e^{\int -r(s) ds} x(t).
	\end{equation*}
	The LHS can then be written as a product rule, so that 
	\begin{equation*}
		\frac{d}{dt} \bigg( y(t) e^{\int -r(s) ds} \bigg) =  \frac{dy}{dt} e^{\int -r(s) ds}  + y(t) e^{\int -r(s) ds} \frac{d}{dt} \bigg( \int -r(s) ds \bigg)  = e^{\int -r(s) ds} x(t).
	\end{equation*}
	The last derivative follows from the fundamental theorem of calculus for indefinite integrals. 
	
	Alternatively, since I know that I will work on the definite time horizon $t \in [0,T]$, I can choose a slightly different integrating factor: I can write $u(t) = e^{- \int_0^t r(s) ds}$, so 
	\begin{equation*}
		e^{\int_0^t -r(s) ds} \frac{dy}{dt} - e^{\int_0^t -r(s) ds}  r(t) y(t) = e^{\int_0^t -r(s) ds} x(t).
	\end{equation*}
	Using Leibniz rule, I have 
	\begin{equation*}
		\frac{d}{dt} \bigg( y(t) e^{\int_0^t -r(s) ds} \bigg) = \frac{dy}{dt}  e^{\int_0^t -r(s) ds} + y(t) e^{\int_0^t -r(s) ds} \frac{d}{dt} \bigg( \int_0^t -r(s) ds \bigg) =  \frac{dy}{dt}  e^{\int_0^t -r(s) ds} - y(t) e^{\int_0^t -r(s) ds} r(t).
	\end{equation*}
	Now, I have 
	\begin{equation*}
		\frac{d}{dt} \bigg( y(t) u(t) \bigg) = u(t) x(t).
	\end{equation*}
	Finally, this implies 
	\begin{equation*}
		y(T) u(T) - y(0) u(0) = \int_0^T u(t) x(t),
	\end{equation*}
	or, noting $u(0) = 1$, 
	\begin{equation*}
		y(T) e^{- \int_0^T r(s) ds} = y(0) +  \int_0^T e^{- \int_0^t r(s) ds} x(t) dt.
	\end{equation*}
	Rearranging, 
	\begin{equation*}
		y(T)  = y(0) e^{\int_0^T r(s) ds} +  \int_0^T e^{\int_t^T r(s) ds }  x(t) dt.
	\end{equation*}
\end{proof}
\begin{enumerate}[(a)]


\item Use this result to characterize the initial lifetime wealth $W$ of the household. We have $W = W(a_0, \bm{r}, \bm{w})$. 


\vspace{5mm}
\noindent
It will be a useful exercise to characterize the response of initial lifetime wealth $dW$ to a general perturbation of this economy, $\{da_0, d \bm r, d \bm w\}$. We have:
\begin{equation*}
	dW = W_{a_0} da_0 + W_{\bm r} d \bm{r} + W_{ \bm w} d \bm w. 
\end{equation*}

\item  Work out each of these derivatives ($ W_{a_0}$, $W_{\bm r}$ and $W_{ \bm w}$) and interpret 

\vspace{5mm}
\noindent
\textbf{Step \#2: Euler equation}



\item Use the HJB to derive the euler equation
\begin{equation*}
	\frac{dV_a}{V_a} = \frac{du_c}{u_c} =  (\rho - r_t) dt. 
\end{equation*}


\item In this setting, where $dr_t$ and $dw_t$ are entirely deterministic, the Euler equation is of course also a deterministic equation. Prove the following Lemma (where $R_{s,t}=e^{-\int_s^t r_s ds}$)
\begin{lem} 
	
	The continuous-time Euler equation in this simple setting without uncertainty between two dates $t > s$ is given by
	\begin{equation*}
		u_c(c_0) = e^{- \rho (t-s)} R_{s,t} u_c(c_t). 
	\end{equation*}
\end{lem}

\item Show that with CRRA utility we can write consumption as 
\begin{equation*}
	c_t = c_0 \bigg[ e^{- \rho t} R_{0,t} \bigg]^\frac{1}{\gamma} .
\end{equation*}

%\vspace{3mm}
%\noindent
%In particular, as long as households remain unconstrained I can express consumption at time $t$ in terms of the consumption choice at any other date. In particular, 
%\begin{equation*}
	%c_t = (u_c)^{-1} \bigg[ u_c(c_0) \frac{1 }{e^{-\rho t}R_{0,t}} \bigg].
%\end{equation*}
%Now the only thing I need to be really careful about is whether I want preference parameters (other than $\rho$) to be time-varying. That is, do I want $u_t = u(c_t)$, or $u_t = u(t, c_t)$? If I decide that preferences are only a function of consumption but otherwise constant, then so is $u_c$ and 
%\begin{equation*}
	%c_t = c_0 (u_c)^{-1} \bigg[ \frac{1 }{e^{-\rho t}R_{0,t}} \bigg].
%\end{equation*}
%In particular, under CRRA where $u_c = c^{-\gamma}$, I get 
%\begin{equation*}
	%c_t = c_0 \bigg[ e^{- \rho t} R_{0,t} \bigg]^\frac{1}{\gamma} .
%\end{equation*}



\vspace{5mm}
\noindent
\textbf{Step \#3: MPC}



\vspace{3mm}
\noindent
All that is left for me to do is put together the lifetime budget constraint with the Euler equation, and then take a derivative. To that end, define the household's MPC as 
\begin{equation*}
	\text{MPC}_{0,t} = \frac{\partial c_t}{\partial a_0} = \frac{\partial c_t}{\partial W}.
\end{equation*}
This definition of course captures the intuition that the household experiences a marginal change in assets (or wealth or unearned income) in period $0$, and then changes his path of consumption expenditures $\{ c_t\}$ accordingly. 



\item Using the lifetime budget and the consumption policy function show
\begin{align*}
	W &= c_0 \int_0^\infty e^{- \frac{\rho}{\gamma} t} R_{0,t}^\frac{1-\gamma}{\gamma} dt \\
\end{align*}
\item To get started with a simple case, assume that $r_t = r$ is constant. Then 
\begin{equation*}
	R_{0,t} = e^{\int_0^t r ds } = e^{ r t}.
\end{equation*}
Let $\kappa =- \frac{1}{\gamma} \big[ \rho - (1-\gamma) r \big] $, prove the following Lemma

\vspace{3mm}
\noindent
\begin{lem}
	The MPC in this setting is constant and given by
	\begin{equation*}
		\text{MPC} = \kappa
	\end{equation*}
	
\end{lem}

\end{enumerate}

%%%%%%%%%%%%%%%%%%%%%%%%%%%%%%%%%%%%%%%%%%%%%%%%%%%%%%%%%%%%
%%%%%%%%%%%%%%%%%%%%%%%%%%%%%%%%%%%%%%%%%%%%%%%%%%%%%%%%%%%%
\vspace{10mm}
\section*{Problem 6}



For the discrete-time, representative agent economy, the Euler equation is given by
\begin{equation*}
	U'(C_t) = \beta E[R_{t+1}^j U'(C_{t+1})]
\end{equation*}
for the potentially stochastic return $R_{t+1}^j$ of any asset $j$, and where $\beta = e^{-\rho}$. We can write 
\begin{equation*}
	1 = e^{-\rho} E\bigg[R_{t+1}^j \frac{U'(C_{t+1})}{U'(C_t)} \bigg].
\end{equation*}
\begin{enumerate}[(a)]
\item Starting with a CRRA utility function, and let $r_{t+1}^j = \ln R_{t+1}^j$ . Show 
\begin{equation*}
	1 = E\bigg[ e^{r_{t+1}^j - \rho -\gamma \Delta \ln  C_{t+1}} \bigg].
\end{equation*}




\vspace{5mm}
\noindent
\textbf{Euler equation under log-normality.} A log-normal RV is characterized via the representation
\begin{equation*}
	X = e^{\mu + \sigma Z},
\end{equation*}
where $Z$ is a standard normal random variable, and $(\mu,\sigma)$ are the parameters of the log-normal. The mean of the log-normal is given by
\begin{equation*}
	E(X) = e^{\mu + \frac{1}{2} \sigma^2}
\end{equation*}
and its variance by
\begin{equation*}
	\text{Var}(X) = [e^{\sigma^2}-1] e^{2\mu + \sigma^2}.
\end{equation*}


The Euler equation can be further simplified when we assume
\begin{equation*}
	R_{t+1}^j = e^{r_{t+1}^j + \sigma^j \epsilon_{t+1}^j - \frac{1}{2} (\sigma^j)^2},
\end{equation*}
where $\epsilon_{t+1}^j \sim \mathcal{N}(0,1)$, so that 
\begin{equation*}
	R_{t+1}^j  \sim \log \mathcal{N} \bigg(r_{t+1}^j - \frac{1}{2} (\sigma^j)^2, \sigma^j\bigg).
\end{equation*}
Assume also that $\Delta \ln  C_{t+1}$ is conditionally normal, with mean $\mu_{C,t}$ and variance $\sigma_{C,t}^2$. Furthermore assume that the two normals are also jointly, conditionally normal.

\item Derive the asset pricing equation
\begin{equation*}
	1 = E_t [\exp(X_t)],
\end{equation*}
where 
\begin{equation*}
	X_t = -\rho + r_{t+1}^j + \sigma^j \epsilon_{t+1}^j - \frac{1}{2} (\sigma^j)^2 - \gamma \Delta \ln C_{t+1} 
\end{equation*}
so
\begin{equation*}
X_t \sim -\rho + \mathcal{N}\bigg(r_{t+1}^j - \frac{1}{2} (\sigma^j)^2 - \gamma \mu_{C,t}, (\sigma^j)^2 + \gamma^2 \sigma_{C,t}^2 - 2 \rho_{j,C} \gamma \sigma^j \sigma_{C,t} \bigg).
\end{equation*}

\item Taking expecations and logs show:
\begin{equation}
	0 =  - \rho + r_{t+1}^j - \frac{1}{2} (\sigma^j)^2 - \gamma E_t(\Delta \ln C_{t+1}) + \frac{1}{2} \text{Var}_t( \sigma^j \epsilon_{t+1}^j - \gamma \Delta \ln C_{t+1})
\end{equation}


\item Use this last formula to derive the risk-free rate $r^f$ (Hint: for $j=f$ set $\sigma^f=0$)


\item Consider a class of equities with risk $\sigma^E$, we define the equity premium as $$\pi_{t+1}^E \equiv r_{t+1}^E - r_{t+1}^f $$
Show $$\pi_{t+1}^E = \gamma \sigma_{C,E}$$ where $\sigma_{C,E}$ is the covariance between equity returns and log consumption growth. 

\end{enumerate}

\end{document}











