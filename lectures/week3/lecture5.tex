\documentclass[10pt]{beamer}
\usepackage{amsmath,amsfonts,microtype,nicefrac,amssymb, amsthm,centernot}
\usefonttheme{professionalfonts}
\usefonttheme{serif}

\usepackage{palatino}
\usepackage{mathpazo} % or: {newpxtext,newpxmath}
\renewcommand\familydefault{\rmdefault}

\usepackage[utf8]{inputenc}
\usepackage[T1]{fontenc}
\usepackage{textcomp}
\usepackage{bm}

\usepackage{tikz}
\usetikzlibrary{arrows.meta}
\usetikzlibrary{positioning}
\usetikzlibrary{arrows,shapes}
\usepackage{caption, subcaption}

%% BEAMER BUTTON
%\setbeamertemplate{button}{\tikz
%\node[
%	inner xsep = 2pt, 
%	draw = structure!0, 
%	fill = myblue, 
%	rounded corners = 4pt]{\color{white} \tiny\insertbuttontext};
%}


%% ALGORITHM 
\usepackage{algorithm}
\usepackage[noend]{algpseudocode}
\usepackage{multimedia}

%% THEME
\usetheme[frametitleformat=regular,titleformat=regular]{Madrid}
\setbeamerfont{frametitle}{shape=\normalfont}
\setbeamerfont{title}{shape=\normalfont}

%% PATHS
\graphicspath{{./results/}}
\makeatletter
\def\input@path{{../draft/tables/latexData/}}
\makeatother

%% FIGURE ENVIRONMENT 
\usepackage{booktabs,siunitx}

\usepackage{pgfplots} 
%\usepackage[outdir=./figures]{epstopdf}
\usepackage{epstopdf}
\usepackage{float}
\usepackage{graphicx}

\usepackage[absolute,overlay]{textpos}

%% COLORS
%% LINKS
\definecolor{myred}{RGB}{163,32,45}
\definecolor{navyblue}{rgb}{0.05,0.2,0.70}
\definecolor{myblue}{RGB}{0,51,150}
\definecolor{myorange}{RGB}{255,140,0}
\definecolor{myref}{RGB}{160,160,160}
\definecolor{shock}{RGB}{0, 125, 34}%{50, 168, 82}

%% TRANSPARENCY

\usepackage{transparent}

%% BEAMER TEMPLATE
\usetheme{Boadilla}

\makeatother
\setbeamertemplate{itemize items}{\large\raisebox{-0.25ex}{\textbullet}}
\setbeamertemplate{itemize subitem}{\footnotesize\raisebox{0.15ex}{--}}
\setbeamertemplate{itemize subsubitem}{\Tiny\raisebox{0.7ex}{$\blacktriangleright$}}
%\setbeamertemplate{itemize subsubitem}{\color{yellow}$\blacksquare$}

\setbeamertemplate{enumerate item}[default]
\setbeamertemplate{enumerate subitem}{\textbullet}
\setbeamertemplate{footline}{}
\makeatletter
\setbeamertemplate{navigation symbols}{}


%% FORMATTING AUTHORS
%\usepackage{authblk}
\usepackage{url}
\usepackage{multirow}
\usepackage{array}

%% FRAME ITEMIZE SPACING
\usepackage{xpatch}

\xpatchcmd{\itemize}
{\def\makelabel}
{\setlength{\itemsep}{2.5ex}\def\makelabel}
{}
{}

\xpatchcmd{\enumerate}
{\def\makelabel}
{\setlength{\itemsep}{10ex}\def\makelabel}
{}
{}

%% APPENDIX 
\usepackage{appendixnumberbeamer}


%% TITLE AND OPENING
\title{\large M2: Lecture 5}
\subtitle{Dynamic Optimization in Continuous Time}
\author{Andreas Schaab}
\date{}


\begin{document}
\tikzstyle{every picture}+=[remember picture]
%\everymath{\displaystyle}
\thispagestyle{empty}
\maketitle 
\newpage

\addtocounter{framenumber}{-1}




%%%%%%%%%%%%%%%%%%%%%%%%%%  SLIDE   %%%%%%%%%%%%%%%%%%%%%%%%%%%%%%%%
\begin{frame}{Outline of today's lecture}
\addtocounter{framenumber}{-1}

\begin{enumerate}
\item Neoclassical growth model in continuous time
\vspace{3mm}
\item Calculus of variations
\vspace{3mm}
\item Optimal control theory
\vspace{3mm}
\item Example: 
\end{enumerate}
\end{frame}



%%%%%%%%%%%%%%%%%%%%%%%%%%  SLIDE   %%%%%%%%%%%%%%%%%%%%%%%%%%%%%%%%
\begin{frame}{1. Neoclassical growth model in continuous time}
\begin{itemize}
	\item The lifetime value of the representative household is
	\begin{equation*}
		v(k_0) = \max_{\{ c_t \}_{t \geq 0} } \int_0^\infty e^{-\rho t} u(c_t) dt
	\end{equation*}
	subject to
	\begin{align*}
		\dot k_t &= F(k_t) - \delta k_t - c_t \\
		k_0 &\text{ given },
	\end{align*}
	where $\dot x_t = \frac{d}{dt} x_t$, $\rho$ is the discount rate, $c_t$ is the rate of consumption, $u(\cdot)$ is instantaneous utility flow, and $\dot k_t$ is the rate of (net) capital accumulation
	
	\item No uncertainty for now
	
	\item This is the \textbf{sequence problem} in continuous time
\end{itemize}
\end{frame}


%%%%%%%%%%%%%%%%%%%%%%%%%%  SLIDE   %%%%%%%%%%%%%%%%%%%%%%%%%%%%%%%%
\begin{frame}{2. Calculus of variations}
\begin{itemize}
\item Resources:
\begin{itemize}
	\item LeVeque: Finite Difference Methods for Ordinary and Partial Differential Equations
	
	\item Kamien and Schwartz: Dynamic Optimization
	
	\item Gelfand and Fomin: Calculus of Variations
\end{itemize}

\item This dynamic optimization problem is associated with the Lagrangian
\begin{equation*}
	L = \int_0^\infty e^{-\rho t} \bigg[ u(c_t) + \mu_t \bigg( F(k_t) - \delta k_t - c_t - \dot k_t \bigg) \bigg] dt 
\end{equation*}

\item $\mu_t$ is the Lagrange multiplier on the capital accumulation ODE

\item What do we do with $\dot k_t$??

\end{itemize}
\end{frame}


%%%%%%%%%%%%%%%%%%%%%%%%%%  SLIDE   %%%%%%%%%%%%%%%%%%%%%%%%%%%%%%%%
\begin{frame}{}
\begin{itemize}		
\item Integrate by parts:
\begin{align*}
	\int_0^\infty e^{-\rho t} \mu_t \dot k_t dt &= e^{-\rho t} \mu_t k_t \Big|_0^\infty - \int_0^\infty \frac{d}{dt} \bigg( e^{-\rho t} \mu_t \bigg) k_t dt \\
	&= - \mu_0 k_0 + \int_0^\infty e^{-\rho t} \rho \mu_t k_t dt - \int_0^\infty e^{-\rho t} \dot \mu_t k_t dt
\end{align*}

\item Plugging into Lagrangian: 
\begin{equation*}
	L = \int_0^\infty e^{-\rho t} \bigg[ u(c_t) + \mu_t \bigg( F(k_t) - \delta k_t - c_t \bigg) - \rho \mu_t k_t + \dot \mu_t k_t \bigg] dt + \mu_0 k_0
\end{equation*}

\item What have we accomplished? 

\item Notice $\mu_0 k_0$, this is crucial. What's intuition? 

\end{itemize}
\end{frame}


%%%%%%%%%%%%%%%%%%%%%%%%%%  SLIDE   %%%%%%%%%%%%%%%%%%%%%%%%%%%%%%%%
\begin{frame}{}
\begin{equation*}
	L = \int_0^\infty e^{-\rho t} \bigg[ u(c_t) + \mu_t \bigg( F(k_t) - \delta k_t - c_t \bigg) - \rho \mu_t k_t + \dot \mu_t k_t \bigg] dt + \mu_0 k_0
\end{equation*}

\begin{itemize}
\item The planner optimizes over paths $\{ c_t \}$ and $\{ k_t \}$

\item At an optimum, there cannot be \textit{any} small perturbation in these paths that the planner finds preferable

\item Let $\{ c_t \}$ and $\{ k_t \}$ be \textit{candidate} optimal paths. Consider $\hat c_t = c_t + \alpha h_t^c$ and $\hat k_t = k_t + \alpha h_t^k$ for arbitrary functions $h_t^c$ and $h_t^k$
\end{itemize}


\vspace{-2mm}
\begin{align*}
	L(\alpha) = \int_0^\infty e^{-\rho t} \bigg[ &u(c_t + \alpha h_t^c) + \mu_t \bigg( F(k_t + \alpha h_t^k) - \delta k_t - \delta \alpha h_t^k - c_t - \alpha h_t^c \bigg) \\
	&- \rho \mu_t (k_t + \alpha h_t^k) + \dot \mu_t (k_t + \alpha h_t^k) \bigg] dt + \mu_0 (k_0 + \alpha h_0^k)
\end{align*}

\vspace{-2mm}
\begin{itemize}
\item What about \textit{boundary conditions}? At $t=0$, capital stock is fixed ($k_0$ given) while consumption is free. So must have: $h_0^k = 0$ while $h_0^c$ is free
\end{itemize}
\end{frame}


%%%%%%%%%%%%%%%%%%%%%%%%%%  SLIDE   %%%%%%%%%%%%%%%%%%%%%%%%%%%%%%%%
\begin{frame}{}

Necessary condition for optimality: $\frac{d}{d \alpha} L(0) = 0$
\begin{align*}
	L(\alpha) = \int_0^\infty e^{-\rho t} \bigg[ &u(c_t + \alpha h_t^c) + \mu_t \bigg( F(k_t + \alpha h_t^k) - \delta k_t - \delta \alpha h_t^k - c_t - \alpha h_t^c \bigg) \\
	&- \rho \mu_t (k_t + \alpha h_t^k) + \dot \mu_t (k_t + \alpha h_t^k) \bigg] dt + \mu_0 (k_0 + \alpha h_0^k)
\end{align*}

Work this out yourselves (many times, in many applications!)
\begin{align*}
	\frac{d}{d \alpha}L(0) = \int_0^\infty e^{-\rho t} \bigg[ &u'(c_t) h_t^c + \mu_t \bigg( F'(k_t) h_t^k  - \delta h_t^k - h_t^c \bigg) \\
	&- \rho \mu_t h_t^k + \dot \mu_t h_t^k \bigg] dt + \mu_0 h_0^k
\end{align*}
where $h_0^k = 0$ because $k_0$ is fixed


\end{frame}



%%%%%%%%%%%%%%%%%%%%%%%%%%  SLIDE   %%%%%%%%%%%%%%%%%%%%%%%%%%%%%%%%
\begin{frame}{}
	
Group terms: 
\begin{align*}
	\frac{d}{d \alpha}L(0) = \int_0^\infty e^{-\rho t} \bigg[ &\bigg( u'(c_t) - \mu_t \bigg) h_t^c + \bigg( \mu_t \Big( F'(k_t) - \delta \Big) - \rho \mu_t + \dot \mu_t \bigg) h_t^k \bigg] dt 
\end{align*}

\vspace{5mm}
\textbf{Fundamental Theorem of the Calculus of Variations}: Since $h_t^c$ and $h_t^k$ were arbitrary, we must have \textit{pointwise}
\begin{align*}
	0 &= u'(c_t) - \mu_t \\
	0 &= \mu_t \Big( F'(k_t) - \delta \Big) - \rho \mu_t + \dot \mu_t
\end{align*}

\vspace{5mm}
\textbf{Proposition.} (Euler equation for marginal utility) 
\begin{equation*}
	\frac{\dot \mu_t}{\mu_t} = \frac{\dot u_{c, t}}{u_{c, t}} = \rho -  F'(k_t) + \delta = \rho - r_t 
\end{equation*}
	
\end{frame}


%%%%%%%%%%%%%%%%%%%%%%%%%%  SLIDE   %%%%%%%%%%%%%%%%%%%%%%%%%%%%%%%%
\begin{frame}{}

\begin{itemize}
\item We have now solved the neoclassical growth model in continuous time. Its solution is given by a system of two ODEs. 

\item Suppose $u(c) = \log(c)$ and $F(k) = k^\alpha$, then:
\begin{align*}
	\frac{\dot c_t}{c_t} &= \alpha k_t^{\alpha - 1} - \delta - \rho \\
	\dot k_t &= k_t^\alpha - \delta k_t - c_t
\end{align*}
with $k_0$ given

\item Derive the consumption Euler equation yourselves!

\item What are the boundary conditions? (Always ask about BCs!) 
\begin{itemize}
	\vspace{1mm}
	\item Initial condition on capital: $k_0$ given
	\vspace{-3mm}
	\item Terminal condition on consumption : $\lim_{T \to \infty} c_T = c_\text{ss}$ 
\end{itemize}

\end{itemize}
	
\end{frame}



%%%%%%%%%%%%%%%%%%%%%%%%%%  SLIDE   %%%%%%%%%%%%%%%%%%%%%%%%%%%%%%%%
\begin{frame}{3. Optimal control theory}
\begin{itemize}
\item Optimal control theory emerged from the calculus of variations

\item Applies to dynamic optimization problems in continuous time that feature (ordinary) differential equations as constraints

\item Again the neoclassical growth model:
\begin{equation*}
	v(k_0) = \max_{\{ c_t \}_{t \geq 0} } \int_0^\infty e^{-\rho t} u(c_t) dt
\end{equation*}
subject to
\begin{align*}
	\dot k_t &= F(k_t) - \delta k_t - c_t,  \quad k_0 \text{ given}
\end{align*}

\item Three new terms: 
\begin{itemize}
\vspace{1mm}
\item \textbf{State variable}: $k_t$

\vspace{-2mm}
\item \textbf{Control variable}: $c_t$

\vspace{-2mm}
\item \textbf{Hamiltonian}: $H(c_t, k_t, \mu_t) = u(c_t) + \mu_t \big[ F(k_t) - \delta k_t - c_t \big]$
\end{itemize}
\end{itemize}	
\end{frame}



%%%%%%%%%%%%%%%%%%%%%%%%%%  SLIDE   %%%%%%%%%%%%%%%%%%%%%%%%%%%%%%%%
\begin{frame}{}
\begin{itemize}
\item With Hamiltonian in hand, \textit{copy-paste} formula that we can always use:

\vspace{1mm}
\textbf{Optimality condition}: $\frac{\partial}{\partial c} H = 0$

\vspace{2mm}
\textbf{Multiplier condition}: $\rho \mu_t - \dot \mu_t = \frac{\partial}{\partial k} H$

\vspace{2mm}
\textbf{State condition}: $\dot k_t = \frac{\partial}{\partial \mu} H$

\item This gives us the same equations that we derived using calc of variations:
\begin{align*}
	u'(c_t) &= \mu_t \\
	\rho \mu_t - \dot \mu_t &= \mu_t (F'(k_t) - \delta) \\
	\dot k_t &= F(k_t) - \delta k_t - c_t
\end{align*}

\item We again get system of Euler equation and capital accumulation:
\begin{align*}
	\dot c_t &= \frac{u'(c_t)}{u''(c_t)} \Big( \rho - F'(k_t) + \delta \Big) \\
	\dot k_t &= F(k_t) - \delta k_t - c_t
\end{align*}

\end{itemize}
\end{frame}




%%%%%%%%%%%%%%%%%%%%%%%%%%  SLIDE   %%%%%%%%%%%%%%%%%%%%%%%%%%%%%%%%
\begin{frame}{4. Simple example}
\begin{itemize}
	
\item Credit: Kamien-Schwartz p. 129

\item Simple problem: not much intuition, but illustrates mechanics
\begin{equation*}
	\max \int_0^1 (x + u) dt 
\end{equation*}
subject to $\dot x = 1 - u^2$ and initial condition $x_0 = 1$

\item Step 1: form Hamiltonian $H(t, x, u, \lambda) = x + u + \lambda (1 - u^2)$

\item Step 2: necessary conditions (note: no discounting here)
\begin{align*}
	0 &= H_u = 1 - 2 \lambda u \\
	- \dot \lambda &= H_x = 1
\end{align*}
and terminal condition $\lambda_1 = 0$ (because $u_1$ is \textit{free})

\end{itemize}
\end{frame}



%%%%%%%%%%%%%%%%%%%%%%%%%%  SLIDE   %%%%%%%%%%%%%%%%%%%%%%%%%%%%%%%%
\begin{frame}{}
\begin{itemize}

\item Step 3: manipulate necessary conditions:
\begin{align*}
	\lambda &= 1 - t \\
	u &= \frac{1}{2 \lambda}
\end{align*}
and therefore: $u = \frac{1}{2} (1 - t)$

\item Finally: solve for all paths (control, state, multiplier)
\begin{align*}
	x_t &= t - \frac{1}{4} (1 - t) + \frac{5}{4} \\
	\lambda_t &= 1 - t \\
	u_t &= \frac{1}{2} (1 - t)
\end{align*}

\end{itemize}
\end{frame}






%%%%%%%%%%%%%%%%%%%%%%%%%%%%%%%%%%%%%%%%%%%%%%%%%%%%%%%%%%%%%%
%%%%%%%%%%%%%%%%%%%%%%%%%%%%%%%%%%%%%%%%%%%%%%%%%%%%%%%%%%%%%%
%%%%%%%%%%%%%%%%%%%%%%%%%%%%%%%%%%%%%%%%%%%%%%%%%%%%%%%%%%%%%%

\appendix


\end{document}









