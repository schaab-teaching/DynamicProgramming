\documentclass[11pt]{extarticle}
\usepackage{fullpage,amsmath,amsfonts,microtype,nicefrac,amssymb, amsthm}
\usepackage[left=1in, bottom=1in, top=1in, right = 1in]{geometry}
\usepackage{textcomp}
\usepackage{mathpazo}
\usepackage{mathrsfs}
\usepackage[T1]{fontenc}
\usepackage[utf8]{inputenc}
\usepackage[english]{babel}
\usepackage{graphicx}

\usepackage{microtype}

\usepackage{bm}
\usepackage{dsfont}
\usepackage{enumerate}
\usepackage{ragged2e}

\setlength{\parindent}{24pt}
\setlength{\jot}{8pt}


\usepackage[shortlabels]{enumitem}


%% FOOTNOTES
\usepackage[bottom]{footmisc}
\usepackage{footnotebackref}


%% FIGURE ENVIRONMENT
%\graphicspath{{}}
\usepackage[margin=15pt, font=small, labelfont={bf}, labelsep=period]{caption}
\usepackage{subcaption}
\captionsetup[figure]{name={Figure}, position=above}
\usepackage{float}
\usepackage{epstopdf}


%% NEW COMMANDS
\renewcommand{\baselinestretch}{1.25} 
\renewcommand{\qedsymbol}{$\blacksquare$}
\newcommand{\R}{\mathbb{R}}
\newcommand{\indep}{\mathrel{\text{\scalebox{1.07}{$\perp\mkern-10mu\perp$}}}}
\renewcommand{\b}{\begin}
\newcommand{\e}{\end}

%% NEWTHEOREM
\theoremstyle{plain}
\newtheorem{thm}{Theorem}
\newtheorem{lem}[thm]{Lemma}
\newtheorem{prop}[thm]{Proposition}

\theoremstyle{definition}
\newtheorem{defn}[thm]{Definition}
\newtheorem{ex}[thm]{Example}
\newtheorem{remark}[thm]{Remark}
\newtheorem{cor}[thm]{Corollary}

%% LINKS and COLORS
\usepackage[dvipsnames]{xcolor}
\usepackage{hyperref}
\definecolor{myred}{RGB}{163, 32, 45}
\hypersetup{
	%backref=true,
	%pagebackref=true,
	colorlinks=true,
	urlcolor=myred,
	citecolor=myred, 
	linktoc=all,     
	linkcolor=myred,
}

%% TABLE OF CONTENTS
\addto\captionsenglish{
	\renewcommand{\contentsname}
	{}% This removes the heading over the table of contents.
}



%%%%%%%%%%%%%%%%%%%%%%%%%%%%%%%%%%%%%%%%%%%%%%%%%%%%%%%%%%%%
%%%%%%%%%%%%%%%%%%%%%%%%%%%%%%%%%%%%%%%%%%%%%%%%%%%%%%%%%%%%
%%%%%%%%%%%%%%%%%%%%%%            END PREAMBLE           %%%%%%%%%%%%%%%%%%%%%%%%
%%%%%%%%%%%%%%%%%%%%%%%%%%%%%%%%%%%%%%%%%%%%%%%%%%%%%%%%%%%%
%%%%%%%%%%%%%%%%%%%%%%%%%%%%%%%%%%%%%%%%%%%%%%%%%%%%%%%%%%%%

\title{Dynamic Optimization: Problem Set \#2}

\author{Andreas Schaab}

\date{Fall, 2022}



\begin{document}

\maketitle
\thispagestyle{empty}
\setcounter{page}{0}


%%%%%%%%%%%%%%%%%%%%%%%%%%%%%%%%%%%%%%%%%%%%%%%%%%%%%%%%%%%%
%%%%%%%%%%%%%%%%%%%%%%%%%%%%%%%%%%%%%%%%%%%%%%%%%%%%%%%%%%%%
\vspace{10mm}
\section*{Problem 1: proof of contraction mapping theorem **}

\textbf{Credit:} David Laibson


\vspace{5mm}
\noindent
In class, we defined the Bellman operator $B$, which operates on functions $w$, and is defined by
\begin{equation*}
	(Bw)(x) \equiv \max_{x' \in \Gamma(x)} \bigg\{ F(x, x') + \beta w(x') \bigg\}
\end{equation*}
for all $x \in \mathcal X$ in the state space, where $\Gamma(x)$ is some constraint set---in our case, this was the budget constraint. The definition is expressed pointwise, but it applies to all possible values in the state space. We call $B$ an operator because it maps a function $w$ to a new function $Bw$. So both $w$ and $Bw$ map $\mathcal X$ into $\mathbb R$. Operator $B$ maps \textit{functions} and is therefore called a functional operator. In class, we showed that the solution of the Bellman equation---the value function---is a fixed point of the Bellman operator.

What does it mean to \textit{iterate} $B^n w$?
\begin{align*}
	(Bw) (x) &= \max_{x' \in \Gamma(x)} \bigg\{ F(x, x') + \beta w(x') \bigg\} \\
	(B(Bw)) (x) &= \max_{x' \in \Gamma(x)} \bigg\{ F(x, x') + \beta (Bw)(x') \bigg\} \\
	(B(B^2w)) (x) &= \max_{x' \in \Gamma(x)} \bigg\{ F(x, x') + \beta (B^2w)(x') \bigg\} \\
	\vdots & \vdots  \\
	(B(B^nw)) (x) &= \max_{x' \in \Gamma(x)} \bigg\{ F(x, x') + \beta (B^nw)(x') \bigg\}.
\end{align*}
What does it mean for functions to converge to a limiting function? Let $v_0$ be some guess for the value function, then convergence would mean
\begin{equation*}
	\lim_{n \to \infty} B^n v_0 = v.
\end{equation*}
And why might $B^n w$ converge as $n \to \infty$? The answer is that $B$ is a \textit{contraction mapping}.


\vspace{5mm}
\begin{defn}
	
	Let $(S, d)$ be a metric space and $B: S \to S$ be a function that maps $S$ into intself. $B$ is a contraction mapping if for some $\beta \in (0, 1)$, $d(Bf, Bg) \leq \beta d(f, g)$, for any two functions $f$ and $g$.\footnote{
		A metric $d$ is a way of representing the distance between two functions, or two members of (metric) space $S$. One example: the supremum pointwise gap.
	}
	
\end{defn}


\vspace{5mm}
\noindent
Intuitively, $B$ is a contraction mapping if applying the operator $B$ to any two functions $f$ and $g$ (that are not the same) moves them strictly closer together. $Bf$ and $Bg$ are strictly closer together than $f$ and $g$. We can now state the contraction mapping theorem. 


\vspace{5mm}
\begin{thm}
	
	If $(S, d)$ is a complete metric space and $B: S \to S$ is a contraction mapping, then: 
	\begin{enumerate}[(i)]
		\item $B$ has exactly one fixed point $v \in S$
		\item For any $v_0 \in S$, $\lim_{n \to \infty} B^n v_0 = v$
		\item $B^n v_0$ has an exponential convergence rate at least as great as $- \ln(\beta)$
	\end{enumerate}
	
\end{thm}


\vspace{10mm}
\noindent
In this problem, we will illustrate and prove the contraction mapping theorem.

\begin{enumerate}[(a)]
\item Consider the contraction mapping $(Bw)(x) \equiv h(x) + \alpha w(x)$ with $\alpha \in (0, 1)$. Iteratively apply the operator $B$ and show that 
\begin{equation*}
	\lim_{n \to \infty} (B^n f)(x) = \frac{h(x)}{1-\alpha}
\end{equation*}
Argue that this shows that the fixed point of this operator $B$ is consequently the function $v(x) = \frac{1}{1-\alpha} h(x)$. Show that $(Bv)(x) = v(x)$. 


\item Now we will prove the contraction mapping theorem in 3 steps (we will not prove the convergence rate). Show that $\{ B^n f_0\}_{n=0}^\infty$ is a Cauchy sequence. (Cauchy sequence definition: Fix any $\epsilon > 0$. Then there exists $N$ such that $d(B^m f_0, B^n f_0) \leq \epsilon$ for all $m, n \leq N$.) 

\textbf{Solution.} Choose some $f_{0}\in S.$ \ Let $f_{n}=B^{n}f_{0}.$ Since $B$ is a contraction 
\begin{equation*}
	d(f_{2},f_{1})=d(Bf_{1},Bf_{0})\leq \delta d(f_{1},f_{0}).
\end{equation*}
Continuing by induction, 
\begin{equation*}
	d(f_{n+1},f_{n})\leq \delta ^{n}d(f_{1},f_{0})\text{ \ \ }\forall n 
\end{equation*}
We can now bound the distance between $f_{n}$ and $f_{m}$ when $m>n.$%
\begin{eqnarray*}
	d(f_{m},f_{n}) &\leq
	&d(f_{m},f_{m-1})+...+d(f_{n+2},f_{n+1})+d(f_{n+1},f_{n}) \\
	&\leq &\left[ \delta ^{m-1}+...+\delta ^{n+1}+\delta ^{n}\right]
	d(f_{1},f_{0}) \\
	&=&\delta ^{n}\left[ \delta ^{m-n-1}+...+\delta ^{1}+1\right] d(f_{1},f_{0})
	\\
	&<&\frac{\delta ^{n}}{1-\delta }d(f_{1},f_{0})
\end{eqnarray*}%
So $\{f_{n}\}_{n=0}^{\infty }$ is Cauchy. \ \ $\checkmark $

\item Show that the limit point $v$ is a fixed point of $B$. 

\textbf{Solution.} Since $S$\ is complete $f_{n}\rightarrow v\in S.$

We now have a candidate fixed point $v\in S.$

To show that $Bv=v,$ note 
\[
\begin{array}{lll}
	d(Bv,v) & \leq & d(Bv,B^{n}f_{0})+d(B^{n}f_{0},v) \\ 
	& \leq & \delta d(v,B^{n-1}f_{0})+d(B^{n}f_{0},v)%
\end{array}%
. 
\]%
These inequalities must hold for all $n.$

And both terms on the RHS converge to zero as $n\rightarrow \infty .$

So $d(Bv,v)=0,$ implying that $Bv=v.$ $\ \ \checkmark $



\item Show that only one fixed point exists. 

\textbf{Solution.} Now we show that our fixed point is unique.

Suppose there were two fixed points: $v\neq v^{\ast }.$

Then $Bv=v$ and $Bv^{\ast }=v^{\ast }$ (since fixed points).

Also have $d(Bv,Bv^{\ast })<d(v,v^{\ast })$ (since $B$ is contraction)

So, $d(v,v^{\ast })=d(Bv,Bv^{\ast })<d(v,v^{\ast }).$

Contradiction.

So the fixed point is unique. \ \ $\checkmark $


\end{enumerate}




%%%%%%%%%%%%%%%%%%%%%%%%%%%%%%%%%%%%%%%%%%%%%%%%%%%%%%%%%%%%
%%%%%%%%%%%%%%%%%%%%%%%%%%%%%%%%%%%%%%%%%%%%%%%%%%%%%%%%%%%%
\vspace{10mm}
\section*{Problem 2: Blackwell's sufficiency conditions  **}

\textbf{Credit:} David Laibson


\vspace{5mm}
\noindent
We now show that there are in fact sufficient conditions for an operator to be contraction mapping.

\begin{thm}
	
	(Blackwell's sufficient conditions) Let $X \subset \mathbb R^l$ and let $C(X)$ be a space of bounded functions $f:X \to \mathbb R$, with the sup-metric. Let $B : C(X) \to C(X)$ be an operator satisfying two conditions:
	\begin{enumerate}[1.]
		\item monotonicity: if $f,g\in C(X)$ and $f(x)\leq g(x)$ $\forall x\in X$,%
		\newline
		then $(Bf)(x)\leq (Bg)(x),$ $\forall x\in X$\newline
		
		\item discounting: there exists some $\delta \in (0,1)$ such that 
		\[
		\lbrack B(f+a)](x)\leq (Bf)(x)+\delta a\text{ \ }\forall \text{ }f\in C(X),\
		a\geq 0,\ x\in X. 
		\]%
	\end{enumerate}
	
	\noindent
	Then, $B$ is a contraction with modulus $\delta$.
\end{thm}

\noindent
Note that $a$ is a constant and $(f+a)$ is the function generated by adding $a$ to the function $f$. Blackwell's conditions are sufficient but not necessary for $B$ to be a contraction.

\vspace{10mm}
\noindent
In this problem, we will prove these sufficient conditions.
\begin{enumerate}[(a)]
\item Let $d$ be the sup-metric and show that, for any $f, g \in C(X)$, we have $f(x) \leq g(x) + d(f, g)$ for all $x$

\textbf{Solution.} Follows directly from definition of sup-metric.

\item Use monotonicity and discounting to show that, for any $f, g \in C(X)$, we have $(Bf)(x) \leq (Bg)(x) + \delta d(f, g)$ and $(Bg)(x) \leq (Bf)(x) + \delta d(f, g)$. 

\textbf{Solution.} Using monotonicity and discounting, we have for all $x$
\[
\left( Bf\right) (x)\leq \lbrack B(g+d(f,g))](x)\leq \left( Bg\right)
(x)+\delta d(f,g) 
\]
\[
\left( Bg\right) (x)\leq \lbrack B(f+d(f,g))](x)\leq \left( Bf\right) \left(
x\right) +\delta d(f,g) 
\]

\item Combine these to show that $d(Bf, Bg) \leq \delta d(f, g)$. 

\textbf{Solution.}
\[
\left( Bf\right) \left( x\right) -\left( Bg\right) \left( x\right) \leq
\delta d(f,g) 
\]%
\[
\left( Bg\right) \left( x\right) -\left( Bf\right) \left( x\right) \leq
\delta d(f,g) 
\]%
\[
\left\vert (Bf)(x)-(Bg)(x)\right\vert \leq \delta d(f,g) 
\]%
\[
\sup_{x}\left\vert (Bf)(x)-(Bg)(x)\right\vert \leq \delta d(f,g) 
\]%
\[
d(Bf,Bg)\leq \delta d(f,g)\text{ \ \ \ }\checkmark 
\]%
\end{enumerate}



%%%%%%%%%%%%%%%%%%%%%%%%%%%%%%%%%%%%%%%%%%%%%%%%%%%%%%%%%%%%
%%%%%%%%%%%%%%%%%%%%%%%%%%%%%%%%%%%%%%%%%%%%%%%%%%%%%%%%%%%%
\vspace{10mm}
\section*{Problem 3: example of Blackwell's conditions * (Canonical)}

\textbf{Credit:} David Laibson

\vspace{5mm}
\noindent
We will now work out a simple example to illustrate these sufficient conditions. In particular, consider the Bellman operator in a consumption problem (with stochastic asset returns, stochastic labor income, and a liquidity constraint). 
\begin{equation*}
	(Bf)(x)=\sup_{c\in \lbrack 0,x]}\left\{ u(c)+\delta Ef(\tilde{R}_{+1}(x-c)+%
	\tilde{y}_{+1})\right\} \text{ \ }\forall x\text{\ } 
\end{equation*}

\vspace{10mm}
\noindent
\begin{enumerate}
\item Check the first of Blackwell's conditions: monotonicity

\textbf{Solution.} Assume $f(x)\leq g(x)$ $\forall x$. Suppose $%
c_{f}^{\ast }$ is the optimal policy when the continuation value function is 
$f.$%
\begin{eqnarray*}
	(Bf)(x) &=&\sup_{c\in \lbrack 0,x]}\left\{ u(c)+\delta Ef(\tilde{R}%
	_{+1}(x-c)+\tilde{y}_{+1})\right\} \\
	&=&u(c_{f}^{\ast })+\delta Ef(\tilde{R}_{+1}(x-c_{f}^{\ast })+\tilde{y}_{+1})
	\\
	&\leq &u(c_{f}^{\ast })+\delta Eg(\tilde{R}_{+1}(x-c_{f}^{\ast })+\tilde{y}%
	_{+1}) \\
	&\leq &\sup_{c\in \lbrack 0,x]}\left\{ u(c)+\delta Eg(\tilde{R}_{+1}(x-c)+%
	\tilde{y}_{+1})\right\} \\
	&=&(Bg)(x)
\end{eqnarray*}

\item Check the second of Blackwell's conditions: discounting

\textbf{Solution.} Adding a constant ($\Delta $) to an optimization problem does not affect optimal choice, so
\begin{eqnarray*}
	\lbrack B(f+\Delta )](x) &=&\sup_{c\in \lbrack 0,x]}\left\{ u(c)+\delta E 
	\left[ f(\tilde{R}_{+1}(x-c)+\tilde{y}_{+1})+\Delta \right] \right\} \\
	&=&\sup_{c\in \lbrack 0,x]}\left\{ u(c)+\delta Ef(\tilde{R}_{+1}(x-c)+\tilde{%
		y}_{+1})\right\} +\delta \Delta \\
	&=&(Bf)(x)+\delta \Delta
\end{eqnarray*}

\end{enumerate}





%%%%%%%%%%%%%%%%%%%%%%%%%%%%%%%%%%%%%%%%%%%%%%%%%%%%%%%%%%%%
%%%%%%%%%%%%%%%%%%%%%%%%%%%%%%%%%%%%%%%%%%%%%%%%%%%%%%%%%%%%
\vspace{10mm}
\section*{Problems 4 -- 8}

For all other solutions, you can check here (\textbf{credit} David Laibson):
\url{https://projects.iq.harvard.edu/econ2010c/problem-sets-david-laibson}





\end{document}











