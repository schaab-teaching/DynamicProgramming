\documentclass[10pt]{beamer}
\usepackage{amsmath,amsfonts,microtype,nicefrac,amssymb, amsthm,centernot}
\usefonttheme{professionalfonts}
\usefonttheme{serif}

\usepackage{palatino}
\usepackage{mathpazo} % or: {newpxtext,newpxmath}
\renewcommand\familydefault{\rmdefault}

\usepackage[utf8]{inputenc}
\usepackage[T1]{fontenc}
\usepackage{textcomp}
\usepackage{bm}

\usepackage{tikz}
\usetikzlibrary{arrows.meta}
\usetikzlibrary{positioning}
\usetikzlibrary{arrows,shapes}
\usepackage{caption, subcaption}

%% BEAMER BUTTON
%\setbeamertemplate{button}{\tikz
%\node[
%	inner xsep = 2pt, 
%	draw = structure!0, 
%	fill = myblue, 
%	rounded corners = 4pt]{\color{white} \tiny\insertbuttontext};
%}


%% ALGORITHM 
\usepackage{algorithm}
\usepackage[noend]{algpseudocode}
\usepackage{multimedia}

%% THEME
\usetheme[frametitleformat=regular,titleformat=regular]{Madrid}
\setbeamerfont{frametitle}{shape=\normalfont}
\setbeamerfont{title}{shape=\normalfont}

%% PATHS
\graphicspath{{./results/}}
\makeatletter
\def\input@path{{../draft/tables/latexData/}}
\makeatother

%% FIGURE ENVIRONMENT 
\usepackage{booktabs,siunitx}

\usepackage{pgfplots} 
%\usepackage[outdir=./figures]{epstopdf}
\usepackage{epstopdf}
\usepackage{float}
\usepackage{graphicx}

\usepackage[absolute,overlay]{textpos}

%% COLORS
%% LINKS
\definecolor{myred}{RGB}{163,32,45}
\definecolor{navyblue}{rgb}{0.05,0.2,0.70}
\definecolor{myblue}{RGB}{0,51,150}
\definecolor{myorange}{RGB}{255,140,0}
\definecolor{myref}{RGB}{160,160,160}
\definecolor{shock}{RGB}{0, 125, 34}%{50, 168, 82}

%% TRANSPARENCY

\usepackage{transparent}

%% BEAMER TEMPLATE
\usetheme{Boadilla}

\makeatother
\setbeamertemplate{itemize items}{\large\raisebox{-0.25ex}{\textbullet}}
\setbeamertemplate{itemize subitem}{\footnotesize\raisebox{0.15ex}{--}}
\setbeamertemplate{itemize subsubitem}{\Tiny\raisebox{0.7ex}{$\blacktriangleright$}}
%\setbeamertemplate{itemize subsubitem}{\color{yellow}$\blacksquare$}

\setbeamertemplate{enumerate item}[default]
\setbeamertemplate{enumerate subitem}{\textbullet}
\setbeamertemplate{footline}{}
\makeatletter
\setbeamertemplate{navigation symbols}{}


% TODO:
% Introduce page number but no other footer
% Define appropriate itemize and enumerate environments so I never have to touch these
% What about an automatic subitemize environment?
% Write an external template that gets loaded on each for each lecture .tex


%% FORMATTING AUTHORS
%\usepackage{authblk}
\usepackage{url}
\usepackage{multirow}
\usepackage{array}

%% FRAME ITEMIZE SPACING
\usepackage{xpatch}

\xpatchcmd{\itemize}
{\def\makelabel}
{\setlength{\itemsep}{2.5ex}\def\makelabel}
{}
{}

\xpatchcmd{\enumerate}
{\def\makelabel}
{\setlength{\itemsep}{10ex}\def\makelabel}
{}
{}

%% APPENDIX 
\usepackage{appendixnumberbeamer}


%% TITLE AND OPENING
\title{\large M2: Lecture 8}
\subtitle{Theory of the Consumption Function (I)}
\author{Andreas Schaab}
\date{}


\begin{document}
\tikzstyle{every picture}+=[remember picture]
%\everymath{\displaystyle}
\thispagestyle{empty}
\maketitle 
\newpage

\addtocounter{framenumber}{-1}




%%%%%%%%%%%%%%%%%%%%%%%%%%  SLIDE   %%%%%%%%%%%%%%%%%%%%%%%%%%%%%%%%
\begin{frame}{Outline of today's lecture}
\addtocounter{framenumber}{-1}

\begin{enumerate}
\item Canonical income fluctuations problem in discrete time

\vspace{2mm}
\item Permanent income hypothesis

\vspace{2mm}
\item Certainty equivalence and consumption as a martingale

\vspace{2mm}
\item Linearization of Euler equation

\vspace{2mm}
\item Euler equation empirics

\end{enumerate}
\end{frame}



%%%%%%%%%%%%%%%%%%%%%%%%%%  SLIDE   %%%%%%%%%%%%%%%%%%%%%%%%%%%%%%%%
\begin{frame}{1. Canonical model of consumption}
\begin{itemize}
\item The standard model is known as the \textbf{consumption-savings} or \textbf{income-fluctuations} problem
\begin{equation*}
	V(a_0) = \max_{ \{c_t\}_{t=0}^\infty } \, \mathbb{E}_0 \sum_{t=0}^\infty \beta^t u(c_t)
\end{equation*}
subject to
\begin{equation*}
	a_{t+1} = R_{t+1} (a_t - c_t) + y_t
\end{equation*}

\item $a_t$ is wealth, $\{ R_{t+1} \}$ is deterministic (gross) interest rate process, and $y_t$ is iid. income risk

\item Assume $u(\cdot)$ concave ($u' > 0$ and $u''< 0$ for all $c$) and $\lim_{c \to 0} u'(c) = \infty$

\item What should we assume about borrowing capacity?
\begin{itemize}
\vspace{1mm}
\item Natural borrowing limit: $a_t \geq a^n$

\vspace{-2mm}
\item Ad-hoc borrowing limit: $a_t \geq \underline a$
\end{itemize}

\end{itemize}
\end{frame}


%%%%%%%%%%%%%%%%%%%%%%%%%%  SLIDE   %%%%%%%%%%%%%%%%%%%%%%%%%%%%%%%%
\begin{frame}{}
\begin{itemize}
\item Bellman equation:
\begin{align*}
	V_t(a) &= \max_{c, a'} \Big\{ u(c) + \beta \mathbb{E}_t V_{t+1}(a') \Big\} \;\; s.t. \;\; a' = R_{t+1} (a - c) + y
\end{align*}

\item Bellman equation may not be stationary or time-independent: $R_t$

\item If income process $\{y_t\}$ persistent, would need $y$ as second state variable

\item First-order conditions (with ad-hoc borrowing limit $\underline a$):
\begin{align*}
	u'(c_t(a)) = &\; \beta R_{t+1} \mathbb{E}_t \frac{\partial V_{t+1}(a')}{\partial a'} \quad \text{ if }  a' > \underline a \\
	u'(c_t(a)) \geq &\; \beta R_{t+1} \mathbb{E}_t \frac{\partial V_{t+1}(a')}{\partial a'} \quad \text{ if }  a' = \underline a
\end{align*}

\end{itemize}
\end{frame}


%%%%%%%%%%%%%%%%%%%%%%%%%%  SLIDE   %%%%%%%%%%%%%%%%%%%%%%%%%%%%%%%%
\begin{frame}{}
\begin{itemize}
\item Envelope theorem: $\frac{\partial V_t(a)}{\partial a} = u'(c_t(a))$

\item So we again get a consumption Euler equation:
\begin{align*}
	u'(c_t) = &\; \beta R_{t+1} \mathbb{E}_t u'(c_{t+1}) \quad \text{ if }  a' > \underline a \\
	u'(c_t) \geq &\; \beta R_{t+1} \mathbb{E}_t u'(c_{t+1})  \quad \text{ if }  a' = \underline a
\end{align*}
\end{itemize}


\vspace{6mm}
Perturbation intuition:
\begin{itemize}
\item What is the cost of consuming $\epsilon$ dollars less today? 
\begin{equation*}
	\text{Utility loss today} = \epsilon \cdot u'(c_t)
\end{equation*}

\item What is the expected, discounted gain of consuming $\epsilon \cdot R_{t+1}$ dollars more tomorrow?
\begin{equation*}
	\text{Utility gain tomorrow} = \beta (\epsilon \cdot R_{t+1}) \mathbb{E}_t u'(c_{t+1})
\end{equation*}
\end{itemize}

\end{frame}


\iffalse

%%%%%%%%%%%%%%%%%%%%%%%%%%  SLIDE   %%%%%%%%%%%%%%%%%%%%%%%%%%%%%%%%
\begin{frame}{2. Linearization of Euler equation}
\begin{itemize}
\item 

\end{itemize}
\end{frame}



%%%%%%%%%%%%%%%%%%%%%%%%%%  SLIDE   %%%%%%%%%%%%%%%%%%%%%%%%%%%%%%%%
\begin{frame}{3. Permanent income hypothesis}
\begin{itemize}
\item 

\end{itemize}
\end{frame}



%%%%%%%%%%%%%%%%%%%%%%%%%%  SLIDE   %%%%%%%%%%%%%%%%%%%%%%%%%%%%%%%%
\begin{frame}{4. Certainty equivalence and consumption as a martingale}
\begin{itemize}
\item 

\end{itemize}
\end{frame}



%%%%%%%%%%%%%%%%%%%%%%%%%%  SLIDE   %%%%%%%%%%%%%%%%%%%%%%%%%%%%%%%%
\begin{frame}{5. Eat-the-pie problem}
\begin{itemize}
\item 

\end{itemize}
\end{frame}


\fi








%%%%%%%%%%%%%%%%%%%%%%%%%%%%%%%%%%%%%%%%%%%%%%%%%%%%%%%%%%%%%%
%%%%%%%%%%%%%%%%%%%%%%%%%%%%%%%%%%%%%%%%%%%%%%%%%%%%%%%%%%%%%%
%%%%%%%%%%%%%%%%%%%%%%%%%%%%%%%%%%%%%%%%%%%%%%%%%%%%%%%%%%%%%%

\appendix


\end{document}









