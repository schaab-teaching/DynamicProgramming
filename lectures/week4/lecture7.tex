\documentclass[10pt]{beamer}
\usepackage{amsmath,amsfonts,microtype,nicefrac,amssymb, amsthm,centernot}
\usefonttheme{professionalfonts}
\usefonttheme{serif}

\usepackage{palatino}
\usepackage{mathpazo} % or: {newpxtext,newpxmath}
\renewcommand\familydefault{\rmdefault}

\usepackage[utf8]{inputenc}
\usepackage[T1]{fontenc}
\usepackage{textcomp}
\usepackage{bm}

\usepackage{tikz}
\usetikzlibrary{arrows.meta}
\usetikzlibrary{positioning}
\usetikzlibrary{arrows,shapes}
\usepackage{caption, subcaption}

%% BEAMER BUTTON
%\setbeamertemplate{button}{\tikz
%\node[
%	inner xsep = 2pt, 
%	draw = structure!0, 
%	fill = myblue, 
%	rounded corners = 4pt]{\color{white} \tiny\insertbuttontext};
%}


%% ALGORITHM 
\usepackage{algorithm}
\usepackage[noend]{algpseudocode}
\usepackage{multimedia}

%% THEME
\usetheme[frametitleformat=regular,titleformat=regular]{Madrid}
\setbeamerfont{frametitle}{shape=\normalfont}
\setbeamerfont{title}{shape=\normalfont}

%% PATHS
\graphicspath{{./results/}}
\makeatletter
\def\input@path{{../draft/tables/latexData/}}
\makeatother

%% FIGURE ENVIRONMENT 
\usepackage{booktabs,siunitx}

\usepackage{pgfplots} 
%\usepackage[outdir=./figures]{epstopdf}
\usepackage{epstopdf}
\usepackage{float}
\usepackage{graphicx}

\usepackage[absolute,overlay]{textpos}

%% COLORS
%% LINKS
\definecolor{myred}{RGB}{163,32,45}
\definecolor{navyblue}{rgb}{0.05,0.2,0.70}
\definecolor{myblue}{RGB}{0,51,150}
\definecolor{myorange}{RGB}{255,140,0}
\definecolor{myref}{RGB}{160,160,160}
\definecolor{shock}{RGB}{0, 125, 34}%{50, 168, 82}

%% TRANSPARENCY

\usepackage{transparent}

%% BEAMER TEMPLATE
\usetheme{Boadilla}

\makeatother
\setbeamertemplate{itemize items}{\large\raisebox{-0.25ex}{\textbullet}}
\setbeamertemplate{itemize subitem}{\footnotesize\raisebox{0.15ex}{--}}
\setbeamertemplate{itemize subsubitem}{\Tiny\raisebox{0.7ex}{$\blacktriangleright$}}
%\setbeamertemplate{itemize subsubitem}{\color{yellow}$\blacksquare$}

\setbeamertemplate{enumerate item}[default]
\setbeamertemplate{enumerate subitem}{\textbullet}
\setbeamertemplate{footline}{}
\makeatletter
\setbeamertemplate{navigation symbols}{}


%% FORMATTING AUTHORS
%\usepackage{authblk}
\usepackage{url}
\usepackage{multirow}
\usepackage{array}

%% FRAME ITEMIZE SPACING
\usepackage{xpatch}

\xpatchcmd{\itemize}
{\def\makelabel}
{\setlength{\itemsep}{2.5ex}\def\makelabel}
{}
{}

\xpatchcmd{\enumerate}
{\def\makelabel}
{\setlength{\itemsep}{10ex}\def\makelabel}
{}
{}

%% APPENDIX 
\usepackage{appendixnumberbeamer}


%% TITLE AND OPENING
\title{\large M2: Lecture 7}
\subtitle{Stochastic Dynamic Programming}
\author{Andreas Schaab}
\date{}


\begin{document}
\tikzstyle{every picture}+=[remember picture]
%\everymath{\displaystyle}
\thispagestyle{empty}
\maketitle 
\newpage

\addtocounter{framenumber}{-1}




%%%%%%%%%%%%%%%%%%%%%%%%%%  SLIDE   %%%%%%%%%%%%%%%%%%%%%%%%%%%%%%%%
\begin{frame}{Outline of today's lecture}
\addtocounter{framenumber}{-1}

\begin{enumerate}
\item What is the generator of a stochastic process?
\item Stochastic neoclassical growth model
\item Stochastic neoclassical growth with diffusion process
\item Stochastic neoclassical growth with Poisson process
\item Many examples
\end{enumerate}
\end{frame}



%%%%%%%%%%%%%%%%%%%%%%%%%%  SLIDE   %%%%%%%%%%%%%%%%%%%%%%%%%%%%%%%%
\begin{frame}{1. The generator of a stochastic process}
\begin{itemize}
\item We start with the diffusion process
\begin{equation*}
	dX = \mu(t, X) dt + \sigma (t, X) dB
\end{equation*}
where $dB$ is a standard Brownian motion

\item The generator $\mathcal A$ tells us how the stochastic process is \textit{expected} to evolve

\item The generator $\mathcal A$ is a functional operator

\item Formally, for $f(t, X)$, we have 
\begin{equation*}
	\mathcal A f = \lim_{\Delta t \to 0} \mathbb E_t \frac{ f(t + \Delta t, X(t + \Delta t)) - f(t, X(t)) }{\Delta t}
\end{equation*}

\item We will now show that:
\begin{equation*}
	\mathcal A f = \partial_t f(t, X) + \mu(t, X) \partial_X f(t, X) + \frac{1}{2} \sigma(t, X)^2 \partial_{XX} f(t, X)
\end{equation*}

\item For the general / multi-dimensional version see Oksendal
\end{itemize}
\end{frame}



%%%%%%%%%%%%%%%%%%%%%%%%%%  SLIDE   %%%%%%%%%%%%%%%%%%%%%%%%%%%%%%%%
\begin{frame}{}
\begin{itemize}
\item Next, we consider the poisson process $\{ Y_t \}$ where $Y_t \in \{ Y^1, Y^2 \}$. This is a two-state Markov chain in continuous time.

\item We assume that the Poisson intensity / arrival rate / hazard rate is $\lambda$

\item The generator is now given by
\begin{equation*}
	\mathcal A f (Y^j) = \lambda \Big[ f( Y^{-j} ) - f (Y^j) \Big]
\end{equation*}

\item Intuition: at rate $\lambda$ you transition, so you lose the value of your current state, $f(Y^j)$, and obtain the value of the new state, $f(Y^{-j})$

\item Again see Oksendal for general version of this and more details
\end{itemize}
\end{frame}




%%%%%%%%%%%%%%%%%%%%%%%%%%  SLIDE   %%%%%%%%%%%%%%%%%%%%%%%%%%%%%%%%
\begin{frame}{2. Neoclassical stochastic growth}
\begin{itemize}
\item Time is continuous and the horizon is infinite, $t \in [0, \infty)$

\item Economy populated by representative household that operates production technology $F(k_t, z_t)$ where $z_t$ is exogenous productivity

\item Assume $F(\cdot)$ is well behaved: $F(0, z) = 0$, as well as $F_k$, $F_z > 0$, and $F_{kk} < 0$

\item At time $t = 0$, economy's initial state is $(k_0, z_0)$

\item Lifetime value of household is given by
\begin{equation*}
	V(k_0, z_0) = \max_{ \{c_t\}_{t \geq 0}} \mathbb{E}_0 \int_0^\infty e^{- \rho t} u(c_t) dt 
\end{equation*}
where $u(\cdot)$ is instantaneous utility flow and $\{ c_t \}_{t \geq 0}$ is stochastic consumption process. $\mathbb{E}_0$ denotes expectation over future productivity realizations. We assume labor supply is inelastic and normalized to $1$

\item Capital evolves as before: $\frac{d}{dk} k_t = F(k_t, z_t) - \delta k_t - c_t$
\end{itemize}
\end{frame}



%%%%%%%%%%%%%%%%%%%%%%%%%%  SLIDE   %%%%%%%%%%%%%%%%%%%%%%%%%%%%%%%%
\begin{frame}{3. Productivity as a diffusion process}
\begin{itemize}
\item We start with diffusion process:
\begin{equation*}
	dz_t = - \theta z_t dt + \sigma dB_t,
\end{equation*}
where $\theta$ and $\sigma$ are constants

\item This is a continuous-time, mean-reverting AR(1) process called the Ornstein-Uhlenbeck process

\item State space is now given by
\begin{equation*}
	\Big\{ (k, z) \mid k \in [0, \bar k] \text{ and } z \in [\underline z, \bar z] \Big\}
\end{equation*}
\end{itemize}
\end{frame}


%%%%%%%%%%%%%%%%%%%%%%%%%%  SLIDE   %%%%%%%%%%%%%%%%%%%%%%%%%%%%%%%%
\begin{frame}{}
\begin{itemize}
\item In discrete time, we would have 
\begin{equation*}
	V(k_t, z_t) = \max_c \Big\{ u(c) \Delta t + \frac{1}{1 + \rho \Delta t} \mathbb{E}_t V(k_{t + \Delta t}, z_{t + \Delta t} ) \Big\}
\end{equation*}

\item Difference from previous lecture: $\mathbb E$ because there is uncertainty
\begin{align*}
	(1 + \rho \Delta t) V(k_t, z_t) &= \max_c \Big\{ (1 + \rho \Delta t) u(c) \Delta t+ \mathbb{E}_t V(k_{t + \Delta t}, z_{t + \Delta t})  \Big\} \\
	\rho \Delta t V(k_t, z_t) &= \max_c \Big\{ u(c) \Delta t+ \mathbb{E}_t V(k_{t + \Delta t}, z_{t + \Delta t} ) - V(k_t, z_t) \Big\} \\
	\rho V(k_t, z_t) &= \max_c \Big\{ u(c) + \mathbb{E}_t \frac{V(k_{t + \Delta t}, z_{t + \Delta t} ) - V(k_t, z_t)}{\Delta t} \Big\}
\end{align*}

\item Take limit $\Delta t \to 0$ and drop time subscripts: 
\begin{equation*}
	\rho V(k, z) = \max_c \Big\{ u(c) + \mathbb{E} \frac{d V(k, z)}{dt} \Big\}
\end{equation*}

\item What remains? Characterizing continuation value $\frac{d}{dt} V(k, z)$ (i.e., characterizing how process $dV$ evolves)

\end{itemize}
\end{frame}


%%%%%%%%%%%%%%%%%%%%%%%%%%  SLIDE   %%%%%%%%%%%%%%%%%%%%%%%%%%%%%%%%
\begin{frame}{}
\begin{itemize}
\item The generator $\mathcal A$ is exactly the answer to this question! I.e., 
\begin{align*}
	\mathbb E \frac{d V(k, z)}{d t} &= \mathcal A V(k , z) \\
	&= \Big( F(k, z) - \delta z - c \Big) \partial_k V(k, z) - \theta z \partial_z V(k, z) + \frac{\sigma^2}{2} \partial_{zz} V(k, z)
\end{align*}

\item Therefore, we arrive at the Hamilton-Jacobi-Bellman equation
\begin{align*}
	\rho V(k, z) = \max_c \Big\{ & u(c) + \Big( F(k, z) - \delta z - c \Big) \partial_k V(k, z) \\
	&- \theta z \partial_z V(k, z) + \frac{\sigma^2}{2} \partial_{zz} V(k, z) \Big\}
\end{align*}
with first-order condition 
\begin{equation*}
	u'(c(k, z)) = \partial_k V(k, z)
\end{equation*}
\end{itemize}
\end{frame}



%%%%%%%%%%%%%%%%%%%%%%%%%%  SLIDE   %%%%%%%%%%%%%%%%%%%%%%%%%%%%%%%%
\begin{frame}{4. Productivity as a Poisson process}
\begin{itemize}
\item Next, consider Poisson process for $\{ z_t \}$ with $z_t \in \{z^L, z^H\}$

\item Generator now given by
\begin{equation*}
	\mathcal A V(k, z^j) = \Big( F(k, z) - \delta z - c \Big) \partial_k V(k, z) + \lambda \Big[ V(k, z^{-j}) - V(k, z^j) \Big]
\end{equation*}

\item Note: derivation of HJB exactly as before \textit{up to} characterizing $\mathbb E [d V]$

\item With Poisson process, HJB becomes
\begin{align*}
	\rho V(k, z^j) = \max_c \Big\{ & u(c) + \Big( F(k, z) - \delta z - c \Big) \partial_k V(k, z) \\
	&+ \lambda \Big[ V(k, z^{-j}) - V(k, z^j) \Big] \Big\}
\end{align*}
with first-order condition 
\begin{equation*}
	u'(c(k, z^j)) = \partial_k V(k, z^j)
\end{equation*}
\end{itemize}
\end{frame}



%%%%%%%%%%%%%%%%%%%%%%%%%%  SLIDE   %%%%%%%%%%%%%%%%%%%%%%%%%%%%%%%%
\begin{frame}{5.1. Example: income fluctuations}
\begin{itemize}
\item Economy is populated by representative household that faces income risk

\item Household accumulates wealth according to
\begin{equation*}
	\dot a_t = r a_t + e^{z_t} - c_t 
\end{equation*}
subject to borrowing constraint $a_t \geq 0$

\item Preferences again: $V_0 = \max \mathbb{E}_0 \int_0^\infty e^{- \rho t} u(c_t) dt$

\item Income follows diffusion process: $d y_t = - \theta y_t dt + \sigma dB_t$

\item Away from borrowing constraint, HJB given by
\begin{equation*}
	\rho V = \max_c \Big\{ u(c) + (r a + e^{z} - c) V_a - \theta z V_z + \frac{\sigma^2}{2} V_{zz} \Big\}
\end{equation*}
with $V_a = \partial_a V(a, z)$ (you'll see this often) 
\end{itemize}
\end{frame}


%%%%%%%%%%%%%%%%%%%%%%%%%%  SLIDE   %%%%%%%%%%%%%%%%%%%%%%%%%%%%%%%%
\begin{frame}{5.2. Example: firm profit maximization}
\begin{itemize}
\item Firm maximizes NPV of profit: $V_0 = \max \mathbb{E}_0 \int_0^\infty e^{- r t} \pi_t dt$

\item For now, profit given by: $\pi_t = A_t n_t^\alpha - w_t n_t$ where firm chooses labor $n_t$ \\
Assume $\alpha < 1$, so this is a decreasing-returns production function

\item Firm is small and takes wage $\{ w_t \}$ as given (wages determined in general equilibrium)

\item Productivity follows two-state high-low process, with $A_t \in \{ A^\text{rec}, A^\text{boom} \}$

\item Recursive representation: $A$ is only state variable, $w_t = w(A_t)$
\begin{equation*}
	r V(A^\text{boom}) = \max_n \Big\{ A^\text{boom} n^\alpha - w(A^\text{boom}) n + \lambda \Big[ V(A^\text{rec}) - V(A^\text{boom}) \Big] \Big\}
\end{equation*}
with first-order condition
\begin{equation*}
	n = \bigg(\frac{\alpha A^j}{w(A^j)}\bigg)^\frac{1}{1-\alpha}
\end{equation*}

\end{itemize}
\end{frame}


%%%%%%%%%%%%%%%%%%%%%%%%%%  SLIDE   %%%%%%%%%%%%%%%%%%%%%%%%%%%%%%%%
\begin{frame}{5.3. Example: capital investment with adjustment cost}
\begin{itemize}
\item Firm again maximizes NPV of profit: $V_0 = \max \mathbb{E}_0 \int_0^\infty e^{- r t} \pi_t dt$

\item Now: let $\psi(\cdot)$ denote an adjustment cost
\begin{align*}
	\pi_t &= e^{A_t} k_t^\alpha - Q_t \iota_t - \psi(\iota_t, k_t) \\
	d k_t &= (\iota_t - \delta k_t) dt \\
	d A_t &= - \theta A_t dt + \sigma dB_t
\end{align*}

\item Firm is small and takes capital price as given 

\item Recursive representation in terms of $(k, A)$, i.e., $Q_t = Q(k_t, A_t)$
\begin{align*}
	r V(k, A) = \max_\iota \Big\{ &e^{A_t} k_t^\alpha - Q(A) \iota_t - \psi(\iota_t, k_t) + (\iota - \delta k) \partial_k V(k, A) \\
	& - \theta A \partial_A V(k, A) + \frac{\sigma^2}{2} \partial_{AA} V(k, A) \Big\}
\end{align*}
with first-order condition: $Q(k, A) + \partial_\iota \psi(\iota(k, A), k) = \partial_k V(k, A)$
\end{itemize}
\end{frame}


%%%%%%%%%%%%%%%%%%%%%%%%%%  SLIDE   %%%%%%%%%%%%%%%%%%%%%%%%%%%%%%%%
\begin{frame}{5.4. Example: investing in stocks}
\begin{itemize}
\item Suppose you optimize lifetime utility $V_0 = \mathbb{E}_0 \int_0^\infty u(c_t) dt$

\item You can trade two assets: riskfree bond (return $r dt$), and risky stock 
\begin{equation*}
	dR = (r + \pi) dt + \sigma dB, \text{ where } \pi \text{ is the equity premium}
\end{equation*}

\item You have wealth $a_t$ and invest a share $\theta_t$ in stocks, thus,
\begin{equation*}
	da_t = \theta_t a_t dR_t + (1-\theta_t) a_t r_t dt + y - c_t
\end{equation*}
or, rearranging, and dropping $t$ subscripts
\begin{equation*}
	da = r a + \theta a \pi dt + y - c + \theta a \sigma dB 
\end{equation*}

\item HJB becomes:
\begin{equation*}
	\rho V(a) = \max_{c, \theta} \Big\{ u(c) + (  r a + \theta a \pi dt + y - c  ) V'(a) + \frac{1}{2} (\sigma \theta a)^2 V''(a) \Big\}
\end{equation*}
with FOCs: (i) $u'(c) = V'(a)$ and (ii) $\theta = -\frac{\pi}{\sigma^2} \frac{ V'(a) }{a V''(a)}$
\end{itemize}
\end{frame}



%%%%%%%%%%%%%%%%%%%%%%%%%%  SLIDE   %%%%%%%%%%%%%%%%%%%%%%%%%%%%%%%%
\begin{frame}{5.5. Example: tax competition}
\begin{itemize}
\item Two countries, $i \in \{A, B\}$, setting corporate tax rates $\tau_t^i$ on firms operating / headquartered in country $i$

\item Mass of multinational firms $j$, with $\mu_t$ denoting $\%$ in country $A$ at time $t$

\item Firms relocate activity / headquarters at rate $\theta$ towards low-tax country:
\begin{equation*}
	d \mu_t = \theta \mu_t (\tau_t^B - \tau_t^A)^\gamma dt  
\end{equation*}

\item Country $A$ maximizes tax revenue: $\max \int_0^\infty e^{- \rho t} \tau_t^A \mu_t dt$. Countries compete over taxes $\{ \tau_{it} \}$

\item Dynamic Nash: country $A$ sets $\tau_t^A$ as best response taking $\tau_t^B$ as given 

\item Recursive representation: the only state variable is $\mu_t$
\begin{equation*}
	\rho V^A(\mu) = \max_{\tau^A} \Big\{ \tau^A \mu + \theta \mu \Big( \tau^B(\mu) - \tau^A \Big)^\gamma \partial_\mu V^A(\mu) \Big\}
\end{equation*}
Best response strategies: $0 = \mu + \gamma \theta \mu (\tau^B(\mu) - \tau^A)^{\gamma - 1} V_\mu^A(\mu)$
\end{itemize}
\end{frame}









%%%%%%%%%%%%%%%%%%%%%%%%%%%%%%%%%%%%%%%%%%%%%%%%%%%%%%%%%%%%%%
%%%%%%%%%%%%%%%%%%%%%%%%%%%%%%%%%%%%%%%%%%%%%%%%%%%%%%%%%%%%%%
%%%%%%%%%%%%%%%%%%%%%%%%%%%%%%%%%%%%%%%%%%%%%%%%%%%%%%%%%%%%%%

\appendix


\end{document}









